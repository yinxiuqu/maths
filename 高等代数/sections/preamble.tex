%% 此文件为引导区设置和自定义内容,插入序言区

% 公式按节编号
\numberwithin{equation}{section}
% 图片按节编号
\numberwithin{figure}{section}

% 重定义三大定理环境风格,名称前空两格
\newtheoremstyle{plain}{\topsep}{\topsep}{\citshape}{}{%
  \hspace*{2em}\color{ecolor}\bfseries}{}{%
  0.5em}{%
  \thmname{#1}\thmnumber{ #2}\thmnote{ (#3)}}

\newtheoremstyle{definition}{\topsep}{\topsep}{\citshape}{}{%
  \hspace*{2em}\color{ecolor}\bfseries}{}{%
  0.5em}{}

\newtheoremstyle{remark}{\topsep}{\topsep}{\cnormal}{}{
  \hspace*{2em}\color{ecolor}\bfseries}{}{
  0.5em}{}

% 设置目录居中
\renewcommand\contentsname{\begin{center}\color{ecolor}目\hspace{1em}录\end{center}}

% 重定义proof环境,去掉证明后的实心点变为冒号
\makeatletter
\renewenvironment{proof}[1][\proofname]{\par
\pushQED{\qed}%
\normalfont \topsep6\p@\@plus6\p@\relax
\trivlist
\item\relax
{\hspace{2em}\itshape
#1\@addpunct{:}}\hspace\labelsep\ignorespaces
}{%
  % \rightline{$\square$}\popQED\endtrivlist\@endpefalse
  \popQED\qedsymbol{$\square$}\endtrivlist\@endpefalse
}
\makeatother

%% 定义deduction、theory环境,使其按节编号
\newtheorem{deduction}{推论}[section]
\newtheorem{theory}{引理}[section]
\newtheorem{question}{命题}[section]
\newtheorem{application}{应用}[section]
\newtheorem{property}{性质}[subsection]

%% 定义solution、notice、generalize环境和名称,添加冒号
\makeatletter
\newenvironment{solution}{\par\textbf{\color{ecolor}\solutionname\@addpunct{:}}}{\par}
\newenvironment{notice}{\par\textbf{\color{ecolor}\noticename}}{\par}
\newenvironment{generalize}{\par\textbf{\color{ecolor}\generalizename}}{\par}
\makeatother
\newcommand{\solutionname}{解}
\newcommand{\noticename}{注}
\newcommand{\generalizename}{推广}

% 定义常用的数学符号
\DeclareMathOperator{\st}{{}s.t.{}}
\DeclareMathOperator{\IM}{Im}
\DeclareMathOperator{\rank}{rank}
\DeclareMathOperator{\tr}{tr}
\DeclareMathOperator{\diag}{diag}
\DeclareMathOperator{\Tr}{Tr}


%% 控制编译时只编译某些部分
%\includeonly{./sections/section9}

%%% Local Variables:
%%% mode: latex
%%% TeX-master: "../main"
%%% End:
