\section{二次型}

\subsection{二次型的化简与矩阵的合同}

\begin{theory}
  下列变换都是合同变换,称为对称的初等变换:

  (1)对换$A$的第$i$行与第$j$行,再对换第$i$列与第$j$列,即$P_{ij}AP_{ij}, P'_{ij}=P_{ij}$;

  (2) $A$的第$i$行乘以常数$C$,再将第$i$列乘以常数$C$,即$P_i(C)AP_i(C),P'_i(C)=P_i(C)$;

  (3) $A$的第$i$行乘以常数$C$加到第$j$行上,再将第$i$列乘以常数$C$加到第$j$列上,
  即
  \[T_{ij}(C)AT_{ji}(C), T'_{ij}(C)=T_{ji}(C).\]
\end{theory}

\begin{theory}
  设$A$是数域$\mathbb{K}$上的非零对称阵,则必存在非异阵$C$,
  使$C'AC$的第$(1,1)$元素不等于零.
\end{theory}

\begin{proof}
  设$A=(a_{ij})_{n\times n}$, 若$a_{11}\neq 0$,
  则结论成立;

  下设$a_{11}=0$,若$\exists 2 \leq i \leq n \st a_{ii}\neq 0$,则
  \begin{equation}\label{eq:Congruence1}
    A= 
    \begin{pmatrix}
      0&&&\\
       &\ddots&&\\
       &&a_{ii}&\\
      &&&\ddots
    \end{pmatrix} \xlongrightarrow[P_{ij}AP_{ij}] {\text{第(1)类合同变换}}
    \begin{pmatrix}
      a_{ii}&&&\\
            &\ddots&&\\
            &&0&\\
      &&&\ddots
    \end{pmatrix};
  \end{equation}

  下设$\forall i, a_{ii}=0$,由$A\neq 0$可知$\exists a_{ij}\neq 0, 1\leq i < j \leq n$,则
  \[
    A =
    \begin{pmatrix}
      &\vdots&&\vdots&\\
      \cdots&0&\cdots&a_{ij}&\cdots\\
      &\vdots&&\vdots&\\
      \cdots&a_{ji}&\cdots&0&\cdots\\
      &\vdots&&\vdots&
    \end{pmatrix} \xLongrightarrow [T_{ji}(1)AT_{ij}(1)]{\text{第(3)类合同变换}}
    \begin{pmatrix}
      &\vdots&&\vdots&\\
      \cdots&2a_{ij}&\cdots&a_{ij}&\cdots\\
      &\vdots&&\vdots&\\
      \cdots&a_{ji}&\cdots&0&\cdots\\
      &\vdots&&\vdots&
    \end{pmatrix},
  \]
  再重复\eqref{eq:Congruence1}式的操作即可得结论.
\end{proof}

\begin{theorem}
  设$A$是数域$\mathbb{K}$上的$n$阶对称阵,则必存在$\mathbb{K}$上的$n$阶非异阵$C$,
  使$C'AC$为对角阵.
\end{theorem}

\begin{deduction}
  合同变换是若干次对称初等变换的复合.
\end{deduction}

\begin{proof}
  设矩阵$A$可通过合同变换$C'AC$化为对角阵,
  其中$C$为非异阵.
  \begin{align}
    & \because C \text{为非异阵},
      \text{故可令}C=P_1P_2\cdots P_r, P_i\text{为初等矩阵},\notag\\
    & \therefore C'AC = (P_1P_2\cdots P_r)'A(P_1P_2\cdots P_r)
      = (P'_r\cdots(P'_2(P'_1AP_1)P_2)\cdots P_r)\label{eq:Congruence2}
    \end{align}
  \end{proof}

  \begin{definition}
    下列变换为合同变换,称为分块对称初等变换:

    (1)对换$A$的第$i$分块行与第$j$分块行,
    再对换第$i$分块列与第$j$分块列,
    即$P_{ij}AP'{ij}$;

    (2)第$i$分块行左乘非异阵$M$,
    第$i$分块列右乘$M'$,
    即$P_i(M)AP_i(M'),P'_i(M)=P_i(M')$;

    (3) $A$的第$i$分块行左乘矩阵$M$加到第$j$分块行上,
    第$i$分块列右乘$M'$加到第$j$分块列上,
    即$T_{ij}(M)AT_{ji}(M'), T_{ij}(M)'=T_{ji}(M')$.
  \end{definition}

\subsection{二次型的化简}

\subsubsection{配方法}

\paragraph{基本公式}

\[ (\sum_{i=1}^nx_i)^2=\sum_{i=1}^nx_i^2+2\sum_{1\leq i < j\leq n}x_ix_j
\]

\paragraph{步骤}

(1)将含$x_1$的所有项凑成完全平方式(消去$x_1$);

(2)剩余项中将含$x_2$的所有项凑成完全平方式(消去$x_2$);

\ldots\ldots

(n-1)剩余项中将含含$x_{n-1}$的所有项凑成完全平方式.

\subsubsection{对称初等变换法}

$A$为对称阵,则$\exists$非异阵$C \st C'AC=\Lambda$为对角阵,
$C$的形式如\eqref{eq:Congruence2}所示.

\paragraph{方法}

(1)构造一个$n\times 2n$的矩阵$(\begin{smallmatrix}\begin{BMAT}{c.c}{c}A&I\end{BMAT}\end{smallmatrix})$,
对该矩阵整体实施行变换;

(2)对左边的分块矩阵$A$实施对称的列变换,
直到左边的分块矩阵变为对角阵$\Lambda$,
此时右边的分块阵是过度阵$C$的转置,即$C'$.
\begin{align*}
  & \begin{pmatrix}
    \begin{BMAT}{c.c}{c}
      A&I_n
    \end{BMAT}
  \end{pmatrix} \xlongrightarrow[\text{整体实施行变换}]{\text{整体左乘}P'_1}
  \begin{pmatrix}
    \begin{BMAT}{c.c}{c}
      P'_{1}A&P'_{1}
    \end{BMAT}
  \end{pmatrix}
  \xlongrightarrow[\text{对左边分块矩阵实施对称的列变换}]{\text{仅对左边分块矩阵右乘}P_1}
  \begin{pmatrix}
    \begin{BMAT}{c.c}{c}
      P'_{1}AP_{1}&P'_{1}
    \end{BMAT}
  \end{pmatrix}\\
  & \xlongrightarrow {\text{多次实施对称初等变换}} \cdots \longrightarrow
  \begin{pmatrix}
    \begin{BMAT}{c.c}{c}
      P'_r\cdots P'_2P'_1AP_1P_2\cdots P_r&P'_r\cdots P'_2P'_1
    \end{BMAT}
  \end{pmatrix}
\end{align*}
此时,左边的分块矩阵即为对角阵$\Lambda$,
右边的分块矩阵即为$C'$.

\subsection{惯性定理}

\subsubsection{实二次型}

设是对称阵$A=\diag\{d_1,d_2,\cdots,d_r,0,\cdots,0\}$,
把正项和负项分别放在一起,可设
$d_1>0,\cdots,d_p>0;d_{p+1}<0,\cdots,d_r<0.$
$A$所代表的二次型为
\begin{equation}\label{eq:inertia1}
  f(x_1,x_2,\cdots,x_n)=d_1x_1^2+d_2x_2^2+\cdots+d_rx_r^2.
\end{equation}
令
\[ y_1=\sqrt{d_1}x_11,\cdots,y_p=\sqrt{d_p}x_p; \]
\[ y_{p+1}=\sqrt{-d_{p+1}}x_{p+1},\cdots,y_r=\sqrt{-d_r}x_r; \]
\[ y_j=x_j(j=r+1,\cdots,n), \]
则\eqref{eq:inertia1}式变为
\begin{equation}\label{eq:inertia2}
  f=y_1^2+\cdots+y_p^2-y_{p+1}^2-\cdots-y_r^2.
\end{equation}
该事实等价于$A$合同于下列对角阵:
\begin{equation}\label{eq:inertia3}
  \diag\{1,\cdots,1;-1,\cdots,-1;0,\cdots,0\},
\end{equation}
上式中有$p$个$1$,$q$个$-1$,$n-r$个零.
\eqref{eq:inertia2}式中的二次型称为$f$的{\heiti 规范标准型}.
\begin{theorem}
  设$f(x_1,x_2,\cdots,x_n)$是一个$n$元实二次型,且$f$可化为两个标准型:
  \begin{align*}
    &c_1y_1^2+\cdots+c_py_p^2-c_{p+1}y_{p+1}^2-\cdots-c_ry_r^2,&\\
    &d_1z_1^2+\cdots+d_kz_k^2-d_{k+1}z_{k+1}^2-\cdots-d_rz_r^2,&
  \end{align*}
  其中$c_i>0, d_i>0$,则必有$p=k$.
\end{theorem}

\begin{definition}
  设$f(x_1,x_2\cdots,x_n)$是一个实二次型,若它能化为
  形如\eqref{eq:inertia2}式的形状,则称
  $r$是该二次型的秩, $p$是它的正惯性指数,
   $q=r-p$是它的负惯性指数, $s=p-q$称为$f$的符号差.
 \end{definition}

 \begin{theorem}
   秩与符号差(或正负惯性指数)是是对称阵在合同关系下的全系不变量.
 \end{theorem}

\subsubsection{复二次型}
因为复二次型
\[ f(x_1,x_2,\cdots,x_n)=d_1x_1^2+d_2x_2^2+\cdots+d_rx_r^2 \]
必可化为
\[ z_1^2+z_2^2+\cdots+z_r^2, \]
其中$z_i=\sqrt{d_i}x_i (i=1,2,\cdots,r), z_j=x_j (j=r+1,\cdots,n)$.
所以复对称阵的合同关系只有一个全系不变量,那就是秩$r$.

\subsection{正定型与正定矩阵}

\begin{example}
  \begin{align*}
    & (1) f=x_1^2+x_2^2+\cdots+x_n^2 && \text{正定型}\\
    & (2) f=x_1^2+x_2^2+\cdots+x_r^2(r \leq n) && \text{半正定型}\\
    & (3) f=-x_1^2-x_2^2-\cdots-x_n^2 && \text{负定型}\\
    & (4) f=-x_1^2-x_2^2-\cdots-x_r^2(r \leq n) && \text{半负定型}\\
    & (5) f=x_1^2+\cdots+x_p^2-x_{p+1}^2-\cdots-x_r^2(1\leq p < r \leq n) && \text{不定型}
  \end{align*}
\end{example}

\begin{theorem}\label{thm:PN1}
  设实二次型$f(x_1,\cdots,x_n)$的秩为$r$,正负惯性指数分别为$p, q$,则

  (1) $f$正定 $\Longleftrightarrow$ $p=n$;

  (2) $f$半正定 $\Longleftrightarrow$ $p=r$;

  (3) $f$负定 $\Longleftrightarrow$ $q=n$;

  (4) $f$半负定 $\Longleftrightarrow$ $q=r$;

  (5) $f$不定 $\Longleftrightarrow$ $p>0$且$q>0$.
\end{theorem}

\begin{proof}
  (1)充分性:设$p=n$,则存在非异线性变换
  \[ x=Cy \st f=y_1^2+y_2^2=\cdots+y_n^2. \]
  任取$0 \neq \alpha = (a_1,a_2,\cdots,a_n) \in \mathbb{R}$,从而
  \[ \beta = C^{-1}\alpha\text{是非零实列向量}, \beta=(b_1,b_2,\cdots,b_n)', \]
  则
  \[ f(a_1,a_2,\cdots,a_n)=b_1^2+b_2^2+\cdots+b_n^2 >0 \Longrightarrow f \text{正定}. \]

  必要性:用反证法,设$p<n$,则
  $\exists$非异线性变换$x=Cy$使得
  \[ f=y_1^2+\cdots+y_p^2-y_{p+1}^2-\cdots-y_r^2(p \leq r). \]
  令
  \[b_1=\cdots=b_p=0, b_{p+1}=\cdots=b_n=1, \]
  从而$\beta=(b_1,b_2,\cdots,b_n)$是非零实列向量,故
  \[ \alpha = (a_1,a_2,\cdots,a_n)'=C\beta \neq 0 \text{是非零列向量} , \]
  于是
  \[ f(a_1,a_2,\cdots,a_n)=b_1^2+\cdots+b_p^2-b_{p+1}^2-\cdots-b_r^2 \leq 0, \]
  这与$f$是正定型矛盾.

  (2)(3)(4)(5)可类似证明.
\end{proof}

\begin{theorem}\label{thm:PN2}
  设$A$为实对称阵, $\rank(A)=r$,则

  (1) $A$正定阵 $\Leftrightarrow A$合同于$I_n$;

  (2) $A$半正定阵 $\Longleftrightarrow A$合同于
  $\left(\begin{smallmatrix}
    I_r&O\\O&O
  \end{smallmatrix}\right)$;

(3) $A$负定阵 $\Longleftrightarrow A$合同于$-I_n$;

(4) $A$半负定阵 $\Longleftrightarrow A$合同于
  $\left(\begin{smallmatrix}
    -I_r&O\\O&O
  \end{smallmatrix}\right)$;

(5) $A$不定阵 $\Longleftrightarrow A$合同于
  $\left(\begin{smallmatrix}
    I_p&&\\&-I_q&\\&&O
  \end{smallmatrix}\right)$.
\end{theorem}

\begin{remark}
  \begin{align*}
    & f(A)  \text{负定} && \Longleftrightarrow && -f(-A) \text{正定};\\
    & f(A)  \text{半负定} && \Longleftrightarrow && -f(-A) \text{半正定}.
  \end{align*}
\end{remark}

\begin{theorem}\label{thm:PN3}
  设$A$为$n$阶是对称阵,则
  \[ A\text{正定} \Longleftrightarrow A \text{的$n$个顺序主子式全大于零}. \]
\end{theorem}

\begin{proof}
  必要性:设$A=(a_{ij})$为正定阵,则
  \[ f(x_1,\cdots,x_n)=x'Ax=\sum_{j=1}^n\sum_{i=1}^na_{ij}x_ix_j \]
  为正定型,令
  \[ f_k(x_1,\cdots,x_k) \xlongequal {\text{def}} f(x_1,\cdots,x_k,0,\cdots,0)
    = \sum_{j=1}^k\sum_{i=1}^ka_{ij}x_ix_j
  \]
  $f_k$的相伴对称矩阵为$A_k$,对$\forall 0 \neq (C_1,C_2,\cdots,C_k)' in \mathbb{R}^k$,有
  \[ f_k(C_1,C_2,\cdots,C_k)=f(C_1,\cdots,C_k,0,\cdots,0) > 0, \]
  注意到$(C_1,C_2,\cdots,C_k,0,\cdots,0)'$为非零$n$维列向量,
  由此可知
  \[f_k\text{为正定型} \Longleftrightarrow A_k\text{为正定阵}(1\leq k \leq n).\]
  由定理\ref{thm:PN2}(1)可知,存在非异阵$B_k \in M_k(\mathbb{R}) \st$
  \begin{align*}
    & B'_kA_kB_k=I_k \Longrightarrow 1 = |I_k|=|B'_kA_kB_k|=|B_k|^2\cdot|A_k|\\
    \Longrightarrow & |A_k| > 0 (1 \leq k \leq n)
  \end{align*}

  充分性:设$A$的$n$个顺序主子式全大于零,
  对$n$进行归纳.
  当$n=1$时,$A=(a_{11}) \Longleftrightarrow f=a_{11}x_1^2$,
  由$a_{11}>0$可知$f=a_{11}x_1^2$为正定型,即$A$为正定型.
  设阶数$<n$时结论成立,现证$n$阶的情形.此时$A$可写成如下形式:
  \begin{align*}
    & A =
    \begin{pmatrix}
      A_{n-1}& \alpha\\
      \alpha'&a_{nn}
    \end{pmatrix} \xlongrightarrow {\text{第一分块行乘以$-\alpha'A_{n-1}^{-1}$加到第二分块行}}
    \begin{pmatrix}
      A_{n-1}& \alpha\\
      O&a_{nn}-\alpha'A_{n-1}^{-1}\alpha
    \end{pmatrix}\\
    & \xlongrightarrow {\text{第一分块列乘以$-A_{n-1}^{-1}\alpha$加到第二分块列}}
    \begin{pmatrix}
      A_{n-1}&O\\
      O&a_{nn}-\alpha'A_{n-1}^{-1}\alpha
    \end{pmatrix}=B
  \end{align*}
  $A_{n-1}$的$n-1$个顺序主子式是$A$的前$n-1$个顺序主子式,
  从而它们全大于零,由归纳假设可知$A_{n-1}$为正定阵.
  上式是是合同变换,属于第三类分块对称初等变换,
  所以不改变行列式的值.因此
  \[ |A|=|A_{n-1}|\cdot(a_{nn}-\alpha'A_{n-1}^{-1}\alpha)=|B|. \]
  上式中,
  \[\because |A|>0, |A_{n-1}|>0, 
    \therefore \alpha'A_{n-1}^{-1}\alpha>0. \]
  因为$A_{n-1}$正定,故存在非异阵
  $C_{n-1} \in M_{n-1}(\mathbb{R}) \st C'_{n-1}A_{n-1}C_{n-1}=I_{n-1}$,因此
  \begin{align*}
    & \begin{pmatrix}
      C'_{n-1}&O\\
      O&1
    \end{pmatrix}\cdot B\cdot
    \begin{pmatrix}
      C_{n-1}&O\\
      O&1
    \end{pmatrix}=
    \begin{pmatrix}
      I_{n-1}&O\\
      O&a_{nn}-\alpha'A_{n-1}^{-1}\alpha
    \end{pmatrix}\\
    & \Longrightarrow A \text{的正惯性指数}=n
  \end{align*}
  即$A$为正定阵.
\end{proof}

\begin{example}
  试求$t$的取值范围,使下列二次型为正定型:
  \[ f(x_1,x_2,x_3,x_4)=x_1^2+4x_2^2+4x_3^2+3x_4^2+2tx_1x_2-2x_1x_3+4x_2x_3. \]
\end{example}

\begin{solution}
  该二次型的相伴矩阵为
  \[ A =
    \begin{pmatrix}
      1&t&-1&0\\
      t&4&2&0\\
      -1&2&4&0\\
      0&0&0&3
    \end{pmatrix}. \]
  $A$的顺序主子式为
  \[ |A_1|=1>0,
    |A_2|=
    \begin{vmatrix}
      1&t\\
      t&4
    \end{vmatrix}=4-t^2,
  \]
  \[ |A_3|=
    \begin{vmatrix}
      1&t&-1\\
      t&4&2\\
      -1&2&4
    \end{vmatrix}=-4(t-1)(t+2),
    |A_4|=-12(t-1)(t+2). \]
  要使$A$正定,必须
  \[ \begin{cases}
    4-t^2>0\\
    -4(t-1)(t+2)>0
  \end{cases} \]
  解得$-2<t<1$.
\end{solution}

\begin{deduction}\label{ddn:PN1}
  设$A$为$n$阶正定实对称阵,则

  (1) $A$的所有主子阵都是正定阵;

  (2) $A$的所有主子式全大于零,特别地,主对角元素全大于零;

  (3) $A$中之元素绝对值最大值只出现在主对角线上.
\end{deduction}

\begin{proof}
  (1)取$A$中第$i_1,i_2,\cdots,i_k$行、列构成主子阵$B$,
  设$A$对应的相伴二次型为
  \[f(x_1,\cdots,x_n)=x'Ax, \]
  令
  \[ f_B(x_{i_1},x_{i_2},\cdots,x_{i_k}) \xlongequal {\text{def}}
    f(0,\cdots,0,x_{i_1},0,\cdots,0,x_{i_2},0,\cdots,0,x_{i_k}). \]
  由$f$正定$\Longrightarrow f_B$正定$\Longrightarrow$相伴矩阵$B$正定.

  (2)是(1)的推论.

  (3)用反证法:设绝对值最大的元素为$|a_{ij}|$,其中$i<j$.
  考虑$2$阶主子式
  \begin{equation}\label{eq:PN1}
    A
    \begin{pmatrix}
      i&j\\
      i&j
    \end{pmatrix}=
    \begin{vmatrix}
      a_{ii}&a_{ij}\\
      a_{ji}&a_{jj}
    \end{vmatrix}=a_{ii}a_{jj}-a_{ij}^2,
  \end{equation}
  \[ \because |a_{ij}| \geq a_{ii}>0, |a_{ij}| \geq a_{jj}>0,
    \therefore \text{\eqref{eq:PN1}式小于零.} \]
  这和(2)矛盾.
\end{proof}

\begin{deduction}\label{ddn:PN2}
  设$A$为$n$阶正定实对称阵,则$A$的特征值全大于零.
\end{deduction}

\begin{proof}
  {\heiti 事实:}设
  \[
    \xi = (a_1,a_2,\cdots,a_n)'\neq 0 \in \mathbb{C}^n,
  \]
  则
  \[
    \overline{\xi'}\cdot\xi=(\overline{a_1},\overline{a_2},\cdots,\overline{a_n})
      \begin{pmatrix}
        a_1\\a_2\\\vdots\\a_n
      \end{pmatrix}=\sum_{i=1}^n|a_i|^2 > 0.
    \]
    任取$A$的特征值$\lambda$,对应特征向量为$\xi$.
    \[ A\text{正定} \Longrightarrow
      \exists\text{非异阵}C \in M_n(\mathbb{R}) \st
      A = C'I_nC = C'C. \]
    \[ A\xi =\lambda_0\xi \Longrightarrow
      \lambda_0\overline{\xi'}\xi = \overline{\xi'}A\xi = \overline{\xi'}C'C\xi
      =\overline{(C\xi)'}(C\xi).
    \]
    \[
      \xi \neq 0 \Longrightarrow \overline{\xi'}\xi >0,  
      C\xi \neq 0 \Longrightarrow \overline{(C\xi)'}(C\xi) > 0
      \Longrightarrow \lambda_0 = \overline{(C\xi)'}(C\xi)/\overline{\xi'}\xi > 0.
     \]
   \end{proof}

   \begin{theorem}
     设$A$为$n$阶实对称阵,则下列结论等价:

     (1) $A$正定;

     (2) $A$合同于$I_n$;

     (3) $\exists$非异阵$C \in M_n(\mathbb{R}) \st A=C'C$;

     (4) $A$的所有主子式全大于零;

     (5) $A$的$n$个顺序主子式全大于零;

     (6) $A$的特征值全大于零.
   \end{theorem}

   \begin{proof}
     (1) $\xLongleftrightarrow {\text{定理\ref{thm:PN2}}}$ (2)
     $\xLongleftrightarrow {\text{定义}}$ {3}

     (1) $\xLongrightarrow {\text{推论\ref{ddn:PN1}}}$ (4)
     $\Longrightarrow$ (5) $\xLongrightarrow {\text{定理\ref{thm:PN3}}}$ (1)

     (1) $\xLongrightarrow {\text{推论\ref{ddn:PN2}}}$ (6) $\xLongrightarrow {\text{第九章讲}}$ (1)
   \end{proof}

\subsection{半正定型和半正定矩阵}

\begin{theory}\label{thr:SP1}
  设$A$是$n$阶实对称阵,则
  \[ A\text{半正定} \Longleftrightarrow
    \exists C \in M_n(\mathbb{R}) \st A=C'C. \]
\end{theory}

\begin{proof}
  先证充分性:由定理\ref{thm:PN2}(2)可知,
  \[ A\text{合同于}
    \begin{pmatrix}
      I_r&O\\
      O&O
    \end{pmatrix}, r=\rank(A), \]
  即$\exists$非异阵$B \in M_n(\mathbb{R}) \st$
  \[ A=B'
    \begin{pmatrix}
      I_r&O\\
      O&O
    \end{pmatrix}B. \]
  令
  \[ C=
    \begin{pmatrix}
      I_r&O\\
      O&O
    \end{pmatrix}B,
  \]
  则$A=C'C$.

  再证必要性:设$A=C'C$,对$0 \neq \alpha \in \mathbb{R}^n$有
  \[ \alpha'A\alpha = \alpha'C'C\alpha=(C\alpha)'(C\alpha). \]
  设$C\alpha =
  \begin{pmatrix}
    a_1\\a_2\\\vdots\\a_n
  \end{pmatrix} \in \mathbb{R}
  $,则上式可写为:
  \[ \alpha'A\alpha = a_1^2+a_2^2+\cdots+a_n^2 \geq 0. \]
\end{proof}

\begin{theory}\label{thr:SP2}
  设$A$为$n$阶实对称阵,则
  \[ A\text{半正定} \Longleftrightarrow
  \forall t \in \mathbb{R}^+, A+tI_n\text{为正定阵}.\]
\end{theory}

\begin{proof}
  先证充分性: $\forall 0 \neq \alpha \in \mathbb{R}^n$,有
  \[ \alpha'(A+tI_n)\alpha = \alpha'A\alpha+t\alpha'\alpha. \]
  \[
    \because \alpha'A\alpha \geq 0, t\alpha'\alpha>0,
    \therefore \alpha'(A+tI_n)\alpha > 0. \]

  再证必要性: $\forall 0 \neq \alpha \in \mathbb{R}^n$,有
  \[ \alpha'(A+tI_n)\alpha = \alpha'A\alpha + t\alpha'\alpha >0. \]
  令$t \longrightarrow 0^+$,则$\alpha'A\alpha \geq 0$.
\end{proof}

\begin{example}
  $A=\diag\{1,0,-1\}, |A_1|=1, |A_2|=|A_3|=0$,
  不能推出$A$为半正定阵.
\end{example}

\begin{theory}\label{thr:SP3}
  设$A$是$n$阶实对称阵,则
  \[ A \text{半正定} \Longleftrightarrow
    A \text{的所有主子式全大于或等于零}. \]
\end{theory}

\begin{proof}
  先证充分性:令
  \[ f(x_1,x_2,\cdots,x_n)=x'Ax,  \]
  则$f$为半正定型.
  取$A$第$i_1,\cdots,i_r$行,列交叉处的元素构成的主子阵$B$,
  构造二次型:
  \[ f_B(x_{i_1},\cdots,x_{i_r})=f(0,\cdots,0,x_{i_1},0,\cdots,0,x_{i_2},0,\cdots,0,x_{i_r}).
  \]
  \[
    f\text{半正定} \Longrightarrow f_B\text{半正定} \Longleftrightarrow B\text{半正定}
    \Longrightarrow A
    \begin{pmatrix}
      i_1&\cdots&i_r\\
      i_1&\cdots&i_r
    \end{pmatrix}=|B| \geq 0.
  \]

  再证必要性:
  \[
    |A+tI_n|=t^n+C_1t^{n-1}+\cdots+C_{n-1}t+C_n,
  \]
  其中$C_i$为$A$ 的所有$i$阶主子式之和,因此
  \[
    C_i \geq 0, \forall 1 \leq i \leq n \Longrightarrow
    \forall t \in \mathbb{R}^+, |A+tI_n|>0.
  \]
  设$A_k$为$A$的顺序主子式$(1 \leq k \leq n)$,
  可得$A_k$的所有主子式都大于或等于零,
  重复上述计算,可得
  \[
    |A_k+tI_k|>0, \forall 1 \geq k \geq n,\forall t \in \mathbb{R}^+
    \Longrightarrow A+tI_n\text{是正定阵}, \forall t \in \mathbb{R}^+
    \xLongrightarrow {\text{引理\ref{thr:SP2}}} A\text{是半正定阵}.
    \]
  \end{proof}

\begin{theory}\label{thr:SP4}
  设$A$为$n$阶实对称阵,则
  \[
    A\text{半正定} \Longleftrightarrow A \text{的所有特征值全大于或等于零}.
    \]
\end{theory}

\begin{proof}
  设$A$的特征值为$\lambda_1,\lambda_2,\cdots,\lambda_n$.
  \begin{align*}
    A\text{半正定}  \xLongleftrightarrow {\text{引理\ref{thr:SP2}}} &
    A+tI_n\text{正定},\forall t \in \mathbb{R}+\\
                   \Longleftrightarrow &  \lambda_i+t>0, \forall 1\leq i \leq n, \forall t >0\\
    \Longrightarrow & \lambda_i \geq 0, \forall 1 \geq i \geq n.
    \end{align*}
\end{proof}

\begin{theorem}
  设$A$为$n$阶实对称阵,则下列结论等价:

  (1) $A$半正定;

  (2) $A$合同于
  $\begin{pmatrix}
    I_r&O\\
    O&O
  \end{pmatrix}$;

  (3) $\exists C \in M_n(\mathbb{R}) \st A=C'C$;

  (4) $A$的所有主子式全部大于或等于零;

  (5) $A$的所有特征值全大于或等于零.
\end{theorem}

\begin{property}
  设$A=(a_{ij})$为半正定阵,若$a_{ii}=0$,则$A$的第$i$行和第$i$列元素全为零.
\end{property}

\begin{proof}
  任取$a_{ij},a_{ji} (i\neq j)$所在的二阶主子式,即
  \[
    \begin{vmatrix}
      a_{ii}&a_{ij}\\
      a_{ji}&a_{jj}
    \end{vmatrix} = -a_{ij}^2 \geq 0,
  \]
  上式中$A_{ii}=0$,从而$
  \begin{cases}
    a_{ij}=0\\
    a_{ji}=0
  \end{cases}
  $
\end{proof}

\begin{property}
  $A$为半正定阵,
  $\alpha \in \mathbb{R}^n \st \alpha'A\alpha=0$,
  则$A\alpha=0$.
\end{property}

\begin{proof}
  $\exists C \in M_n(\mathbb{R}) \st A=C'C$,所以
  \[
    0 = \alpha'A\alpha = \alpha'C'C\alpha =(C\alpha)'(C\alpha)
    \Longrightarrow C\alpha=0
    \Longrightarrow A\alpha=C'C\alpha =C'(C\alpha)=0.
    \]
\end{proof}

\subsection{Hermite型}
二次型
\[ f(x_1,x_2,\cdots,x_n)=d_1x_1^2+d_2x_2^2+\cdots+d_rx_r^2, \]
其中$d_1,d_2,\cdots,d_r \in \mathbb{C}$,
$x_1,x_2,\cdots,x_n$为复变元.
则$\forall (a_1,a_2,\cdots,a_n)'\in \mathbb{C}^2$,
\[ f(a_1,a_2,\cdots,a_n)=d_1a_1^2+d_2a_2^2+\cdots+d_ra_r^2 \in \mathbb{C}. \]

做如下调整:

(1)加限制性条件,使得$d_1,d_2,\cdots,d_r$为实数;

(2)将$x_i^2$变为$|x_i|^2=\overline{x_i}x_i$.

此时$f \in \mathbb{R}$,可写成如下形式:
\[
  f(x_1,x_2,\cdots,x_n)=d_1\overline{x_1}x_1+d_2\overline{x_2}x_2+\cdots+d_r\overline{x_r}x_r.
\]
调整后, $f$已经不是多项式了,不是二次型,而是Hermite型.

\begin{definition}
  下列$n$个复变元的二次齐次函数称为Hermite型:
  \[
    f(x_1,x_2,\cdots,x_n) = \sum_{j=1}^n\sum_{i=1}^na_{ij}\overline{x_i}x_j,
  \]
  其中$\overline{a_{ij}}=a_{ji} ( \forall 1 \leq i, j \leq n)$.
  此时, $f$为复变元实值函数.
\end{definition}

\begin{theory}
  定义从$\{n\text{阶Hermite阵}\}\longrightarrow \{n\text{元Hermite型}\}$
  的映射$\varphi$:
  \[
    A \mapsto f=\overline{x'}Ax,
  \]
  求证:$\varphi$为一一映射.
\end{theory}

\begin{proof}
  每个Hermite矩阵的各个元素都可写成一个Hermite型的对应项系数,
  因此可知$\varphi$显然为满射.下证$\varphi$为单射:

  设$A, B$为Hermite阵且
  $\overline{x'}Ax=\overline{x'}Bx$,则
  \begin{equation}\label{eq:Hermite1}
    \overline{x'}(A-B)x=0.
  \end{equation}
  令$C=(c_{ij})=A-B$,
  $x$可在复数域内取任意值\eqref{eq:Hermite1}式都成立,因此可令$x$分别取下列值:
  \begin{align*}
    & x=
    \bordermatrix{
      ~&1\cr
         1&0\cr
            ~&\vdots\cr
               i&1\cr
                  ~&\vdots\cr
                     n&0} \Longrightarrow C_{ii}=0 (\forall i),
    x=
    \bordermatrix{
      ~&1\cr
         1&0\cr
            ~&\vdots\cr
               i&1\cr
                  ~&\vdots\cr
                     j&1\cr
                        ~&\vdots\cr
                           n&0} \Longrightarrow
                              C_{ij}+C_{ji}=0 (\forall i,j),\\
    & x=
    \bordermatrix{
      ~&1\cr
         1&0\cr
            ~&\vdots\cr
               i&1\cr
                  ~&\vdots\cr
                     j&\sqrt{-1}\cr
                        ~&\vdots\cr
                           n&0} \Longrightarrow
                              C_{ij}-C_{ji}=0 (\forall i,j).
  \end{align*}
  由以上可知, $A-B=C=0$,因此, $A=B$.
\end{proof}

\begin{theorem}
  设$A$为Hermite阵,
  则$\exists$非异阵$C \in \mathbb{C}^n \st \overline{C'}AC$为是对角阵.
\end{theorem}

\begin{proof}
  共轭对称初等变换为负相合变换:
  (1)对换第$i$行与第$j$行,对换第$i$列与第$j$列:
  \[
    P_{ij}AP_{ij}, \overline{P'_{ij}}=P_{ij};
  \]

  (2)第$i$行乘以非零复数$C$,第$i$列乘以$\overline{C}$:
  \[
    P_i(C)AP_i(\overline{C}), \overline{P_i(C)'}=P_i(\overline{C});
  \]

  (3)第$i$行乘以复数$C$加到第$j$行上,
  再将第$i$列乘以$\overline{C}$加到第$j$列上:
  \[
    T_{ij}(C)AT_{ji}(\overline{C}), \overline{T_{ij}(C)'}=T_{ji}(\overline{C}).
  \]

  {\heiti 证明概要:} 令$A=(a_{ij})$.

  (I)若$a_{11}\neq 0$, 则符合要求.

  (II)若$a_{11}=0, a_{ii}\neq 0 (2\leq i \leq n)$,
  则经过$P_{1i}AP_{1i}$之后,$a_{11}\neq 0$.

  (III)若$a_{ii}=0(\forall 1 \leq i \leq n)$,任取$a_{ij}\neq 0(i<j)$.
  \[
    A=\bordermatrix{
      ~&i&j\cr
      i&0&a_{ij}\cr
      j&\overline{a_{ij}}&0}
    \xlongrightarrow [\text{第$j$行乘以$a_{ij}$再加到第$i$行上}]
    {\text{第$j$列乘以$\overline{a_{ij}}$再加到第$i$列上}} \bordermatrix{
      ~&i&j\cr
      i&2|a_{ij}|^2&a_{ij}\cr
      j&\overline{a_{ij}}&0}
  \]
  此时$a_{ii}$已不为零,再通过上述第(III)步调整,可使$a_{11}\neq 0$.
  $A$可变为下列形式:
  \[
    A=
    \begin{pmatrix}
      a_{11}&a_{12}&\cdots&a_{1n}\\
      a_{21}&&&\\
      \vdots&&*&\\
      a_{n1}&&&
    \end{pmatrix}
    \xlongrightarrow[\text{第$1$列分别乘以适当常数后加到第$2-n$列上}]{\text{第$1$行分别乘以适当常数后加到第$2-n$行上}}
      \begin{pmatrix}
        a_{11}&0&\cdots&0\\
        0&&&\\
        \vdots&&A_{n-1}&\\
        0&&&
      \end{pmatrix}
    \]
    上式中, $A_{n-1}$为$n-1$阶Hermite阵,
    再用归纳法即可得证.
\end{proof}

\begin{theorem}[惯性定理]
  设$f(x_1,x_2,\cdots,x_n)$是一个Hermite型,则它总可化为如下标准型:
  \[
    \overline{y_1}y_1+\cdots+\overline{y_p}y_p-
    \overline{y_{p+1}}y_{p+1}-\cdots-\overline{y_r}y_r,
  \]
  且若$f$又可化为另一个标准型:
  \[
    \overline{z_1}z_1+\cdots+\overline{z_k}z_k-
    \overline{z_{k+1}}z_{k+1}-\cdots-\overline{z_r}z_r,
  \]
  则$p=k$.
\end{theorem}

\begin{theorem}
  秩$r$与符号差$s$(或正负惯性指数p,q)是Hermite型(阵)在复相合关系下的全系不变量.
\end{theorem}

\begin{definition}
  $f(x_1,\cdots,x_n)=\overline{x'}Ax$是hermite型,
  其相伴Hermite矩阵为$A$, $\forall 0 \neq \alpha, \beta \in \mathbb{C}^n$,
  则有如下定义和结论:
  \begin{align*}
    &\overline{\alpha'}A\alpha & f &&A\\
    &>0&\text{正定}&&\text{正定}\\
    &\geq 0&\text{半正定}&&\text{半正定}\\
    &<0&\text{负定}&&\text{负定}\\
    &\leq 0&\text{半负定}&&\text{半负定}\\
    &\overline{\alpha'}A\alpha>0, \overline{\beta'}A\beta<0 & \text{不定型} && \text{不定型}
  \end{align*}
\end{definition}

\begin{theorem}
  \begin{align*}
    & (1) f\text{正定} &\Longleftrightarrow && p=n && A\text{正定} & \xLongleftrightarrow{\text{复相合于}} I_n;\\
    & (2) f\text{半正定}& \Longleftrightarrow && p=r && A\text{半正定} & \xLongleftrightarrow{\text{复相合于}}
                                                       \begin{pmatrix}
                                                         I_r&O\\
                                                         O&O
                                                       \end{pmatrix};\\
    & (3) f\text{负定} &\Longleftrightarrow && q=n && A\text{负定} & \xLongleftrightarrow{\text{复相合于}} -I_n;\\
    & (4) f\text{半负定} &\Longleftrightarrow && q=r && A\text{半负定} & \xLongleftrightarrow{\text{复相合于}}
                                                       \begin{pmatrix}
                                                         -I_r&O\\
                                                         O&O
                                                       \end{pmatrix};\\
    & (5) f\text{不定} &\Longleftrightarrow && p>0\text{且}q>0 && A\text{不定} & \xLongleftrightarrow{\text{复相合于}}
                                                                 \begin{pmatrix}
                                                                   I_p&&\\
                                                                   &-I_q&\\
                                                                          &&O
                                                                 \end{pmatrix}.
  \end{align*}
\end{theorem}

\begin{remark}
  \[
    f(A)\text{负定(半负定)} \Longleftrightarrow -f(-A)\text{正定(半正定)}.
    \]
\end{remark}

\begin{theorem}
  设$A$为Hermite阵,则下列结论等价:

  (1) $A$正定;

  (2) $A$复相合于$I_n$;

  (3) $\exists$非异阵$C \in M_n(\mathbb{C}) \st A=\overline{C'}C$;

  (4) $A$的所有主子式全大于零;

  (5) $A$的$n$个顺序主子式全大于零;

  (6) $A$的特征值全大于零.
\end{theorem}

\begin{theorem}
  设$A$为Hermite阵,则下列结论等价:

  (1) $A$半正定;

  (2) $A$复相合于$
  \begin{pmatrix}
    I_r&O\\
    O&O
  \end{pmatrix}
  $;

  (3) $\exists C \in M_n(\mathbb{C}) \st A=\overline{C'}C$;

  (4) $A$的所有主子式全大于或等于零;

  (5) $A$的特征值全大于或等于零.
\end{theorem}
%%% Local Variables:
%%% mode: LaTeX
%%% TeX-master: "../main"
%%% End:
