\section{相似标准型}
\subsection{多项式矩阵}
\begin{definition}
  $M(\lambda)=M_m\lambda^m+M_{m-1}\lambda^{m-1}+\cdots+M_1\lambda+M_0$,
  其中$M_i \in M_{r\times s}(\mathbb{K})$,称$M(\lambda)$为{\heiti 矩阵多项式}.
  若$M_m\neq 0$,定义$\deg M(\lambda)=m$, 约定$\deg 0=-\infty$.
  \begin{align*}
    & N(\lambda)=N_n\lambda^n+N_{n-1}\lambda^{n-1}+\cdots+N_1\lambda+N_0, N_n \neq 0, \text{则}\\
    & M(\lambda)=N(\lambda) \xLongleftrightarrow {\text{def}} m=n, M_i=N_i, \forall 0 \leq i \leq m
  \end{align*}
  设$m \geq n$,\\
  定义加法:$M(\lambda)+N(\lambda)=M_m\lambda^m+\cdots+M_{n+1}\lambda^{n+1}
  +(M_n+N_n)\lambda^n+\cdots+(M_0+N_n)$\\
  定义数乘:$k\cdot M(\lambda)=k\cdot M_m\lambda^m+\cdots+k\cdot M_0$\\
  定义乘法:$M(\lambda)\cdot N(\lambda)=P(\lambda)=\sum P_k\lambda^k,
  P_k=\sum\limits_{i+j=k}M_iN_j$
\end{definition}
\begin{theory}
  $\deg(M(\lambda)\cdot N(\lambda))\leq \deg M(\lambda) + \deg N(\lambda)$,
  若$M_m$或$N_n$是可逆阵,则等号成立.
\end{theory}
\begin{proof}
  $M(\lambda)\cdot N(\lambda)$的首项是$M_m\cdot N_n\cdot \lambda^{m+n}$,若
  $M_m$或$N_n$可逆,则$M(\lambda)\cdot N(\lambda)\neq 0$.
\end{proof}
\begin{theorem}\label{thm:model-main1}
  设$A,B$是数域$\mathbb{K}$上的矩阵,则$A,B$相似的充分必要条件是
  $\lambda-$矩阵$\lambda I-A$与$\lambda I-B$相抵.
\end{theorem}

\subsection{矩阵的法式}
\begin{theorem}\label{thm:model1}
  设$A(\lambda)$是一个$n$阶$\lambda-$矩阵,则$A(\lambda)$相抵于对角阵
  \begin{equation*}
    \diag\{d_1(\lambda),d_1(\lambda),\cdots,d_r(\lambda);0,\cdots,0\},
  \end{equation*}
  其中$d_i(\lambda)$是非零首一多项式且$d_i(\lambda)\mid d_{i+1}(\lambda) (i=1,2,\cdots,r-1)$.
\end{theorem}
\begin{theory}\label{thr:model1}
  设$A(\lambda),B(\lambda)$为$n$阶$\lambda-$矩阵,则:

  (1) $|A(\lambda)\cdot B(\lambda)|=|A(\lambda)|\cdot |B(\lambda)|$;

  (2) $A(\lambda)A(\lambda)^*=A(\lambda)^*A(\lambda)=|A(\lambda)|\cdot I_n$.
\end{theory}
\begin{proof}
  (1)令$f(\lambda)=|A(\lambda)\cdot B(\lambda)|-|A(\lambda)|\cdot |B(\lambda)|$
  是$\lambda$的多项式,\\
  对任意的$a \in \mathbb{K}$,
  $f(a)=|A(a)\cdot B(a)|-|A(a)|\cdot |B(a)| = 0$,
  从而$f(\lambda) \equiv 0$.

  (2) 令$(f_{ij}(\lambda))_{n\times n}=A(\lambda)\cdot A(\lambda)^* -
  |A(\lambda)|\cdot I_n, f_{ij}(\lambda) \in \mathbb{K}[\lambda]$, \\
  $\forall a \in \mathbb{K},
  (f_{ij}(a)) = A(a)\cdot A(a)^* -
  |A(a)|\cdot I_n = 0 \Longrightarrow f_{ij}(\lambda) \equiv 0, \forall i,j$
\end{proof}
\begin{theorem}
  设$A(\lambda)$是$n$阶$\lambda-$矩阵,
  则下列结论等价:

  (1) $A(\lambda)$是可逆$\lambda-$阵;

  (2) $|A(\lambda)|$是非零常数;

  (3) $A(\lambda)$的相抵标准型为$I_n$;

  (4) $A(\lambda)$只通过初等行(列)变换可变为$I_n$;

  (5) $A(\lambda)$是初等$\lambda-$矩阵的乘积.
\end{theorem}
\begin{proof}
  (1) $\Longrightarrow$ (2): 
  $\exists B(\lambda) \st A(\lambda)B(\lambda)=B(\lambda)A(\lambda)=I_n$,\\
  等式两边同取行列式,得:
  $|A(\lambda)\cdot B(\lambda)|=|A(\lambda)|\cdot |B(\lambda)|=|I_n|=1$,\\
  因为上式左边是两个关于$\lambda$的多项式相乘,右边是1,\\
  因此$|A(\lambda)|$和$|B(\lambda)|$只能是非零常数.

  (2) $\Longrightarrow$ (3): 由定理\ref{thm:model1}
  $\Longrightarrow \exists \lambda-$矩阵$P(\lambda),Q(\lambda)
  \st P(\lambda)A(\lambda)Q(\lambda)=
  \diag \{d_1(\lambda),d_2(\lambda),\\\cdots,d_r(\lambda);0,\cdots,0\}$,
  其中, $d_i(\lambda)$皆为关于$\lambda$的首一多项式,且$d_1(\lambda)\mid
  d_2(\lambda) \mid \cdots \mid d_r(\lambda)$,
  $P(\lambda)$,\\$Q(\lambda)$为初等$\lambda-$矩阵的乘积.
  由引理\ref{thr:model1}可知,
  \[|P(\lambda)|\cdot|A(\lambda)|\cdot|Q(\lambda)|=
    d_1(\lambda)d_2(\lambda)\cdots d_r(\lambda)0\cdots 0\]
  上式左边,$P(\lambda)$,$Q(\lambda)$为初等$\lambda-$矩阵的乘积,
  所以皆不为零,又$|A(\lambda)|$为非零常数,故等式左边不等于零,
  从而等式右边的$n-r$个0均不会出现,因此$r=n$且$d_i(\lambda)$都是非零常数.
  又因为$d_i(\lambda)$皆是首一多项式,故对$\forall i, d_i(\lambda)=1$,
  由此可推出$A(\lambda)=I_n$.

  (3) $\Longrightarrow$ (4): 设$\exists P(\lambda), Q(\lambda)$
  是初等$\lambda-$矩阵的乘积$\st$
  \begin{align*}
    & P(\lambda)A(\lambda)Q(\lambda)=I_n\\
      \Longrightarrow & P(\lambda)A(\lambda)=Q(\lambda)^{-1}\\
    \Longrightarrow & Q(\lambda)P(\lambda)A(\lambda)=I_n
  \end{align*}
  即$A(\lambda)$只需通过初等行变换即可变为$I_n$.
  同理可证$A(\lambda)$只需通过初等列变换即可变为$I_n$.

  (4) $\Longrightarrow$ (5): 设$P(\lambda)$是初等$\lambda-$矩阵$\st$
  \begin{align*}
    & P_r(\lambda)P_{r-1}(\lambda)\cdots P_1(\lambda)A(\lambda)=I_n\\
    \Longrightarrow & P_{r-1}(\lambda)P_{r-2}(\lambda)\cdots P_1(\lambda)A(\lambda)=P_r(\lambda)^{-1}\\
    & \cdots\cdots\\
    \Longrightarrow & A(\lambda) = P_1(\lambda)^{-1}P_2(\lambda)^{-1}\cdots P_r(\lambda)^{-1}\\
    \Longrightarrow & A(\lambda)\text{是初等}\lambda-\text{矩阵的乘积}.
  \end{align*}

  (5) $\Longrightarrow$ (1): 初等$\lambda-$矩阵都是可逆阵,且有限个可逆阵的乘积仍然是可逆阵,因此推出(1),
  此时,
  \[A(\lambda)^{-1}=\frac{1}{|A(\lambda)|}\cdot A(\lambda)^*.\]
  \begin{proof}
    由引理\ref{thr:model1}可知,
    \begin{align*}
      & A(\lambda)\cdot A(\lambda)^*=A(\lambda)^*A(\lambda)=|A(\lambda)|\cdot I_n\\
      \xLongrightarrow {A(\lambda)\text{可逆故}|A(\lambda)\neq 0|}&
        A(\lambda)(\frac{1}{|A(\lambda)|}A(\lambda)^*)=
        (\frac{1}{|A(\lambda)|}A(\lambda)^*)A(\lambda)=I_n\\
      \Longrightarrow & A(\lambda)^{-1}=\frac{1}{|A(\lambda)|}A(\lambda)^*
    \end{align*}
  \end{proof}
\end{proof}
\begin{theorem}\label{thm:model3}
  设$A \in M_n(\mathbb{K})$,则$\lambda-A$相抵于
  \begin{equation*}
    \diag\{1,\cdots,1,d_1(\lambda),\cdots,d_m(\lambda)\},
  \end{equation*}
  其中,$d_i(\lambda)$为非常数首一多项式且
  $d_1(\lambda)\mid d_2(\lambda) \mid\cdots\mid d_m(\lambda)$.
\end{theorem}
\begin{proof}
  由定理\ref{thm:model1}可知,$\exists P(\lambda),Q(\lambda)$为
  可逆$\lambda-$矩阵$\st$
  \begin{equation*}
    P(\lambda)(\lambda I_n-A)Q(\lambda)=
    \diag\{d_1(\lambda),d_2(\lambda),\cdots,d_r(\lambda),0,\cdots,0\}
  \end{equation*}
  两边同时取行列式,得
  \begin{equation*}
    c\cdot|\lambda I_n-A|=
    d_1(\lambda)d_2(\lambda)\cdots d_r(\lambda)0\cdots 0,c \neq 0
  \end{equation*}
  上式左边非零,所以右边也不等于零,故右边的$n-r$个零均不会出现,
  从而$r=n$且
  \begin{align}
    & c|\lambda I_n-A|=d_1(\lambda)d_2(\lambda)\cdots d_n(\lambda)\notag\\
    \Longrightarrow & c=1\notag\\
    \Longrightarrow & |\lambda I_n-A|=d_1(\lambda)d_2(\lambda)\cdots d_n(\lambda)\label{eq:model1}
  \end{align}
  把$d_i(\lambda)$中的$1$都列出,即可得到结论.
\end{proof}
\begin{example}
  若$\deg d_i(\lambda) \geq 1$,则$A=cI_n$.
\end{example}
\begin{proof}
  由\eqref{eq:model1}两边取多项式次数,得
  \begin{equation*}
    n = \sum\limits^n_{i=1}\deg d_i(\lambda) \geq n
  \end{equation*}
  从而$d_1(\lambda),d_2(\lambda),\cdots,d_n(\lambda)$是一次首一多项式.
  又由整除条件可知,
  \begin{equation*}
   d_1(\lambda)=d_2(\lambda)=\cdots=d_n(\lambda)=\lambda-c 
 \end{equation*}
 即$\lambda I_n-A$相抵于矩阵
 \begin{equation*}
  \diag\{\lambda-c,\lambda-c,\cdots,\lambda-c\}=\lambda I_n-cI_n, 
 \end{equation*}

 由定理\ref{thm:model-main1}可知,$A$与$cI_n$相似,从而$cI_n=P^{-1}AP\Longrightarrow A=cI_n$.
\end{proof}
\begin{example}
  已知$A=
  \begin{pmatrix}
    0 & 1 & -1\\
    3 & -2 & 0\\
    -1 & 1 & -1
  \end{pmatrix}
  $,求$\lambda I_n-A$的相似标准型.
\end{example}
\begin{solution}
  \begin{align*}
    \lambda I_3-A= &
    \begin{pmatrix}
      \lambda & -1 & 1\\
      -3 & \lambda+2 & 0\\
      1 & -1 & \lambda+1
    \end{pmatrix}\longrightarrow
    \begin{pmatrix}
      1 & -1 & \lambda+1\\
      -3 & \lambda+2 & 0\\
      \lambda & -1 & 1
    \end{pmatrix}\longrightarrow
    \begin{pmatrix}
      1 & -1 & \lambda+1\\
      0 & \lambda-1 & 3(\lambda+1)\\
      0 & \lambda-1 & -\lambda^2-\lambda+1
    \end{pmatrix}\\
    \longrightarrow &
                      \begin{pmatrix}
                        1 & 0 & 0\\
                        0 & \lambda-1 & 3(\lambda+1)\\
                        0 & \lambda-1 & -\lambda^2-\lambda+1
                      \end{pmatrix}\longrightarrow
                      \begin{pmatrix}
                        1 & 0 & 0\\
                        0 & \lambda -1 & 6\\
                        0 & \lambda-1 & -\lambda^2-4\lambda+4
                      \end{pmatrix}\\
   \longrightarrow &
                      \begin{pmatrix}
                        1 & 0 & 0\\
                        0 & 6 & \lambda-1\\
                        0 & -\lambda^2-4\lambda+4 & \lambda-1
                      \end{pmatrix}\longrightarrow
                     \begin{pmatrix}
                       1 & 0 & 0\\
                        0 & 6 & 6(\lambda-1)\\
                        0 & -\lambda^2-4\lambda+4 & 6(\lambda-1)
                     \end{pmatrix}\\
   \longrightarrow &
                      \begin{pmatrix}
                        1 & 0 & 0\\
                        0 & 6 & 0\\
                        0 & -\lambda^2-4\lambda+4 & (\lambda-1)^2(\lambda^2+4\lambda+2)
                      \end{pmatrix}\longrightarrow
                     \begin{pmatrix}
                       1 & 0 & 0\\
                       0 & 1 & 0\\
                       0 & 0 & (\lambda-1)^2(\lambda^2+4\lambda+2)
                     \end{pmatrix}
  \end{align*}
\end{solution}

\subsection{不变因子}
\begin{theory}
  设$A(\lambda)$的非零行列式因子为
  $D_1(\lambda),D_2(\lambda),\cdots,D_r(\lambda)$,则
  \begin{equation*}
   D_i(\lambda)\mid D_{i+1}(\lambda), \forall 1 \leq i \leq r-1. 
  \end{equation*}
\end{theory}
\begin{proof}
  任取$A(\lambda)$的$i+1$阶子式$M_{i+1}$,
  将$M_{i+1}$按某行(列)展开,
  这里假设按第一行展开:
  \begin{equation}\label{eq:model2}
    M_{i+1}=a_{11}(\lambda)M_{11}-a_{12}(\lambda)M_{12}+\cdots+(-1)^{i+2}a_{1,i+1}M_{1,i+1}
  \end{equation}
  \begin{align*}
    \text{由行列式因子定义} & \Longrightarrow D_i(\lambda)\mid (\lambda)M_{1j}, 1 \leq j \leq i+1\\
                            & \Longrightarrow D_i(\lambda)\mid M_{i+1}\\
                            & \xLongrightarrow {D_i(\lambda)\text{是}M_{i+1}\text{公因式}}
                              D_i(\lambda)\mid D_{i+1}(\lambda), \forall 1 \leq i \leq r-1 
  \end{align*}
\end{proof}
\begin{theorem}\label{thm:model6}
  相抵的$\lambda-$矩阵有相同的行列式因子,从而有相同的不变因子.
\end{theorem}
\begin{proof}
  设
  \begin{equation*}
    B(\lambda)=P(\lambda)A(\lambda)Q(\lambda)
  \end{equation*}
  其中,$P(\lambda),Q(\lambda)$是可逆$\lambda-$矩阵,
  设$A(\lambda),B(\lambda)$的行列式因子分别为
  \[D_k(\lambda),E_k(\lambda), 1 \leq k \leq n \]
  要证
  \[D_k(\lambda)=E_k(\lambda), \forall 1 \leq k \leq n.\]
  \begin{align}
    & B(\lambda)
    \begin{pmatrix}
      i_1 & i_2 & \cdots & i_k\\
      j_1 & j_2 & \cdots & j_k
    \end{pmatrix}\xlongequal {\text{Cauchy-Binet公式}}
    P(\lambda)A(\lambda)Q(\lambda)
    \begin{pmatrix}\label{eq:model3}
      i_1 & i_2 & \cdots & i_k\\
      j_1 & j_2 & \cdots & j_k
    \end{pmatrix}\\
    = & \sum_{\genfrac{}{}{0pt}{3}{1\leq r_1 <\cdots<r_k\leq n}{1\leq s_1 <\cdots<s_k\leq n}}P(\lambda)
      \begin{pmatrix}
        i_1 & i_2 & \cdots & i_k\\
        r_1 & r_2 & \cdots & r_k
      \end{pmatrix}A(\lambda)
      \begin{pmatrix}
        r_1 & r_2 & \cdots & r_k\\
      s_1 & s_2 & \cdots & s_k
      \end{pmatrix}Q(\lambda)
      \begin{pmatrix}\label{eq:model4}
        s_1 & s_2 & \cdots & s_k\\
      j_1 & j_2 & \cdots & j_k
      \end{pmatrix}
  \end{align}
  \begin{asparaenum}[$1^{\circ}.$]
  \item 若$D_k(\lambda)=0$:由\eqref{eq:model4}式可得:
    \[B(\lambda)
      \begin{pmatrix}
        i_1 & i_2 & \cdots & i_k\\
        j_1 & j_2 & \cdots & j_k
      \end{pmatrix}=0,\]
    从而$E_k(\lambda)=0$;
  \item 若$D_k(\lambda)\neq 0$:由行列式因子定义可知:
    \[D_k(\lambda)\mid A(\lambda)
      \begin{pmatrix}
        i_1 & i_2 & \cdots & i_k\\
        j_1 & j_2 & \cdots & j_k
      \end{pmatrix}\]
    从而由\eqref{eq:model4}式可知:
    \[D_k(\lambda)\mid B(\lambda)
      \begin{pmatrix}
        i_1 & i_2 & \cdots & i_k\\
        j_1 & j_2 & \cdots & j_k
      \end{pmatrix}\]
    从而$D_k(\lambda)\mid E_k(\lambda)$.
  \end{asparaenum}

  由于$A(\lambda)=P(\lambda)^{-1}B(\lambda)Q(\lambda)^{-1}$,
  因此同理可证,
  \begin{asparaenum}[$1^{\circ}.$]
  \item $E_k(\lambda)=0 \Longrightarrow D_k(\lambda)=0$;
  \item 若$E_k(\lambda) \neq 0$,则$D_k(\lambda) \neq 0$
    且$E_k(\lambda)\mid D_k(\lambda)$.
  \end{asparaenum}

  由以上可知,
  \[E_k(\lambda)=0 \Longleftrightarrow D_k(\lambda)=0\]
  或者
  \[D_k(\lambda)\mid E_k(\lambda), E_k(\lambda)\mid D_k(\lambda)\]

  结合以上可推出,
  \[\exists 0 \neq c \st D_r(\lambda)=c\cdot E_k(\lambda),\]
  又因为$D_k(\lambda),E_k(\lambda)$都是首一多项式,因此$c=1 \Longrightarrow D_k(\lambda)=E_k(\lambda)$.
\end{proof}
\begin{deduction}\label{ddn:model1}
  设$n$阶$\lambda-$矩阵$A(\lambda)$的法式为
  \[\Lambda=\diag\{d_1(\lambda),d_1(\lambda),\cdots,d_r(\lambda),0,\cdots,0\},\]
  其中$d_i$是非零首一多项式且$d_i(\lambda)\mid d_{i+1}(\lambda)(i=1,2,\cdots,r-1)$,
  则$A(\lambda)$的不变因子为$d_1(\lambda),\\d_2(\lambda),\cdots,d_r(\lambda)$.
  特别,法式和不变因子之间相互唯一确定.
\end{deduction}
\begin{proof}
  因为$A(\lambda)$相抵于$\Lambda$,由定理\ref{thm:model6}可知
  $A(\lambda)$的不变因子即为$\Lambda$的不变因子,
  而$\Lambda$的不变因子即$d_1(\lambda),d_2(\lambda),\cdots,d_r(\lambda)$.
\end{proof}
\begin{deduction}\label{ddn:model2}
  设$A(\lambda),B(\lambda)$为$n$阶$\lambda-$矩阵,
  则$A(\lambda)$与$B(\lambda)$相抵当且仅当它们有相同的法式.
\end{deduction}
\begin{proof}
  若$A(\lambda)$和$B(\lambda)$有相同的法式,设为$\Lambda$,
  则$A(\lambda)\sim \Lambda \sim B(\lambda)$,
  由相抵的传递性可知, $A(\lambda)\sim B(\lambda)$.

  若$A(\lambda)$和$B(\lambda)$相抵,
  设$A(\lambda)$的法式为$\Lambda_1$,
  设$B(\lambda)$的法式为$\Lambda_2$,则
  \[A(\lambda)\sim \Lambda_1,B(\lambda)\sim \Lambda_2,
    \text{且}A(\lambda)\sim B(\lambda)\]
  由相抵的传递性可知, $\Lambda_1\sim \Lambda_2$,
  由定理\ref{thm:model6}可知,
  $\Lambda_1$与$\Lambda_2$有相同的不变因子.
  由于$\Lambda_1$与$\Lambda_2$都是法式,
  所以它们主对角线上的非零首一多项式元素都相同,
  从而$\Lambda_1=\Lambda_2$.
\end{proof}

\begin{deduction}\label{ddn:model3}
  $\lambda-$矩阵在相抵关系下的全系不变量是它的行列式因子组或不变因子组.
\end{deduction}
\begin{proof}
  由定理\ref{thm:model6}可知,相抵的$\lambda-$矩阵有相同的行列式因子(组),
  从而有相同的不变因子(组).
  
  反过来,如果两个$\lambda-$矩阵有相同的行列式因子或不变因子,
  由推论\ref{ddn:model1}可知,它们必具有相同的法式,
  从而由推论\ref{ddn:model2}可知, $A(\lambda)\sim B(\lambda)$.
\end{proof}

\begin{deduction}\label{ddn:model4}
  $n$阶$\lambda-$矩阵$A(\lambda)$的法式与初等变换的选取无关.
\end{deduction}
\begin{proof}
  设$A(\lambda)$可通过某初等变换过程变换为法式$\Lambda_1$,
  设$A(\lambda)$可通过另一初等变换过程变换为法式$\Lambda_2$,
  只要证$\Lambda_1=\Lambda_2$.
  \[A(\lambda)\sim \Lambda_1,
    A(\lambda)\sim \Lambda_2
    \Longrightarrow \Lambda_1 \sim \Lambda_2\]
  由推论\ref{ddn:model2}可知,$\Lambda_1= \Lambda_2$.
\end{proof}
\begin{theorem}\label{thm:model7}
  数域$\mathbb{K}$上$n$阶矩阵$A$与$B$相似的充分必要条件是
  它们的特征矩阵$\lambda I-A$和$\lambda I-B$
  具有相同的行列式因子或不变因子.
\end{theorem}
\begin{proof}
  由定理\ref{thm:model-main1}可知,
  \[A\text{与}B\text{相似} \Longleftrightarrow
    \lambda I -A\text{与}\lambda I-B\text{相抵}.\]
  由推论\ref{ddn:model3}可知,
  $\lambda I -A$与$\lambda I-B$有相同的
  行列式因子或不变因子.
\end{proof}
\begin{notice}
  特征矩阵$\lambda I -A$的行列式因子及不变因子
  均简称为$A$的行列式因子及不变因子.
\end{notice}
\begin{deduction}
  设$\mathbb{F}\subseteq \mathbb{K}$是两个数域,
  $A,B$是$\mathbb{F}$上的两个矩阵,
  则$A$与$B$在$\mathbb{F}$上相似的充分必要条件是
  他们在$\mathbb{K}$上相似.
\end{deduction}
\begin{proof}
  $A$与$B$在$\mathbb{F}$上相似 $\xLongleftrightarrow {\text{def}} \exists$
  非异矩阵$P\in M_n(\mathbb{F}) \st B=P^{-1}AP$,
  $A$与$B$在$\mathbb{K}$上相似 $\xLongleftrightarrow {\text{def}} \exists$
  非异矩阵$Q\in M_n(\mathbb{K}) \st B=Q^{-1}AQ$.

  必要性:设$\exists$非异矩阵$P\in M_n(\mathbb{F}) \st B=P^{-1}AP$,
  可知$P^{-1}\in \mathbb{F}$.
  把$P,P^{-1}$看成是$\mathbb{K}$上的矩阵,
  从而$P$在$\mathbb{K}$上可逆,可以推出$A$与$B$在$\mathbb{K}$上也相似.

  充分性: 若$A$与$B$在$\mathbb{K}$上相似,
  由推论\ref{thm:model7}可知, 
  $\lambda I-A$与$\lambda I-B$作为$\mathbb{K}$上的
  $\lambda-$矩阵有相同的不变因子组(法式).
  实际上$\lambda I-A$与$\lambda I-B$都是$\mathbb{F}$上的$\lambda-$矩阵,
  由推论\ref{ddn:model4}可知,
  在求法式的过程中,只要取$\mathbb{F}[\lambda]$
  上的初等变换,就可以得到$\lambda I-A$与$\lambda I-B$
  的法式,且法式中的不变因子多项式$d_i(\lambda)$
  仍是$\mathbb{F}$上的多项式.因此,存在$\mathbb{F}$上的可逆$\lambda-$矩阵
  $P(\lambda),Q(\lambda),M(\lambda),N(\lambda) \st$
  \[P(\lambda)(\lambda I-A)Q(\lambda)=
    M(\lambda)(\lambda I-B)N(\lambda)=
    \diag\{d_1(\lambda),d_2(\lambda),\cdots,d_n(\lambda)\},\]
  从而
  \[(\lambda I-B)=M(\lambda)^{-1}P(\lambda)(\lambda I-A)Q(\lambda)N(\lambda)^{-1},\]
  即$\lambda I-A$与$\lambda I-B$在$\mathbb{F}$相抵,
  由定理\ref{thm:model-main1}可知,
  $A$与$B$在$\mathbb{F}$上相似.
\end{proof}
\begin{deduction}
  矩阵的不变因子组在基域扩张下不变.
\end{deduction}

\subsection{有理标准型}
\begin{theory}\label{thr:RM1}
  设$r$阶矩阵
  \begin{equation}\label{eq:RM0}
    F= \begin{pmatrix}
    0 & 1 & 0 & \cdots & 0\\
    0 & 0 & 1 & \cdots & 0\\
    \vdots&\vdots&\vdots&&\vdots\\
    0 & 0 & 0 & \cdots & 1\\
    -a_r & -a_{r-1} & -a_{r-2} & \cdots & -a_1
    \end{pmatrix},
  \end{equation}
则

(1) $F$的行列式因子为
\begin{equation}\label{eq:RM1}
  1,\cdots,1,f(\lambda),
\end{equation}
其中共有$r-1$个$1$, $f(\lambda)=\lambda^r+a_1\lambda^{r-1}+\cdots+a_r$, 
$F$的不变因子组也由\eqref{eq:RM1}式给出;

(2) $F$的极小多项式和特征多项式都等于$f(\lambda)$.
\end{theory}
\begin{proof}
  (1) \[\lambda I_r-F=
  \begin{pmatrix}
    \lambda &-1& 0 & \cdots & 0\\
    0 &\lambda & -1 & \cdots & 0\\
    \vdots&\vdots&\vdots&&\vdots\\
    0 & 0 &\cdots &\lambda & -1\\
    a_r & a_{r-1} & a_{r-2} & \cdots &\lambda+a_1
  \end{pmatrix}\]
\[D_r(\lambda)=|\lambda I_r-F|=\lambda^r+a_1\lambda^{r-1}+\cdots+a_r=f(\lambda)\]
对$\forall 1 \leq k < r$, $\lambda I-F$总有一个$k$阶子式$=(-1)^k$,从而
\[D_k(\lambda)=1, \forall 1 \leq k < r.\]

(2)设$F$极小多项式为$m(\lambda)$,只要证$m(\lambda)=f(\lambda)$.

由定理\ref{thm:CH}(Cayley-Hamilton定理)$\Longrightarrow$
$m(\lambda) \mid f(\lambda) \Longrightarrow \deg m(\lambda)\leq r$.

$1^{\circ}$.若$\deg m(\lambda)=r$,则$m(\lambda)=f(\lambda)$;

$2^{\circ}$.若$\deg m(\lambda)<r$,用反正法:设
\[m(\lambda)=c_{r-1}\lambda^{r-1}+c_{r-2}\lambda^{r-2}+
  \cdots+c_1\lambda+c_0, c_i\text{不全为}0,\]
设$e_i$为第$i$个元素为$1$,其余元素为$0$的单位行向量,则
\[e_1F=e_2,e_2F=e_3,\cdots,e_{r-1}F=e_r,\]
即
\begin{equation}\label{eq:RM2}
  e_1F=e_2,e_1F^2=e_3,\cdots,e_1F^{r-1}=e_r.
\end{equation}
由于$m(\Lambda)$是$F$的极小多项式,所以
\[0=m(F)=c_{r-1}F^{r-1}+c_{r-2}F^{r-2}+
  \cdots+c_1F+c_0I_r,\]
上式左右两边同时左乘$e_1$,可得
\[0=c_{r-1}e_1F^{r-1}+c_{r-2}e_1F^{r-2}+
  \cdots+c_1e_1F+c_0e_1,\]
上式各项分别应用\eqref{eq:RM2}式的结论,可得
\[0=c_{r-1}e_r+c_{r-2}e_{r-1}+
  \cdots+c_1e_2+c_0e_1=(c_0,c_1,\cdots,c_{r-1})
  \Longrightarrow c_0=c_1=\cdots=c_{r-1}=0\]
与假设矛盾.
\end{proof}
\begin{theory}\label{thr:RM2}
  设$\lambda-$矩阵$A(\lambda)$相抵于对角$\lambda-$矩阵
  \begin{equation}\label{eq:RM3}
    \diag\{d_1(\lambda),d_2(\lambda),\cdots,d_n(\lambda)\},
  \end{equation}
  $\lambda-$矩阵$B(\lambda)$相抵于对角$\lambda-$矩阵
  \begin{equation}\label{eq:RM4}
    \diag\{d_{i_1}(\lambda),d_{i_2}(\lambda),\cdots,d_{i_n}(\lambda)\},
  \end{equation}
  且$i_1,i_2,\cdots,i_n$是$1,2,\cdots,n$的全排列,
  则$A(\lambda) \sim B(\lambda)$.
\end{theory}
\begin{proof}
  利用行对换及列对换即可将\eqref{eq:RM3}式变换成\eqref{eq:RM4}式,
  因此\eqref{eq:RM3}式所表示的矩阵和\eqref{eq:RM4}式所表示的矩阵相抵,
  所以$A(\lambda) \sim B(\lambda)$.
\end{proof}
\begin{theorem}\label{thm:RM1}
  设$A$是数域$\mathbb{K}$上的$n$阶方阵, $A$的不变因子组为
  \[1,\cdots,1,d_1(\lambda),\cdots,d_k(\lambda),\]
  其中$\deg d_i(\lambda)=m_i$,则$A$相似于下列分块对角阵:
  \[F=
    \begin{pmatrix}
      F_1&&&\\
         &F_2&&\\
         &&\ddots&\\
      &&&F_k
    \end{pmatrix}\]
  其中$F_i$的阶等于$m_i$,且$F_i$是形如\eqref{eq:RM0}式的矩阵, 
  $F_i$的最后一行由$d_i(\lambda)$的系数(除最高次项)的负值组成.
\end{theorem}
\begin{proof}
  \begin{align}
    & \lambda I_n-A \sim \diag\{1,\cdots,1,d_1(\lambda),\cdots,d_k(\lambda)\}\label{eq:RM5}\\
      \Longrightarrow & |\lambda I_n-A|=d_1(\lambda)\cdots d_k(\lambda)\notag
  \end{align}
  则$m_1+m_2+\cdots+m_k=n$, 
  \[\lambda I-F_i=\lambda I-F(d_i(\lambda)) \sim
    \diag\{1,\cdots,1,d_i(\lambda)\},\]
  上式中$d_i(\lambda)$前有$m_i-1$个$1$,则
  \[\lambda I-F=
    \begin{pmatrix}
      \lambda I-F(d_1(\lambda))&&&\\
                               &\lambda I-F(d_2(\lambda))&&\\
                               &&\ddots&\\
      &&&\lambda I-F(d_k(\lambda))
    \end{pmatrix}\]
  即$\lambda I -A$的法式为
  \begin{equation}\label{eq:RM6}
    \diag\{1,\cdots,1,d_1(\lambda);1,\cdots,1,d_2(\lambda);\cdots;1,\cdots,1,d_k(\lambda)\},
  \end{equation}
  上式中,每个$d_i(\lambda)$前都配有$m_i-1$个$1$,即共有$m_1-1+m_2-1+\cdots+m_k-1=n-k$个$1$.
  因此, \eqref{eq:RM6}式是\eqref{eq:RM5}式的一个置换.由引理\ref{thr:RM2}可知,
  $\lambda I_n-A \sim \lambda I_n-F$,
  由定理\ref{thm:model-main1}可知,
  $A \sim F$.
\end{proof}
\begin{theorem}
  设数域$\mathbb{K}$上的$n$阶矩阵$A$的不变因子为
  \[1,\cdots,1,d_1(\lambda),\cdots,d_k(\lambda),\]
  其中$d_i(\lambda)\mid d_{i+1}(\lambda)(i=1,\cdots,k-1)$,
  则$A$的极小多项式$m(\lambda)=d_k(\lambda)$.
\end{theorem}
\begin{proof}
  由定理\ref{eq:RM1}可知,
  \[A \sim \diag\{F(d_1(\lambda)),F(d_2(\lambda)),\cdots,F(d_k(\lambda))\},\]

  $1^{\circ}.$ $m_A(\lambda)=m_F(\lambda)$;

  $2^{\circ}.$ 令$F_i=F(d_i(\lambda))$,则
  \[m_F(\lambda)= [m_{F_1}(\lambda),m_{F_2}(\lambda),\cdots,m_{F_k}(\lambda)].\]
  \begin{align*}
    \text{由引理\ref{thr:RM1}} \Longrightarrow &
      m_{F_i}(\lambda)=d_i(\lambda), 1 \leq i \leq k\\
    \Longrightarrow & m_A(\lambda)=[d_1(\lambda),d_2(\lambda),\cdots,d_k(\lambda)]
       \xlongequal {d_i(\lambda)\mid d_{i+1}(\lambda)} d_k(\lambda).
  \end{align*}
\end{proof}
\begin{deduction}
  设$A\in M_n(\mathbb{F})$,
  其极小多项式为$m_{\mathbb{F}}(\lambda)$. $\mathbb{F} \subseteq \mathbb{K}$,
  将$A$看成是$\mathbb{K}$上的矩阵,极小多项式为$m_{\mathbb{K}}(\lambda)$,
  则$m_{\mathbb{K}}(\lambda)=m_{\mathbb{F}}(\lambda)$.
\end{deduction}

\subsection{初等因子}
\begin{definition}
  设$f(\lambda)\in \mathbb{K}[\lambda], p(\lambda)$为不可约多项式,
  若存在$e\in \mathbb{Z}^+$,使得$p(\lambda)^e\mid f(\lambda)$
  但$p(\lambda)^{e+1}\nmid f(\lambda)$,
  则称$p(\lambda)^e$为$f(\lambda)$的准素因子.
\end{definition}
\begin{definition}
  设$d_1(\lambda),d_2(\lambda),\cdots,d_k(\lambda)$是数域$\mathbb{K}$上矩阵$A$的非常数不变因子,
  在$\mathbb{K}$上把$d_i(\lambda)$分解成不可约因式之积:
  \begin{equation}\label{eq:ED1}
    \begin{split}
      d_1(\lambda) & = p_1(\lambda)^{e_{11}}p_2(\lambda)^{e_{12}}\cdots p_t(\lambda)^{e_{1t}},\\
      d_2(\lambda) & = p_1(\lambda)^{e_{21}}p_2(\lambda)^{e_{22}}\cdots p_t(\lambda)^{e_{2t}},\\
      & \cdots\cdots\\
      d_k(\lambda) & = p_1(\lambda)^{e_{k1}}p_2(\lambda)^{e_{k2}}\cdots p_t(\lambda)^{e_{kt}},
    \end{split}
  \end{equation}
  其中$e_{ij}$是非负整数(注意$e_{ij}$可以为零!).
  由于$d_i(\lambda)\mid d_{i+1}$,因此
  \[e_{1j}\leq e_{2j}\leq \cdots\leq e_{kj}(j=1,2,\cdots,t).\]
  若\eqref{eq:ED1}式中的$e_{ij}>0$,则称$p_j(\lambda)^{e_{ij}}$
  为$A$的一个{\heiti 初等因子}, $A$的全体初等因子成为$A$的{\heiti 初等因子组}.

  由因式分解的唯一性可知$A$的初等因子被$A$的不变因子唯一确定.
  反过来,若给定一组初等因子$p_j(\lambda)^{e_{ij}}$,
  适当添加一些$1$(表示为$p_j(\lambda)^{e_{ij}}=0)$,
  则可将这组初等因子按降幂排列如下:
  \begin{equation}\label{eq:ED2}
    \begin{split}
      p_1(\lambda)^{e_{k1}},&  p_1(\lambda)^{e_{k-1,1}}, \cdots, p_1(\lambda)^{e_{11}},\\
      p_2(\lambda)^{e_{k2}},&  p_1(\lambda)^{e_{k-1,2}}, \cdots, p_1(\lambda)^{e_{12}},\\
      & \cdots\cdots\\
      p_t(\lambda)^{e_{kt}},&  p_1(\lambda)^{e_{k-1,t}}, \cdots, p_1(\lambda)^{e_{1t}},
    \end{split}
  \end{equation}
  令
  \begin{equation*}
    \begin{split}
      d_k(\lambda) & = p_1(\lambda)^{e_{k1}}p_2(\lambda)^{e_{k2}}\cdots p_t(\lambda)^{e_{kt}},\\
      d_{k-1}(\lambda) & = p_1(\lambda)^{e_{k-1,1}}p_2(\lambda)^{e_{k-1,2}}\cdots p_t(\lambda)^{e_{k-1,t}},\\
                   & \cdots\cdots\\
      d_1(\lambda) & = p_1(\lambda)^{e_{11}}p_2(\lambda)^{e_{12}}\cdots p_t(\lambda)^{e_{1t}},
    \end{split}
  \end{equation*}
  则$d_i(\lambda)\mid d_{i+1}(\lambda)(i=1,2,\cdots,k-1)$,
  且$d_1(\lambda),\cdots,d_k(\lambda)$的初等因子组就如
  \eqref{eq:ED2}式所示,因此,给定$A$的初等因子组,
  我们可唯一地确定$A$的不变因子组.这一事实表明,
  $A$的不变因子组与初等因子组在讨论矩阵相似关系中的作用
  是相同的.因此,有下述定理.
\end{definition}
\begin{theorem}
  数域$\mathbb{K}$上的两个矩阵$A$与$B$相似的充分必要条件是
  他们具有相同的初等因子组,即矩阵的初等因子组是矩阵相似关系的全系不变量.
\end{theorem}

\subsection{Jordan标准型}

\begin{theory}\label{thr:JD1}
  $r$阶矩阵
  \[J = \begin{pmatrix}
      \lambda_0 & 1 & & &\\
      & \lambda_0 & 1 & & \\
      & & \ddots & \ddots &\\
      & & & \ddots & 1\\
      & & & & \lambda_0
  \end{pmatrix}\]
的初等因子组为$(\lambda-\lambda_0)^r$.
\end{theory}
\begin{proof}
  \begin{equation}\label{eq:JD1}
    \lambda I-J=
    \begin{pmatrix}
      \begin{BMAT}{c.cccc}{cccc.c}
      \lambda-\lambda_0& -1 & & &\\
      & \lambda-\lambda_0 & -1 & &\\
      &&\ddots&\ddots&\\
      &&&\ddots& -1\\
      &&&&\lambda-\lambda_0
      \end{BMAT}
    \end{pmatrix},
  \end{equation}
  \begin{equation*}
    D_r(\lambda)=|\lambda I_r-J|=(\lambda-\lambda_0)^r,
  \end{equation*}
  如\eqref{eq:JD1}式矩阵中虚线右上角部分所示,
  \begin{align*}
    & \forall 1 \leq k < r, \exists k\text{阶子式}=(-1)^k\\
    \Longrightarrow & D_k(\lambda)=1, \forall 1 \leq k < r.
  \end{align*}
  \eqref{eq:JD1}式的行列式因子为
  \[1,\cdots,1,(\lambda-\lambda_0)^r,\]
  \eqref{eq:JD1}式的不变因子也为上式,
  因此矩阵$J$的初等因子组为$(\lambda-\lambda_0)^r$.
\end{proof}
\begin{theory}\label{thr:JD2}
  设特征矩阵$\lambda I-A$经过初等变换化为下列对角阵:
  \begin{equation}\label{eq:JD2}
    \begin{pmatrix}
    f_1(\lambda)&&&\\
    &f_2(\lambda)&&\\
    &&\ddots&\\
    &&&f_n(\lambda)
    \end{pmatrix}
  \end{equation}
  其中, $f(\lambda)(i=1,\cdots,n)$为非零首一多项式.
  将$f_i(\lambda)$作不可约分解,若$(\lambda-\lambda_0)^k$能整除$f_i(\lambda)$,
  但$(\lambda-\lambda_0)^{k+1}$不能整除$f_i(\lambda)$,
  就称$(\lambda-\lambda_0)^k$是$f_i(\lambda)$的一个准素因子,
  则矩阵$A$的初等因子组等于所有$f_i(\lambda)$的准素因子的集合.
\end{theory}
\begin{proof}
  {\heiti Step 1:}对$\forall 1 \leq i < j \leq n$,设
  \[(f_i(\lambda),f_j(\lambda))=g(\lambda), [f_i(\lambda),f_j(\lambda)]=h(\lambda),\]
  先证
  \begin{equation}\label{eq:JD3}
    \begin{pmatrix}
      f_1(\lambda)&&&&&&\\
                  &\cdots&&&&&\\
                  &&f_i(\lambda)&&&&\\
                  &&&\cdots&&&\\
                  &&&&f_j(\lambda)&&\\
                  &&&&&\cdots&\\
      &&&&&&f_n(\lambda)
    \end{pmatrix}\sim
    \begin{pmatrix}
      f_1(\lambda)&&&&&&\\
      &\cdots&&&&&\\
      &&g(\lambda)&&&&\\
      &&&\cdots&&&\\
      &&&&h(\lambda)&&\\
      &&&&&\cdots&\\
      &&&&&&f_n(\lambda)
    \end{pmatrix}
  \end{equation}
  并且两个对角阵主对焦元素的准素因子全体相同.

  不妨设$i=1, j=2$,设
  \begin{equation}\label{eq:JD4}
    f_1(\lambda)=g(\lambda)q(\lambda),
  \end{equation}
  \begin{equation}\label{eq:JD5}
    h(\lambda)=f_1(\lambda)f_2(\lambda)/g(\lambda)=f_2(\lambda)q(\lambda),
  \end{equation}
  \begin{equation}\label{eq:JD6}
    \exists u(\lambda),v(\lambda) \st
    g(\lambda)=f_1(\lambda)u(\lambda)+f_2(\lambda)v(\lambda).
  \end{equation}
  由于对角阵第$1,2$两行(列)的初等行(列)变换和其他行(列)无关,因此,
  下面只对前$2$行(列)进行操作:
  \begin{align*}
    & \begin{pmatrix}
    f_1(\lambda) & 0\\
    0 & f_2(\lambda)
    \end{pmatrix}\xlongrightarrow{\eqref{eq:JD6}\text{式}}
    \begin{pmatrix}
      f_1(\lambda) & g(\lambda)\\
      0 & f_2(\lambda)
    \end{pmatrix}\xlongrightarrow{\eqref{eq:JD4}\text{式}}
    \begin{pmatrix}
      0 & g(\lambda)\\
      -f_2(\lambda)q(\lambda) & f_2(\lambda)
    \end{pmatrix}\\
    \xlongrightarrow{g(\lambda)\text{是}f_2(\lambda)\text{的因式}}
    & \begin{pmatrix}
      0 & g(\lambda)\\
      -h(\lambda) & 0
    \end{pmatrix}\longrightarrow
      \begin{pmatrix}
        g(\lambda) & 0\\
        0 & h(\lambda)
      \end{pmatrix}.
  \end{align*}
  在数域$\mathbb{K}$上做公共因式分解:
  \begin{equation}\label{eq:JD7}
  \begin{split}
    f_1(\lambda) &= p_1(\lambda)^{r_1}p_2(\lambda)^{r_2}\cdots p_t(\lambda)^{r_t},\\
    f_2(\lambda) &= p_1(\lambda)^{s_1}p_2(\lambda)^{s_2}\cdots p_t(\lambda)^{s_t},
  \end{split}
  \end{equation}
  其中$p_i(\lambda)$是首一互异的不可约多项式且$r_i,s_i\geq 0$,
  令$e_i=\min\{r_i,s_i\}, k_i=\max\{r_i,s_i\}$,则
  \begin{equation}\label{eq:JD8}
    \begin{split}
    g(\lambda) &= p_1(\lambda)^{e_1}p_2(\lambda)^{e_2}\cdots p_t(\lambda)^{e_t},\\
    h(\lambda) &= p_1(\lambda)^{k_1}p_2(\lambda)^{k_2}\cdots p_t(\lambda)^{k_t},
    \end{split} 
  \end{equation}
  \eqref{eq:JD8}式和\eqref{eq:JD7}式中,
  $e_i,k_i$和对应的$r_i,s_i$相同或只是互换个位置,
  由构造过程可知:
  \[\diag\{f_1(\lambda),f_2(\lambda)\}\text{与}
    \diag\{g(\lambda),h(\lambda)\}\text{有相同的准素因子组}.\]

  {\heiti Step 2:}反复利用Step 1,
  将$\diag\{f_1(\lambda),\cdots,f_n(\lambda)$变为法式.
  依次将第$(1,1)$元素与$(2,2)$元素,第$(3,3)$元素,\ldots,第$(n,n)$元素
  进行Step 1的操作,从而使第$(1,1)$元素的多项式幂次均是最小的.
  一般地,设第$(1,1)$元素,\ldots,第$(i-1,i-1)$元素已完成如上操作,
  第$(i,i)$元素,第$(i+1,i+1)$元素,\ldots,第$(n,n)$元素重复进行Step 1的操作,
  可使原矩阵相抵于$\diag\{d_1(\lambda),d_2(\lambda),
  \cdots,d_n(\lambda)\}$且满足
  $d_1(\lambda)\mid d_2(\lambda)\mid \cdots \mid d_n(\lambda)$,
  即原矩阵和其法式相抵.
\end{proof}
\begin{example}\label{exl:JD1}
  $\lambda I-A$相抵于$\diag\{1,(\lambda-1)^2(\lambda+2),\lambda+2,1,\lambda-1\}$,
  求它的所有初等因子.
\end{example}
\begin{solution}
  由引理\ref{thr:JD2}可知,
  初等因子为$\lambda-1,(\lambda-1)^2,\lambda+2,\lambda+2$.
\end{solution}
\begin{theory}\label{thr:JD3}
  设$A=\diag\{J_{r_1}(\lambda_1),J_{r_2}(\lambda_2),\cdots,J_{r_k}(\lambda_k)\}$,
  则$A$的初等因子组为$(\lambda-\lambda_1)^{r_1},(\lambda-\lambda_2)^{r_2},\cdots,J_{r_k}(\lambda)^{r_k}$.
\end{theory}
\begin{proof}
  \begin{equation}\label{eq:JD9}
    \lambda I-A=
    \begin{pmatrix}
      \lambda I-J_{r_1}(\lambda_1)&&&\\
                                  &\lambda I-J_{r_2}(\lambda_2)&&\\
                                  &&\ddots&\\
      &&&\lambda I-J_{r_k}(\lambda_k)
    \end{pmatrix}
  \end{equation}
  由引理\ref{thr:JD1}可推出
   $\lambda I -J_{r_i}(\lambda_i) \sim
   \diag\{1,\cdots,1,(\lambda-\lambda_i)^{r_i}\}$,因此
   \[\lambda I-A \sim
     \diag\{1,\cdots,1,(\lambda-\lambda_1)^{r_1};
     1,\cdots,1,(\lambda-\lambda_2)^{r_2};
     \cdots;1,\cdots,1,(\lambda-\lambda_k)^{r_k}\},\]
   由引理\ref{thr:JD2}可推出,
   $A$的初等因子组为
   $(\lambda-\lambda_1)^{r_1},(\lambda-\lambda_2)^{r_2},\cdots,(\lambda-\lambda_k)^{r_k}$.
 \end{proof}
 \begin{theorem}\label{thm:JD1}
   设$A\in M_n(\mathbb{C})$的初等因子组为
   $(\lambda-\lambda_1)^{r_1},(\lambda-\lambda_2)^{r_2},\cdots,(\lambda-\lambda_k)^{r_k}$,
   则$A$相似于
   $J=\diag\{J_{r_1}(\lambda_1),J_{r_2}(\lambda_2),\cdots,J_{r_k}(\lambda_k)\}$.
 \end{theorem}
 \begin{proof}
   由引理\ref{thr:JD3}可知$J$的初等因子组也为
   $(\lambda-\lambda_1)^{r_1},(\lambda-\lambda_2)^{r_2},\cdots,(\lambda-\lambda_k)^{r_k}$,
   从而
   \[A \simeq J.\]
 \end{proof}
 \begin{definition}
   定理\ref{thm:JD1}中的
   $J=\diag\{J_{r_1}(\lambda_1),J_{r_2}(\lambda_2),\cdots,J_{r_k}(\lambda_k)$
   称为$A$的Jordan标准型, $J_{r_i}(\lambda_i)$称为Jordan块.
 \end{definition}
 \begin{notice}
   Jordan标准型中, Jordan块次序可以互换.
   每个Jordan块由一个初等因子唯一确定.
   \[J_{r}(\lambda_0)\longleftrightarrow (\lambda-\lambda_0)^r,\]
   从而在不考虑Jordan块的次序时, Jordan标准型
   由$A$唯一确定.
 \end{notice}
 \begin{theorem}\label{thm:JD2}
   设$\varphi\in \mathcal{L}(V^n_{\mathbb{C}})$,
   则存在$V$的一组基
   $\{e_1,e_2,\cdots,e_n\}$,
   使得$\varphi$在这组基下的表示矩阵为Jordan标准型
   $J=\diag\{J_{r_1}(\lambda_1),\cdots,J_{r_k}(\lambda_k)\}$.
 \end{theorem}
 \begin{deduction}
   设$A\in M_n(\mathbb{C})$,则下列结论等价:

   (1) $A$可对角化;

   (2) $A$的极小多项式无重根;

   (3) $A$的初等因子都是一次多项式.
 \end{deduction}\label{dd:JD1}
 \begin{proof}
   (1)$\Longrightarrow$(2):由例题\ref{exl:CH1}可得.

   (2)$\Longrightarrow$(3):设$A$的不变因子组为
   $1,\cdots,1,d_1(\lambda),\cdots,d_k(\lambda)$,
   由有理标准型理论可知, $A$的极小多项式为
   $m(\lambda)=d_k(\lambda)$,
   从而$d_k(\lambda)$无重根,
   因此$d_1(\lambda),\cdots,d_k(\lambda)$均无重根,
   由此可推出, $d_i(\lambda)$的准素因子都是一次的,
   即$A$的初等因子都是一次多项式.

   (3)$\Longrightarrow$(1):设$A$的初等因子组为
   $\lambda-\lambda_1,\lambda-\lambda_2,\cdots,\lambda-\lambda_n$,
   由定理\ref{thm:JD2}可知,
   \[A \simeq J=
     \begin{pmatrix}
       \lambda_1&&&\\
                &\lambda_2&&\\
                &&\ddots&\\
       &&&\lambda_n
     \end{pmatrix},\]
   所以,$A$可对角化,即$A$的Jordan块都是一阶的.
 \end{proof}
 
 \begin{question}\label{qs:JD1}
   设$\varphi\in \mathcal{L}(V^n_{\mathbb{C}})$,
   $V_o$ 是$\varphi$的不变子空间.
   若$\varphi$可对角化,则
   $\varphi|_{V_0}$也可对角化.
 \end{question}
 \begin{proof}
   设$\varphi$的极小多项式为$m(\lambda)$,
   $\varphi|_{V_0}$的极小多项式为$g(\lambda)$,则
   \[m(\varphi|_{V_0})=m(\varphi)|_{V_0}=0.\]
   \begin{align*}
     \text{由极小多项式性质} \Longrightarrow & g(\lambda)|m(\lambda)\\
     \varphi\text{可对角化、推论\ref{dd:JD1}} \Longrightarrow & m(\lambda)\text{无重根}
                                                 \Longrightarrow g(\lambda)\text{也无重根}\\
     \xLongrightarrow{\text{推论}\ref{dd:JD1}} & \varphi|_{V_0}\text{也可对角化}.
   \end{align*}
 \end{proof}
 
 \begin{deduction}\label{dd:JD2}
   若
   \[M=
     \begin{pmatrix}
       A^{m\times m} & C^{m\times n}\\
       O & B^{n\times n}
     \end{pmatrix}\]
   可对角化,则$A, B$均可对角化.
 \end{deduction}
 \begin{proof}
   设$M$的极小多项式为$m(\lambda)$,则
   \[0=m(M)=
     \begin{pmatrix}
       m(A)& *\\
       O & m(B)
     \end{pmatrix},\]
   从而
   \[m(A)=m(B)=0.\]
   由极小多项式性质可以推出
   \[m_A(\lambda)\mid m(\lambda), m_B(\lambda)\mid m(\lambda).\]
   \[\text{由推论\ref{dd:JD1}} \Longrightarrow
     m(\lambda)\text{无重根} \Longrightarrow
     m_A(\lambda),m_B(\lambda)\text{无重根} \Longrightarrow
     A, B\text{可对角化}.\]
 \end{proof}

 \begin{question}\label{qs:JD2}
   $\varphi\in \mathcal{L}(V), V=V_1\oplus V_2\oplus\cdots\oplus V_k$,
   $V_i$为$\varphi$的不变子空间.则
   \[\varphi\text{可对角化} \Longleftrightarrow
     \varphi|_{V_i}\text{均可对角化}.\]
\end{question}
\begin{proof}
  {\heiti 必要性}由命题\ref{qs:JD1}可得结论.

  {\heiti 充分性}$\exists V_i$的一组基
  $\{e_{i1},e_{i2},\cdots,e_{in_i}\} \st \varphi|_{V_i}$
  的表示矩阵为对角阵.
  \[V=V_1\oplus V_2\oplus\cdots\oplus V_k \Longrightarrow V_i\text{的基可拼成}V\text{的一组基},\]
  $\because \varphi$在每个$\varphi|_{V_i}$的表示矩阵在对应的基下都是对角阵,\\
  $\therefore \varphi$在这组基下的表示矩阵也为对角阵.
\end{proof}
\begin{deduction}\label{dd:JD3}
  \[A=
    \begin{pmatrix}
      A_1 &&&\\
          & A_2 &&\\
          &&\cdots&\\
      &&&A_k
    \end{pmatrix},\]
  则$A$可对角化 $\Longleftrightarrow$
  $A_1,A_2,\cdots,A_k$均可对角化.
\end{deduction}
\begin{proof}
  \[m_A(\lambda)=[m_{A_1}(\lambda),m_{A_2}(\lambda),\cdots,m_{A_k}(\lambda)].\]
\end{proof}

\begin{deduction}\label{dd:JD4}
  设$A\in M_n(\mathbb{K})$的特征值全部在$\mathbb{K}$中,
  则$A$在$\mathbb{K}$上相似于其Jordan标准型.
\end{deduction}
\begin{proof}
  有定理\ref{thm:JD1}可推出,在复数域上
  \[A \simeq J=\diag\{J_{r_1}(\lambda_1),\cdots,J_{r_K}(\lambda_k)\},\]
  由相似关系的基域扩张不变形可知,在数域$\mathbb{K}$上
  \[A \simeq J.\]
\end{proof}

\begin{example}\label{exl:JD2}
  $A$的初等因子组为$(\lambda-1),(\lambda-1)^3,(\lambda+2)^2,(\lambda+2)^3$,
  求$A$的Jordan标准型.
\end{example}
\begin{solution}
  \[J=\diag\left\{1,
    \begin{pmatrix}
      1&1&\\
       &1&1\\
      &&1
    \end{pmatrix},
    \begin{pmatrix}
      -2 & 1\\
      &-2
    \end{pmatrix},
    \begin{pmatrix}
      -2&1&\\
        &-2&1\\
      &&-2
    \end{pmatrix}\right\}\]
\end{solution}

\begin{example}\label{exl:JD3}
  设$\varphi\in \mathcal{L}(V^4_{\mathbb{C}})$
  在基$\{e_1,e_2,e_3,e_4\}$下的表示矩阵为
  \[A=
    \begin{pmatrix}
      3&1&0&0\\
      -4&-1&0&0\\
      6&1&2&1\\
      -14&-5&-1&0
    \end{pmatrix},\]
  $\varphi$在新基$\{f_1,f_2,f_3,f_4\}$下的表示矩阵为
  Jordan标准型$J$,求过渡矩阵$P$.
\end{example}
\begin{solution}
  \[P^{-1}AP=J,\]
  \[\lambda I-A \longrightarrow \cdots
    \longrightarrow \diag\{1,1,(\lambda-1)^2,(\lambda-1)^2\},\]
  所以$A$的初等因子组为$(\lambda-1)^2,(\lambda-1)^2$,
  由此可算出
  \[J=
    \begin{pmatrix}
      1&1&&\\
      0&1&&\\
       &&1&1\\
      &&0&1
    \end{pmatrix}
  \]
  设$P=(\alpha_1,\alpha_2,\alpha_3,\alpha_4)$,
  \begin{align*}
    & P^{-1}AP=J \Longrightarrow AP=PJ \Longrightarrow
    (A\alpha_1,A\alpha_2,A\alpha_3,A\alpha_4)=
    (\alpha_1,\alpha_2,\alpha_3,\alpha_4)
    \begin{pmatrix}
            1&1&&\\
      0&1&&\\
       &&1&1\\
      &&0&1
    \end{pmatrix}\\
    \Longrightarrow &
                      \left\{
                      \begin{aligned}
    A\alpha_1& =\alpha_1\\
    A\alpha_2& =\alpha_1+\alpha_2 \Longrightarrow
                        (A-I_4)\alpha_2=\alpha_1\\
    A\alpha_3& =\alpha_3\\
    A\alpha_4& =\alpha_3+\alpha_4 \Longrightarrow
                        (A-I_4)\alpha_4=\alpha_3
                      \end{aligned}
                      \right.   \Longrightarrow
   \alpha_1=
                      \begin{pmatrix}
                        -1\\
                        2\\
                        4\\
                        0
                      \end{pmatrix},
   \alpha_3=
                      \begin{pmatrix}
                        -1\\
                        2\\
                        0\\
                        4
                      \end{pmatrix},\\
    \Longrightarrow & \alpha_2=
      \begin{pmatrix}
        \frac{5}{4}\\
        -\frac{7}{2}\\
        0\\
        0
      \end{pmatrix}, 
      \alpha_4=
      \begin{pmatrix}
        \frac{1}{4}\\
        -\frac{3}{2}\\
        0\\
        0
      \end{pmatrix}\Longrightarrow
      P=(\alpha_1,\alpha_2,\alpha_3,\alpha_4)=
    \begin{pmatrix}
      -1&\frac{5}{4}&-1&\frac{1}{4}\\
      2&-\frac{7}{2}&2&-\frac{3}{2}\\
      4&0&0&0\\
      0&0&4&0
    \end{pmatrix}
  \end{align*}
\end{solution}

\subsection{Jordan标准型的进一步讨论和应用}

设$V$是$n$维复线性空间, $\varphi$是$V$上的线性变换.
设$\varphi$的初等因子组为
\begin{equation}\label{eq:JDapp1}
  (\lambda-\lambda_1)^{r_1},(\lambda-\lambda_2)^{r_2},\cdots,
  (\lambda-\lambda_k)^{r_k},
\end{equation}
由定理\ref{thm:JD2}可知,存在$V$的一组基
$\{e_{11},e_{12},\cdots,e{1r_1};e_{21},e_{22},\cdots,e_{2r_2};\cdots;e_{k1},e_{k2},\cdots,e_{kr_k}\}$,
使得$\varphi$在这组基下的表示矩阵为
\[J=
  \begin{pmatrix}
    J_1&&&\\
       &J_2&&\\
       &&\cdots&\\
    &&&J_k
  \end{pmatrix}.\]
上式中每个$J_i$是相应于初等因子$(\lambda-\lambda_i)^{r_i}$的Jordan块,其阶正好为$r_i$.
令$V_i$是由基元$e_{i1}, e_{i2},\\ \cdots, e_{ir_i}$生成的子空间,则
\begin{equation}\label{eq:JDapp2}
  \begin{split}
   & \varphi(e_{i1})  = \lambda_ie_{i1},\\
   & \varphi(e_{i2})  = e_{i1}+\lambda_ie_{i2},\\
                   &\hspace{2em} \cdots\cdots\\
   & \varphi(e_{ir_i})  = e_{ir_i-1}\lambda_ie_{ir_i}.\\
  \end{split}
\end{equation}
这表明$\varphi(V_i)\subseteq V_i$,
即$V_i(i=1,2,\cdots,k)$是$\varphi$的不变子空间,且有
\[V=V_1\oplus V_2 \oplus\cdots\oplus V_k.\]

\begin{theorem}
  线性变换$\varphi$的特征值$\lambda_1$的代数重数等于
  属于$\lambda_1$的Jordan块的阶数之和;
  $\lambda_1$的几何重数等于属于$\lambda_1$的Jordan块的个数.
\end{theorem}

令$\psi=\varphi-\lambda_i I_V$,
则\eqref{eq:JDapp2}式可化为下式:
\begin{equation}\label{eq:JDapp3}
  \begin{split}
    & \psi(e_{i1})=0,\\
    & \psi(e_{i2})=e_{i1},\\
    & \hspace{2em} \cdots\cdots\\
    & \psi(e_{ir_i})=e_{ir_i-1}.
  \end{split}
\end{equation}
即有如下循环链:
\[e_{ir_i} \xlongrightarrow {\psi} e_{ir_i-1} \xlongrightarrow {\psi}
  e_{ir_i-2} \xlongrightarrow {\psi} \cdots \xlongrightarrow {\psi}
  e_{i2} \xlongrightarrow {\psi} e_{i1} \xlongrightarrow {\psi} 0.\]
从而
\[\{e_{ir_i},\psi(e_{ir_i}),\cdots,\psi^{r_i-1}(e_{ir_i})\}\]
构成了$V_i$的一组基.从以下定义可知, $V_i$是关于
$\psi=\varphi-\lambda_i I_V$的循环子空间.

\begin{definition}
  设$\varphi \in \mathcal{L}(V)$, $V_0$是$r$维$\varphi-$不变子空间.
  若存在$0 \neq \alpha \in V_0$使得
  $\{\alpha,\varphi(\alpha),\cdots,\\
  \varphi^{r-1}(\alpha)\}$
  是$V_0$的一组基,则称$V_0$为关于关于$\varphi$的循环子空间,
  $\alpha$称为循环向量.
\end{definition}

\begin{theory}\label{thr:JDapp1}
  设$R(\lambda_1)=V_1\oplus V_2 \oplus \cdots \oplus V_s$,则
  \[R(\lambda_1)=\KER(\varphi-\lambda_1 I_V)^n=
  \{v\in V \mid (\varphi-\lambda_1 I_v)^n(v)=0\}.\]
\end{theory}

\begin{proof}
  先证$R(\lambda_1) \subseteq \KER(\varphi-\lambda_i I_V)^n$:

  任取$v\in R(\lambda_1), v=v_1+v_2+\cdots+v_s, v_i\in V_i$,
  则$V_i$的基有如下循环链:
  \[e_{ir_i} \xlongrightarrow {\varphi-\lambda_1 I_V} e_{ir_i-1} \xlongrightarrow {\varphi-\lambda_1 I_V}
  e_{ir_i-2} \xlongrightarrow {\varphi-\lambda_1 I_V} \cdots \xlongrightarrow {\varphi-\lambda_1 I_V}
  e_{i2} \xlongrightarrow {\varphi-\lambda_1 I_V} e_{i1} \xlongrightarrow {\varphi-\lambda_1 I_V} 0.\]
可得以下结论:
\begin{align*}
  & (\varphi-\lambda_1 I_V)^{r_i}(e_{ij})=0, \forall j \Longrightarrow
  (\varphi-\lambda_1 I_V)^{r_i}(v_i)=0\\
  \Longrightarrow & (\varphi-\lambda_1 I_V)^n(v)=
  (\varphi-\lambda_1 I_V)^n(v_1)+\cdots+(\varphi-\lambda_1 I_V)^n(v_s)=0
\end{align*}
上式中的$n$可缩小到$\max\{r_1,r_2,\cdots,r_s\}$.

再证$\KER(\varphi-\lambda_1 I)^n\subseteq R(\lambda_1)$:

设$v\in V$在上述基下的坐标向量$x=(x_{11},x_{12},\cdots,x_{1r_1},\cdots)'$,
求$(J-\lambda_1 I_n)^nx=0$的解.
\[ (J_{r_i}(\lambda_i)-\lambda_1 I_{r_i})^n=
  \begin{pmatrix}
    \lambda_i-\lambda_1&1&&&\\
                       &\lambda_i-\lambda_1&1&&\\
                       &&\ddots&\ddots&\\
                       &&&\ddots&1\\
    &&&&\lambda_i-\lambda_1
  \end{pmatrix}^n=\left\{
    \begin{aligned}
      0& , 1 \leq i \leq s\\
      \text{非异}& , s < i \leq k
    \end{aligned}\right.
\]
由上式可推出
\[
\left\{
  \begin{aligned}
    & x_{i1}-x_{i2}=\cdots=x_{ir_i}=0 , s < i \leq k\\
    & (x_{11},x_{12}\cdots,x_{1r_1};\cdots;x_{s1},x_{s2},\cdots,x_{sr_s})
    \text{为任意解}
  \end{aligned}\right.
\]
由上式可得
\[\KER(\varphi-\lambda_1 I)^n \subseteq
  V_1\oplus V_2 \oplus \cdots \oplus V_s=R(\lambda_1).\]
\end{proof}
%%% Local Variables:
%%% mode: LaTeX
%%% TeX-master: "../main"
%%% End:
