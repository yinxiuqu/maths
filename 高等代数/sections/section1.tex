\section{行列式}

\subsection{\texorpdfstring{$n$}{\textit{n}}阶行列式}
\begin{definition}\label{dfn:n-order-determinant1}
  由两条竖线围成的$n$行$n$列元素组成的式子(数值)
  称为$n$阶行列式:
  \[
    |A|=
    \begin{vmatrix}
      a_{11}&a_{12}&\cdots&a_{1n}\\
      a_{21}&a){22}&\cdots&a_{2n}\\
      \vdots&\vdots&&\vdots\\
      a_{n1}&a_{n2}&\cdots&a_{nn}
    \end{vmatrix}.
  \]
  有时也记为$\det(A)$, determinant.
  \begin{align*}
    a_{i1}, a_{i2},\cdots, a_{in} \quad & \text{第$i$行}\\
    a_{1j}, a_{2j},\cdots, a_{nj} \quad & \text{第$j$列}\\
    a_{11}, a_{22},\cdots, a_{nn} \quad & \text{主对角线}
  \end{align*}
  $a_{ij}$称为第$(i,j)$元素,删去$a_{ij}$所在的第$i$行,第$j$列,
  剩余元素按原来的顺序构成一个$n-1$阶行列式,称为$a_{ij}$的余子式,
  记为$M_{ij}$.
\end{definition}

\begin{definition}[递归定义]\label{dfn:n-order-determinant2}
  当$n=1$是, $1$阶行列式$|a_{11}| \triangleq a_{11}$.
  设所有$n-1$阶行列式值已定义好,特别地, $M_{ij}$已定义好,
  下面按第一列展开,定义$n$阶行列式:
  \[
    |A| \triangleq a_{11}M_{11}-a_{21}M_{21}+\cdots +(-1)^{n+1}a_{n1}M_{n1}.
  \]
\end{definition}

\begin{definition}\label{dfn:n-order-determinant3}
  $a_{ij}$的代数余子式为
  \[
    A_{ij} = (-1)^{i+j}M_{ij},
  \]
  定义$A$的$n$阶行列式:
  \[
    |A| = a_{11}A_{11}+a_{21}A_{21}+\cdots +a_{n1}A_{n1}.
  \]
\end{definition}

\begin{definition}\label{dfn:n-order-determinant4}
  若$a_{ij}=0, \forall i>j$,即主对角线下方所有元素都为$0$,
  则称$|A|$为上三角行列式;若$a_{ij}=0, \forall i<j$,
  即主对角线上方所有元素都为$0$,则称$|A|$为下三角行列式.
\end{definition}

\subparagraph{\color{ecolor}性质1}
上三角(下三角)行列式值等于主对角线元素之积.

\begin{proof}
  当$n=1$时, $|a_{11}|=a_{11}$,结论成立.
  设结论对$n-1$阶行列式成立,下面证$n$阶的情形.

  $1^{\circ}.$上三角时:
  \[
    |A|=
    \begin{vmatrix}
      a_{11}&a_{12}&\cdots&a_{1n}\\
      0&a_{22}&\cdots&a_{2n}\\
      \vdots&\vdots&&\vdots\\
      0&0&\cdots&a_{nn}
    \end{vmatrix}.
  \]
  根据定义\ref{dfn:n-order-determinant2}, $|A|=a_{11}M_{11}$.
  其中, $M_{11}$为$n-1$阶上三角行列式.由归纳假设,
  \[
    M_{11}=a_{11}a_{22}\cdots a_{nn},
  \]
  因此,
  \[
    |A|=a_{11}a_{22}\cdots a_{nn}.
  \]

  $2^{\circ}.$下三角时:
  \[
    |A|=
    \begin{vmatrix}
      a_{11}&0&0&\cdots&0\\
      a_{21}&a_{22}&0&\cdots&0\\
      a_{31}&a_{32}&a_{33}&\cdots&0\\
      \vdots&\vdots&\vdots&\ddots&\vdots\\
      a_{n1}&a_{n2}&a_{n3}&\cdots&a_{nn}
    \end{vmatrix}.
  \]
  由定义\ref{dfn:n-order-determinant2}
  \[
    |A|=a_{11}M_{11}-a_{21}M_{21}+\cdots +(-1)^{n+1}a_{n1}M_{n1}.
  \]
  考虑余子式$M_{k1}(1\leq k \leq n)$,
  设$M_{k1}=(b_{ij})_{(n-1)\times (n-1)}$,则
  \begin{equation}\label{eq:n-order-determinant1}
    b_{ij}=
    \begin{cases}
      a_{i,j+1}, & 1\leq i \leq k-1,\\
      a_{i+1,j+1}, & k\leq i \leq n-1.
    \end{cases}
  \end{equation}
  $|A|$为下三角行列式,即
  \begin{equation}
    \label{eq:n-order-determinant2}
    \forall i<j, a_{ij}=0.
  \end{equation}
  由\eqref{eq:n-order-determinant1}式和
  \eqref{eq:n-order-determinant2}式可知,
  $M_{k1}$都是下三角行列式.可以断言,
  当$k\geq 2$时, $M_{k1}$必有一个主对角元为$0$,
  因为
  \[
    b_{k-1,k-1}=a_{k-1,k}=0.
  \]
  因此,根据归纳假设,
  \[
    M_{k1}=0(k\geq 2), M_{11}=a_{22}\cdots a_{nn},
  \]
  从而
  \[
    |A|=a_{11}a_{22}\cdots a_{nn}.    
  \]
\end{proof}

\subparagraph{\color{ecolor}性质2}
行列式$|A|$的某行(列)乘以常数$c$,
得到的行列式$|B|=c\cdot|A|$.

\begin{proof}
  当$n=1$时,
  \[
    |B|=|c\cdot a_{11}|=ca_{11}=c\cdot |A|.
  \]
  假设结论对$n-1$阶行列式成立,下面证$n$阶的情形.

  $1^{\circ}.$行的情形:
  \[
    |B|=
    \begin{vmatrix}
      a_{11}&a_{12}&\cdots&a_{1n}\\
      \hdotsfor{4}\\
      ca_{i1}&ca_{i2}&\cdots&ca_{in}\\
      \hdotsfor{4}\\
      a_{n1}&a_{n2}&\cdots&a_{nn}
    \end{vmatrix}.
  \]
  设$|A|$的余子式为$M_{ij}$, $|B|$的余子式为$N_{ij}$.
  由定义\ref{dfn:n-order-determinant2}可得
  \begin{equation}\label{eq:n-order-determinant3}
    |B|=a_{11}N_{11}-a_{21}N_{21}+\cdots+
    (-1)^{i+1}ca_{i1}N_{i+1}+\cdots+(-1)^{n+1}a_{n1}N_{n1}.
  \end{equation}
  当$k\neq i$时, $N_{k1}$是$M_{k1}$的某一行元素乘以$c$得到,
  因此由归纳假设可知
  \begin{equation}\label{eq:n-order-determinant4}
    N_{k1}=c\cdot M_{k1},
  \end{equation}
  当$k=i$时,
  \begin{equation}\label{eq:n-order-determinant5}
    N_{k1}= M_{k1}.
  \end{equation}
  将\eqref{eq:n-order-determinant4}式和\eqref{eq:n-order-determinant5}式
  代入\eqref{eq:n-order-determinant3}式中去:
  \[
    |B|=c(a_{11}M_{11}-a_{21}M_{21}+\cdots+(-1)^{n+1}a_{n1}M_{n1})
    =c|A|.
  \]

  $2^{\circ}.$列的情形:

  (1)乘以$c$的是第$1$列时:
  \[
    |B|=
    \begin{vmatrix}
      ca_{11}&a_{12}&\cdots&a_{1n}\\
      ca_{21}&a_{22}&\cdots&a_{2n}\\
      \hdotsfor{4}\\
      ca_{n1}&a_{n2}&\cdots&a_{nn}
    \end{vmatrix}.
  \]
  此时,$\forall i, N_{i1}=M_{i1}$,所以
  \begin{align*}
    |B|= & ca_{11}N_{11}-ca_{21}N_{21}+\cdots+(-1)^{n+1}ca_{n1}N_{n1}\\
    = & c(a_{11}M_{11}-a_{21}M_{21}+\cdots+(-1)^{n+1}a_{n1}M_{n1})
    = c|A|.
  \end{align*}

  (2)乘以$c$的是第$j$列, $j\geq 2$时:
  \[
    |B|=
    \begin{vmatrix}
      a_{11}&a_{12}&\cdots&ca_{1j}&\cdots&a_{1n}\\
      a_{21}&a_{22}&\cdots&ca_{2j}&\cdots&a_{2n}\\
      \vdots&\vdots&&\vdots&&\vdots\\
      a_{n1}&a_{n2}&\cdots&ca_{nj}&\cdots&a_{nn}
    \end{vmatrix}
  \]
  根据定义\ref{dfn:n-order-determinant2},
  \begin{equation}\label{eq:n-order-determinant6}
    |B|=a_{11}N_{11}-a_{21}N_{21}+\cdots+(-1)^{n+1}a_{n1}N_{n1},
  \end{equation}
  上式中$N_{i1}$是由$M_{i1}$某一列乘以$c$得到,
  由归纳假设可知,
  \begin{equation}\label{eq:n-order-determinant7}
    N_{i1}=c\cdot M_{i1}, \forall i.
  \end{equation}
  将\eqref{eq:n-order-determinant7}式代入\eqref{eq:n-order-determinant6}式,
  可得
  \[
    |B|=c(a_{11}M_{11}-a_{21}M_{21}+\cdots+(-1)^{n+1}a_{n1}M_{n1})
    =c\cdot |A|.
  \]
\end{proof}

\subparagraph{\color{ecolor}性质3}
若$|A|$有一行(列)元素全为$0$,则$|A|=0$.

\begin{proof}
  \[
    |A|=
    \begin{vmatrix}
      a_{11}&a_{12}&\cdots&a_{1n}\\
      a_{21}&a_{22}&\cdots&a_{2n}\\
      \hdotsfor{4}\\
      0&0&\cdots&0\\
      a_{n1}&a_{n2}&\cdots&a_{nn}
    \end{vmatrix} \xlongequal {\text{根据性质2}}
    0\cdot |A| =0.
  \]
\end{proof}

\subparagraph{\color{ecolor}性质4}
对换$|A|$的两个不同行,所得行列式$|B|=-|A|$.

\begin{proof}
  当$n=2$时,容易验证结论成立.
  设结论对$n-1$阶行列式成立,下面证
  结论对$n$阶行列式成立.

  $1^{\circ}.$ 对换相邻两行,即第$i$行与$i+1$行,
  此时
  \[
    |B|=
    \begin{vmatrix}
      a_{11}&a_{12}&\cdots&a_{1n}\\
      \hdotsfor{4}\\
      a_{i+1,1}&a_{i+1,2}&\cdots&a_{i+1,n}\\
      a_{i1}&a_{i2}&\cdots&a_{in}\\
      \hdotsfor{4}\\
      a_{n1}&a_{n2}&\cdots&a_{nn}
    \end{vmatrix},
  \]
  根据定义\ref{dfn:n-order-determinant4},
  \begin{equation}\label{eq:n-order-determinant8}
    |B|= a_{11}N_{11}-a_{21}N_{21}+(-1)^{i+1}a_{i+1,1}N_{i1}+
    (-1)^{i+2}a_{i1}N_{i+1,1}+\cdots+(-1)^{n+1}a_{n1}N_{n1}.
  \end{equation}
  当$k\neq i, i+1$时, $N_{k1}$由$M_{k1}$对换相邻两行得到,
  由归纳假设可知,
  \begin{equation}\label{eq:n-order-determinant9}
    N_{k1}=-M_{k1}, \forall k\neq i, i+1.
  \end{equation}
  当$k=i,i+1$时,
  \begin{equation}\label{eq:n-order-determinant10}
    N_{i1}=M_{i+1,1}, N_{i+1,1}=M_{i,1}.
  \end{equation}
  将\eqref{eq:n-order-determinant9}式和\eqref{eq:n-order-determinant10}式
  都代入\eqref{eq:n-order-determinant8}式中,可得
  \[
    |B|=-(a_{11}M_{11}-a_{21}M_{21}+\cdots+(-1)^{n+1}a_{n1}M_{n1})
    = -|A|.
  \]

  $2^{\circ}.$ 对换$|A|$的第$i$行和第$j$行$(i<j)$.
  因为普通的行对换等价于一系列的相邻行对换的复合,所以
  \[
    \begin{matrix}
      i\\i+1\\\vdots\\j-1\\j
    \end{matrix} \xLongrightarrow {j-i\text{次相邻行对换}}
    \begin{matrix}
      i+1\\\cdots\\j-1\\j\\i
    \end{matrix} \xLongrightarrow {j-i-1\text{次相邻行对换}}
    \begin{matrix}
      j\\i+1\\\vdots\\j-1\\i
    \end{matrix}
  \]
  从上面行指标的对换过程可以看出,任何第$i$行和第$j$行做对换,
  等价于做了$2(j-i)-1$次相邻行对换,根据$1^{\circ}$可得
  \[
    |B|=(-1)^{2(j-1)-1}|A|=-|A|.
  \]
\end{proof}

\subparagraph{\color{ecolor}性质5}

若$|A|$的两行成比例(相等),则
\[
|A|=0.
\]

\begin{proof}
  设$|A|$的第$j$行的元素是第$i$行对应元素的$c$倍,
  则
  \begin{equation}\label{eq:n-order-determinant11}
    |A|=
    \begin{vmatrix}
      a_{11}&a_{12}&\cdots&a_{1n}\\
      \hdotsfor{4}\\
      a_{i1}&a_{i2}&\cdots&a_{in}\\
      \hdotsfor{4}\\
      ca_{i1}&ca_{i2}&\cdots&ca_{in}\\
      \hdotsfor{4}\\
      a_{n1}&a_{n2}&\cdots&a_{nn}
    \end{vmatrix}\xlongequal {\text{根据性质2}}
    c\cdot
    \begin{vmatrix}
            a_{11}&a_{12}&\cdots&a_{1n}\\
      \hdotsfor{4}\\
      a_{i1}&a_{i2}&\cdots&a_{in}\\
      \hdotsfor{4}\\
      a_{i1}&a_{i2}&\cdots&a_{in}\\
      \hdotsfor{4}\\
      a_{n1}&a_{n2}&\cdots&a_{nn}
    \end{vmatrix}.
  \end{equation}
  根据性质4,交换上式最后一个矩阵的第$i$行和第$j$行,得
  \[
        \begin{vmatrix}
        a_{11}&a_{12}&\cdots&a_{1n}\\
      \hdotsfor{4}\\
      a_{i1}&a_{i2}&\cdots&a_{in}\\
      \hdotsfor{4}\\
      a_{i1}&a_{i2}&\cdots&a_{in}\\
      \hdotsfor{4}\\
      a_{n1}&a_{n2}&\cdots&a_{nn}
      \end{vmatrix} = -
      \begin{vmatrix}
       a_{11}&a_{12}&\cdots&a_{1n}\\
      \hdotsfor{4}\\
      a_{i1}&a_{i2}&\cdots&a_{in}\\
      \hdotsfor{4}\\
      a_{i1}&a_{i2}&\cdots&a_{in}\\
      \hdotsfor{4}\\
      a_{n1}&a_{n2}&\cdots&a_{nn}
      \end{vmatrix} \Longrightarrow
          \begin{vmatrix}
            a_{11}&a_{12}&\cdots&a_{1n}\\
      \hdotsfor{4}\\
      a_{i1}&a_{i2}&\cdots&a_{in}\\
      \hdotsfor{4}\\
      a_{i1}&a_{i2}&\cdots&a_{in}\\
      \hdotsfor{4}\\
      a_{n1}&a_{n2}&\cdots&a_{nn}
    \end{vmatrix} =0.
  \]
  将上式结论代入\eqref{eq:n-order-determinant11}式,得
  \[
   |A|=c\times 0 = 0.
  \]
\end{proof}

\subparagraph{\color{ecolor}性质6}

\[
  \begin{vmatrix}
    a_{11}&a_{12}&\cdots&a_{1n}\\
    \hdotsfor{4}\\
    a_{r1}+b_{r1}&a_{r2}+b_{r2}&\cdots&a_{rn}+b_{rn}\\
    \hdotsfor{4}\\
    a_{n1}&a_{n2}&\cdots&a_{nn}
  \end{vmatrix} =
  \begin{vmatrix}
    a_{11}&a_{12}&\cdots&a_{1n}\\
    \hdotsfor{4}\\
    a_{r1}&a_{r2}&\cdots&a_{rn}\\
    \hdotsfor{4}\\
    a_{n1}&a_{n2}&\cdots&a_{nn}    
  \end{vmatrix} +
  \begin{vmatrix}
    a_{11}&a_{12}&\cdots&a_{1n}\\
    \hdotsfor{4}\\
    b_{r1}&b_{r2}&\cdots&b_{rn}\\
    \hdotsfor{4}\\
    a_{n1}&a_{n2}&\cdots&a_{nn}    
  \end{vmatrix}.
\]

\begin{proof}
  当$n=1$时, $|a_{11}+b_{11}|=|a_{11}|+|b_{11}|$,结论成立.
  设结论对$n-1$阶行列式成立,下面证明$n$阶的情形.
  设结论等式左边的行列式为$|C|$,其余子式为$Q_{ij}$;等式右边两个行列式分别为
  $|A|$和$|B|$,它们的余子式分别为$M_{ij}$和$N_{ij}$.根据定义
  \ref{dfn:n-order-determinant2},
  \begin{equation}\label{eq:n-order-determinant12}
    |C|=a_{11}Q_{11}-a_{21}Q_{21}+\cdots+(-1)^{r+1}(a_{r1}+b_{r1})Q_{r1}
    +\cdots+(-1)^{n+1}a_{n1}Q_{n1},
  \end{equation}
  当$k\neq r$时, $Q_{k1}, M_{k1},N_{k1}$满足性质6中的条件,
  由归纳假设可知
  \begin{equation}\label{eq:n-order-determinant13}
    Q_{k1}=M_{k1}+N_{k1}, \forall k\neq r.
  \end{equation}
  当$k=r$时,
  \begin{equation}\label{eq:n-order-determinant14}
    Q_{r1}=M_{r1}=N_{r1}.
  \end{equation}
  将\eqref{eq:n-order-determinant13}式和\eqref{eq:n-order-determinant14}式
  代入\eqref{eq:n-order-determinant12}式,
  \begin{align*}
    |C|= & (a_{11}M_{11}-a_{21}M_{21}+\cdots+(-1)^{r+1}a_{r1}M_{r1}+\cdots+
    (-1)^{n+1}a_{n1}M_{n1})+\\
    & (a_{11}N_{11}-a_{21}N_{21}+\cdots+(-1)^{r+1}b_{r1}N_{r1}+\cdots+
      (-1)^{n+1}a_{n1}N_{n1})\\
    = & |A|+|B|.
  \end{align*}
\end{proof}

\subparagraph{\color{ecolor}性质7}

\[
  \begin{vmatrix}
    a_{11}&a_{12}&\cdots&a_{1n}\\
    \vdots&\vdots&&\vdots\\
    a_{i1}&a_{i2}&\cdots&a_{in}\\
    \vdots&\vdots&&\vdots\\
    ca_{i1}+a_{j1}&ca_{i2}+a_{j2}&\cdots&ca_{in}+a_{jn}\\
    \vdots&\vdots&&\vdots\\
    a_{n1}&a_{n2}&\cdots&a_{nn}
  \end{vmatrix} =
  \begin{vmatrix}
    a_{11}&a_{12}&\cdots&a_{1n}\\
    \vdots&\vdots&&\vdots\\
    a_{i1}&a_{i2}&\cdots&a_{in}\\
    \vdots&\vdots&&\vdots\\
    a_{j1}&a_{j2}&\cdots&a_{jn}\\
    \vdots&\vdots&&\vdots\\    
    a_{n1}&a_{n2}&\cdots&a_{nn}
  \end{vmatrix}
\]

\begin{proof}
  设等式左边的行列式为$|B|$,等式右边的行列式为$|A|$.
  根据性质6对$|B|$进行拆分,
  \[
    |B|=
  \begin{vmatrix}
    a_{11}&a_{12}&\cdots&a_{1n}\\
    \vdots&\vdots&&\vdots\\
    a_{i1}&a_{i2}&\cdots&a_{in}\\
    \vdots&\vdots&&\vdots\\
    a_{j1}&a_{j2}&\cdots&a_{jn}\\
    \vdots&\vdots&&\vdots\\    
    a_{n1}&a_{n2}&\cdots&a_{nn}
  \end{vmatrix} +
  \begin{vmatrix}
    a_{11}&a_{12}&\cdots&a_{1n}\\
    \vdots&\vdots&&\vdots\\
    a_{i1}&a_{i2}&\cdots&a_{in}\\
    \vdots&\vdots&&\vdots\\
    ca_{i1}&ca_{i2}&\cdots&ca_{in}\\
    \vdots&\vdots&&\vdots\\    
    a_{n1}&a_{n2}&\cdots&a_{nn}
  \end{vmatrix},
  \]
  根据性质5,上式最后一个行列式的值为$0$,因此
  \[
    |B|=|A|.
  \]
\end{proof}

\subparagraph{\color{ecolor}性质5'}

若$|A|$有两列成比例(相等),则
\[
  |A|=0.
\]

\begin{proof}
  对阶数$n$进行归纳.当$n=2$时,容易验证结论成立.
  设对$n-1$阶行列式结论成立,现证$n$阶的情形.
  两列成比例情况可以转化为相等的情况,
  因此只需证明两列相等的情形.

  Case 1: 相等的两列都不是第一列.
  \begin{equation}\label{eq:n-order-determinant15}
    |A|=a_{11}M_{11}-a_{21}M_{21}+\cdots+(-1)^{n+1}a_{n1}M_{n1}.
  \end{equation}
  $\forall i$, $M_{i1}$都有两列相等.
  由归纳假设可知,
  \[
    M_{i1}=0, \forall i.
  \]
  将上式代入\eqref{eq:n-order-determinant15}式,可得
  \[
    |A|=0.
  \]

  Case 2: 第$1$列和第$r$列相等.
  如果第一列全部为$0$,结论显然成立,下面证第一列
  不全为$0$的情形,例如设$a_{s1}\neq 0$.
  \[
    |A|=
    \begin{vmatrix}
      a_{11}&\cdots&a_{11}&\cdots&a_{1n}\\
      \vdots&&\vdots&&\vdots\\
      a_{s1}&\cdots&a_{s1}&\cdots&a_{sn}\\
      \vdots&&\vdots&&\vdots\\
      a_{n1}&\cdots&a_{n1}&\cdots&a_{nn}
    \end{vmatrix} \xlongequal
    [\text{第$s$行乘以$-\frac{a_{i1}}{a_{s1}}$后加到第$i$行上}]{\text{根据性质7}} 
    |C|=
    \begin{vmatrix}
      0&\cdots&0&\cdots&*\\
      \vdots&&\vdots&&\vdots\\
      a_{s1}&\cdots&a_{s1}&\cdots&*\\
      \vdots&&\vdots&&\vdots\\
      0&\cdots&0&\cdots&*
    \end{vmatrix}.
  \]
  上式中,$|C|$第$1$列和第$r$列都只有$a_{s1}$非零,将$|C|$按第$1$列展开,
  \[
    |A|=|C|=(-1)^{s+1}a_{s1}Q_{s1},
  \]
  $Q_{i1}$为$|C|$按第$1$列展开时的余子式,
  $Q_{s1}$中,由于第$s$行全去掉了,因此,有一列全部为$0$,
  因此根据性质3, $Q_{s1}=0$,从而
  \[
    |A|=|C|=0.
  \]
\end{proof}

\subparagraph{\color{ecolor}性质6'}

\[
  \begin{vmatrix}
    a_{11}&\cdots&a_{1r}+b_{1r}&\cdots&a_{1n}\\
    a_{21}&\cdots&a_{2r}+b_{2r}&\cdots&a_{2n}\\
    \vdots&&\vdots&&\vdots\\
    a_{n1}&\cdots&a_{nr}+b_{nr}&\cdots&a_{nn}
  \end{vmatrix} =
  \begin{vmatrix}
    a_{11}&\cdots&a_{1r}&\cdots&a_{1n}\\
    a_{21}&\cdots&a_{2r}&\cdots&a_{2n}\\
    \vdots&&\vdots&&\vdots\\
    a_{n1}&\cdots&a_{nr}&\cdots&a_{nn}    
  \end{vmatrix} +
  \begin{vmatrix}
    a_{11}&\cdots&b_{1r}&\cdots&a_{1n}\\
    a_{21}&\cdots&b_{2r}&\cdots&a_{2n}\\
    \vdots&&\vdots&&\vdots\\
    a_{n1}&\cdots&b_{nr}&\cdots&a_{nn}    
  \end{vmatrix}.
\]

\begin{proof}
  $n=1$时显然成立.设阶数为$n-1$时结论成立,
  现证$n$阶时的情形.设上面等式左边的行列式为$|C|$,
  其余子式为$Q_{ij}$;等式右边第一个行列式为$|A|$,
  其行列式为$M_{ij}$;等式右边第二个行列式为$|B|$,
  其行列式为$N_{ij}$.

  Case 1: $r=1$时
  \begin{equation}\label{eq:n-order-determinant16}
    |C|=(a_{11}+b_{11})Q_{11}-(a_{21}+b_{21})Q_{21}+\cdots+
    (-1)^{n+1}(a_{n1}+b_{n1})Q_{n1}.
  \end{equation}
  由于求和的列位于第$1$列,因此第$1$列元素的余子式不包括求和的列,所以
  \[
    Q_{i1}=M_{i1}=N_{i1}, \forall i.
  \]
  将上式代入\eqref{eq:n-order-determinant16},得
  \[
    |C|=|A|+|B|.
  \]

  Case 2: $r>1$时
  \begin{equation}\label{eq:n-order-determinant17}
    |C|=a_{11}Q_{11}-a_{21}Q_{21}+\cdots+(-1)^{n+1}a_{n1}Q_{n1}.
  \end{equation}
  $Q_{i1},M_{i1},N_{i1}$满足性质6'中的条件,
  根据归纳假设可知
  \[
    Q_{i1}=M_{i1}+N_{i1},
  \]
  将上式代入\eqref{eq:n-order-determinant17}可得
  \[
    |C|=|A|+|B|.
  \]
\end{proof}

\subparagraph{\color{ecolor}性质7'}

\[
  \begin{vmatrix}
    a_{11}&\cdots&a_{1r}&\cdots&a_{1s}+ca_{1r}&\cdots&a_{1n}\\
    a_{21}&\cdots&a_{2r}&\cdots&a_{2s}+ca_{2r}&\cdots&a_{2n}\\
    \vdots&&\vdots&&\vdots&&\vdots\\
    a_{n1}&\cdots&a_{nr}&\cdots&a_{ns}+ca_{nr}&\cdots&a_{nn}
  \end{vmatrix} =
  \begin{vmatrix}
    a_{11}&\cdots&a_{1r}&\cdots&a_{1s}&\cdots&a_{1n}\\
    a_{21}&\cdots&a_{2r}&\cdots&a_{2s}&\cdots&a_{2n}\\
    \vdots&&\vdots&&\vdots&&\vdots\\
    a_{n1}&\cdots&a_{nr}&\cdots&a_{ns}&\cdots&a_{nn}
  \end{vmatrix}.
\]

\begin{proof}
  由性质6'+性质5'可证.
  \begin{align*}
  & \begin{vmatrix}
    a_{11}&\cdots&a_{1r}&\cdots&a_{1s}+ca_{1r}&\cdots&a_{1n}\\
    a_{21}&\cdots&a_{2r}&\cdots&a_{2s}+ca_{2r}&\cdots&a_{2n}\\
    \vdots&&\vdots&&\vdots&&\vdots\\
    a_{n1}&\cdots&a_{nr}&\cdots&a_{ns}+ca_{nr}&\cdots&a_{nn}
  \end{vmatrix}\\
    = & 
  \begin{vmatrix}
    a_{11}&\cdots&a_{1r}&\cdots&a_{1s}&\cdots&a_{1n}\\
    a_{21}&\cdots&a_{2r}&\cdots&a_{2s}&\cdots&a_{2n}\\
    \vdots&&\vdots&&\vdots&&\vdots\\
    a_{n1}&\cdots&a_{nr}&\cdots&a_{ns}&\cdots&a_{nn}
  \end{vmatrix}+
  \underbrace{\begin{vmatrix}
    a_{11}&\cdots&a_{1r}&\cdots&a_{1r}&\cdots&a_{1n}\\
    a_{21}&\cdots&a_{2r}&\cdots&a_{2r}&\cdots&a_{2n}\\
    \vdots&&\vdots&&\vdots&&\vdots\\
    a_{n1}&\cdots&a_{nr}&\cdots&a_{nr}&\cdots&a_{nn}
  \end{vmatrix}}_{\text{根据性质5',该行列式值为$0$}}\\
    = &
  \begin{vmatrix}
    a_{11}&\cdots&a_{1r}&\cdots&a_{1s}&\cdots&a_{1n}\\
    a_{21}&\cdots&a_{2r}&\cdots&a_{2s}&\cdots&a_{2n}\\
    \vdots&&\vdots&&\vdots&&\vdots\\
    a_{n1}&\cdots&a_{nr}&\cdots&a_{ns}&\cdots&a_{nn}
  \end{vmatrix}.  
  \end{align*}
\end{proof}

\subparagraph{\color{ecolor}性质4'}

对换$|A|$的两个不同列,得到的行列式$|B|=-|A|$.

\begin{proof}
  设$|B|$是对换$|A|$的第$r$列与第$s$列所得,
  \[
    |A|=
    \begin{vmatrix}
      a_{11}&\cdots&a_{1r}&\cdots&a_{1s}&\cdots&a_{1n}\\
      a_{21}&\cdots&a_{2r}&\cdots&a_{2s}&\cdots&a_{2n}\\
      \vdots&&\vdots&&\vdots&&\vdots\\
      a_{n1}&\cdots&a_{nr}&\cdots&a_{ns}&\cdots&a_{nn}
    \end{vmatrix},
    |B|=
    \begin{vmatrix}
      a_{11}&\cdots&a_{1s}&\cdots&a_{1r}&\cdots&a_{1n}\\
      a_{21}&\cdots&a_{2s}&\cdots&a_{2r}&\cdots&a_{2n}\\
      \vdots&&\vdots&&\vdots&&\vdots\\
      a_{n1}&\cdots&a_{ns}&\cdots&a_{nr}&\cdots&a_{nn}
    \end{vmatrix}.
  \]
  构造一个新行列式$|C|$:
  \[
    |C|=
    \begin{vmatrix}
      a_{11}&\cdots&a_{1r}+a_{1s}&\cdots&a_{1r}+a_{1s}&\cdots&a_{1n}\\
      a_{21}&\cdots&a_{2r}+a_{2s}&\cdots&a_{2r}+a_{2s}&\cdots&a_{2n}\\
      \vdots&&\vdots&&\vdots&&\vdots\\
      a_{n1}&\cdots&a_{nr}+a_{ns}&\cdots&a_{nr}+a_{ns}&\cdots&a_{nn}
    \end{vmatrix}\xlongequal{\text{根据性质5'}} 0.
  \]
  利用性质6'将$|C|$拆分成4个行列式:
  \begin{align*}
    |C|= &
    \begin{vmatrix}
      a_{11}&\cdots&a_{1r}&\cdots&a_{1r}&\cdots&a_{1n}\\
      a_{21}&\cdots&a_{2r}&\cdots&a_{2r}&\cdots&a_{2n}\\
      \vdots&&\vdots&&\vdots&&\vdots\\
      a_{n1}&\cdots&a_{nr}&\cdots&a_{nr}&\cdots&a_{nn}
    \end{vmatrix} +
    \begin{vmatrix}
      a_{11}&\cdots&a_{1s}&\cdots&a_{1s}&\cdots&a_{1n}\\
      a_{21}&\cdots&a_{2s}&\cdots&a_{2s}&\cdots&a_{2n}\\
      \vdots&&\vdots&&\vdots&&\vdots\\
      a_{n1}&\cdots&a_{ns}&\cdots&a_{ns}&\cdots&a_{nn}
    \end{vmatrix} + |A|+|B|\\
    = & |A|+|B|\\
    = & 0\\
    \Longrightarrow & |B|=-|A|.
  \end{align*}
\end{proof}


\subsection{行列式的展开和转置}
\begin{theorem}\label{thm:Determinant1}
  若
  \begin{equation}
    \label{eq:Determinant1}
  |A|=
  \begin{vmatrix}
    a_{11} & a_{12} & \cdots & a_{1n}\\
    a_{21} & a_{22} & \cdots & a_{2n}\\
    \vdots & \vdots & & \vdots\\
    a_{n1} & a_{n2} & \cdots & a_{nn}
  \end{vmatrix},
\end{equation}
第$i$行第$j$列元素$a_{ij}$的代数余子式记为$A_{ij}$,
则对任意的$r(r=1,2,\cdots,n)$列有展开式:
  \begin{equation}
    \label{eq:Determinant2}
    |A|=a_{1r}A_{1r}+ a_{2r}A_{2r}+\cdots+a_{nr}A_{nr}.
  \end{equation}
  又对任意的$s \neq r$,有
  \begin{equation}
    \label{eq:Determinant3}
    a_{1r}A_{1s}+a_{2r}A_{2s}+\cdots+a_{nr}A_{ns}=0
  \end{equation}
\end{theorem}

\begin{proof}
  先证$s=r$的情况.

  方法一:将第$r$列和前面相邻列逐次对换,
  直到第$r$列对换到第$1$列,此时的行列式为$|B|$.
  设$|A|$的余子式为$M_{ij}$, $|B|$的余子式为$N_{ij}$.
  \begin{align*}
    & |A|=
    \begin{vmatrix}
      a_{11}&\cdots&a_{1,r-1}&a_{1r}&\cdots&a_{1n}\\
      a_{21}&\cdots&a_{2,r-1}&a_{2r}&\cdots&a_{2n}\\
      \vdots&&\vdots&\vdots&&\vdots\\
      a_{n1}&\cdots&a_{n,r-1}&a_{nr}&\cdots&a_{nn}
    \end{vmatrix}\\
    & \xlongrightarrow {|A|\text{经过$r-1$次相邻两列对换}} |B|=
    \begin{vmatrix}
      a_{1r}&a_{11}&\cdots&a_{1,r-1}&\cdots&a_{1n}\\
      a_{2r}&a_{21}&\cdots&a_{2,r-1}&\cdots&a_{2n}\\
      \vdots&\vdots&&\vdots&&\vdots\\
      a_{nr}&a_{n1}&\cdots&a_{n,r-1}&\cdots&a_{nn}
    \end{vmatrix}\\
    & \Longrightarrow
      |B|= a_{1r}N_{11}-a_{2r}N_{21}+\cdots+(-1)^{n+1}a_{nr}N_{n1}.
  \end{align*}
  $|A|$经过$r-1$次相邻两列对换得到$|B|$,
  故$|A|=(-1)^{r-1}|B|$.又因为$\forall i, N_{i1}=M_{ir}$,
  代入$|B|$的表达式,得
  \begin{align*}
    |A|& =(-1)^{r-1}|B|=
         (-1)^{r-1}(a_{1r}M_{1r}-a_{2r}M_{2r}+\cdots+(-1)^{n+1}a_{nr}M_{nr})\\
       & = (-1)^{r+1}a_{1r}M_{1r}+(-1)^{r+2}a_{2r}M_{2r}+\cdots+(-1)^{r+n}a_{nr}M_{nr}\\
    & = a_{1r}A_{1r}+a_{2r}A_{2r}+\cdots+a_{nr}A_{nr}
  \end{align*}
  上面推算中, $(-1)^{r-1}=(-1)^{r+1}$.

  方法二:利用性质6'将第$r$列进行拆分.
  \begin{align*}
      |A| = &
    \begin{vmatrix}
    a_{11} & \cdots & a_{1r}& \cdots & a_{1n}\\
    a_{21} & \cdots & 0 & \cdots & a_{2n}\\
    \vdots & \vdots & \vdots & \vdots & \vdots\\
    a_{n1} & \cdots & 0 & \cdots &a_{nn}
    \end{vmatrix} +
    \begin{vmatrix}
    a_{11} & \cdots & 0 & \cdots & a_{1n}\\
    a_{21} & \cdots & a_{2r} & \cdots & a_{2n}\\
    \vdots & \vdots & \vdots & \vdots & \vdots\\
    a_{n1} & \cdots & 0 & \cdots &a_{nn}
  \end{vmatrix} + \cdots +
  \begin{vmatrix}
    a_{11} & \cdots & 0 & \cdots & a_{1n}\\
    a_{21} & \cdots & 0 & \cdots & a_{2n}\\
    \vdots & \vdots & \vdots & \vdots & \vdots\\
    a_{n1} & \cdots & a_{nr} & \cdots &a_{nn}
  \end{vmatrix}\\
  = & a_{1r}A_{1r}+ a_{2r}A_{2r}+\cdots+a_{nr}A_{nr}.
\end{align*}
所以,\eqref{eq:Determinant2}式成立。

再证$s=r$的情况.

在\eqref{eq:Determinant1}式基础上,将其第s列换成第r列,构建一个新行列式,其值为0,即:
\begin{equation*}
  \begin{vmatrix}
    a_{11} & \cdots & a_{1r} & \cdots & a_{1r} & \cdots & a_{1n}\\
    a_{21} & \cdots & a_{2r} & \cdots & a_{2r} & \cdots & a_{2n}\\
    \vdots && \vdots && \vdots && \vdots\\
    a_{n1} & \cdots & a_{nr} & \cdots & a_{nr} & \cdots & a_{nn}
  \end{vmatrix}=0
\end{equation*}
将上面行列式按第s列展开:
\begin{equation*}
  a_{1r}A_{1s}+a_{2r}A_{2s}+\cdots+a_{nr}A_{ns}=0
\end{equation*}
所以\eqref{eq:Determinant3}式成立。

同理可证,按照任意$r(r=1,2,\cdots,n)$行展开有展开式:
\begin{equation}
    \label{eq:Determinant4}
    |A|=a_{r1}A_{1r}+ a_{r2}A_{r2}+\cdots+a_{rn}A_{rn}.
  \end{equation}
  又对任意的$s \neq r$,有
  \begin{equation}
    \label{eq:Determinant5}
    a_{r1}A_{s1}+a_{r2}A_{s2}+\cdots+a_{rn}A_{sn}=0
  \end{equation}
\end{proof}

\subsection{Gramer法则}

\begin{theorem}[Gramer法则]\label{thm:Gramer}
  设有方程组
  \begin{equation}\label{eq:Determinant6}
    \begin{cases}
      a_{11}x_1+a_{12}x_2+\cdots+a_{1n}x_n=b_1,\\
      a_{21}x_1+a_{22}x_2+\cdots+a_{2n}x_n=b_2,\\
      \hspace{5em}\cdots\cdots\cdots\cdots\\
      a_{n1}x_1+a_{n2}x_2+\cdots+a_{nn}x_n=b_n,\\
    \end{cases}
  \end{equation}
  记这个方程组的系数行列式为$|A|$,若$|A|\neq 0$, 则方程组有且仅有一组解:
  \begin{align}\label{eq:Determinant7}
    x_1=\frac{|A_1|}{|A|},x_2=\frac{|A_2|}{|A|},
    \cdots,x_n=\frac{|A_n|}{|A|},
  \end{align}
  其中$|A_j|(j=1,2,\cdots,n)$是一个$n$阶行列式,
  它由$|A|$去掉第$j$列换上方程组的常数项$b_1$, $b_2$, $\cdots$, $b_n$组成的列而成.
\end{theorem}

\begin{proof}

  方法一:只需验证\eqref{eq:Determinant7}式确为方程组\eqref{eq:Determinant6}的解.

  将$|A_j|$按第$j$列展开,得
  \begin{equation}\label{eq:Determinant8}
    |A_j|=b_1A_{1j}+b_2A_{2j}+\cdots+b_nA_{nj}=\sum\limits_{i=1}^nb_iA_{ij},
  \end{equation}

  根据式\eqref{eq:Determinant7}和式\eqref{eq:Determinant8}可得,
  \begin{equation}\label{eq:Determinant9}
    x_j=\frac{|A_j|}{|A|}=\frac{1}{|A|}\sum\limits_{i=1}^nb_iA_{ij},
    j=1,2,\cdots,n.
  \end{equation}

  方程组\eqref{eq:Determinant6}第$k$个方程为
  \begin{equation}\label{eq:Determinant10}
    \sum\limits_{j=1}^na_{kj}x_j=b_k, 1 \leq k \leq n.
  \end{equation}
  将\eqref{eq:Determinant9}代入\eqref{eq:Determinant10}左边得
  \begin{equation*}
    \begin{split}
    \sum\limits_{j=1}^na_{kj}x_j & =
    \sum\limits_{j=1}^n(a_{kj}\frac{1}{|A|}\sum\limits_{i=1}^nb_iA_{ij})\\
    & = \frac{1}{|A|}\sum\limits_{j=1}^n\sum\limits_{i=1}^na_{kj}b_iA_{ij}\\
    & = \frac{1}{|A|}\sum\limits_{i=1}^nb_i(\sum\limits_{j=1}^na_{kj}A_{ij})
    \end{split}
  \end{equation*}

  上式最后一个等式里$\sum\limits_{j=1}^na_{kj}A_{ij}$部分的值如下:

  当$i \neq k$时,为行列式$|A|$第$k$行每个元素与第$i$行同列元素对应的代数余子式的乘积之和,
  其值为0;

  当$i = k$时,为行列式$|A|$按第k行元素展开,
  即第k行每个元素与其对应代数余子式的乘积之和,
  其值为$|A|$.所以,
  \begin{equation*}
    \sum\limits_{j=1}^na_{kj}x_j
    = \frac{1}{|A|}b_k\sum\limits_{j=1}^na_{kj}A_{kj}
    = \frac{b_k}{|A|}|A|
    = b_k
  \end{equation*}

  由于$k$为任意$1$至$n$的自然数,
  因此\eqref{eq:Determinant7}式是\eqref{eq:Determinant6}式的解.

  方法二:利用系数矩阵求逆的方法。
  方程组\eqref{eq:Determinant6}可以写成矩阵形式
  \begin{equation*}
    Ax=\beta,
  \end{equation*}
  其中,$A=(a_{ij})$.若$|A| \neq 0$,则$A^-1$必存在,因此
  \begin{equation*}
    A^{-1}(Ax)=A^{-1}\beta, 
  \end{equation*}
  即
  \begin{equation*}
    x=A^{-1}\beta
  \end{equation*}
  将上式中的矩阵写出来就是
  \begin{equation*}
    \begin{pmatrix}
      x_1\\
      x_2\\
      \vdots\\
      x_n
    \end{pmatrix}=\frac{1}{|A|}
    \begin{pmatrix}
      A_{11} & A_{21} & \cdots & A_{n1}\\
      A_{12} & A_{22} & \cdots & A_{n2}\\
      \vdots & \vdots & & \vdots\\
      A_{1n} & A_{2n} & \cdots & A_{nn}
    \end{pmatrix}
    \begin{pmatrix}
      b_1\\
      b_2\\
      \vdots\\
      b_n
    \end{pmatrix}.
  \end{equation*}
  于是
  \begin{equation*}
    x_1=\frac{1}{|A|}(b_1A_{11}+b_2A_{21}+\cdots+b_nA_{n1})
  \end{equation*}
其中
\begin{equation*}
  |A_1|=
  \begin{vmatrix}
    b_1 & a_{12} & \cdots & a_{1n}\\
    b_2 & a_{22} & \cdots & a_{2n}\\
    \vdots & \vdots & & \vdots\\
    b_n & a_{n2} & \cdots & a_{nn}
  \end{vmatrix}.
\end{equation*}
其余$x_i$同理可得.
\end{proof}

\subsection{行列式的计算}
\begin{example}
  计算$n$阶$Vander Monde$(范德蒙德)行列式:
  \begin{equation*}
    V_n=
    \begin{vmatrix}
      1 & x_1 & x_1^2 & \cdots & x_1^{n-2} & x_1^{n-1}\\[3pt]
      1 & x_2 & x_2^2 & \cdots & x_2^{n-2} & x_2^{n-1}\\[3pt]
      \vdots & \vdots & \vdots & & \vdots & \vdots\\[3pt]
      1 & x_{n-1} & x_{n-1}^2 & \cdots & x_{n-1}^{n-2} & x_{n-1}^{n-1}\\[3pt]
      1 & x_{n-2} & x_{n-2}^2 & \cdots & x_{n-2}^{n-2} & x_{n-2}^{n-1}
    \end{vmatrix}
  \end{equation*}
\end{example}
\begin{solution}
  将第$n-1$列乘以$-x_n$后加到第$n$列上,再将第$n-2$列乘以$-x_n$
  加到第$n-1$列上.这样一直做下去,直到将第一列乘以$-x_n$加到第二列上为止.
  这样变形后行列式值不变.得到
  \begin{equation*}
    V_n  =
    \begin{vmatrix}
    1 & x_1-x_n & x_1^2-x_1x_n & \cdots & x_1^{n-2}-x_1^{n-3}x_n & x_1^{n-1}-x_1^{n-2}x_n\\[3pt]
    1 & x_2-x_n & x_2^2-x_2x_n & \cdots & x_2^{n-2}-x_2^{n-3}x_n & x_2^{n-1}-x_2^{n-2}x_n\\[3pt]
    \vdots & \vdots & \vdots & & \vdots & \vdots\\[3pt]
    1 & x_{n-1}-x_n & x_{n-1}^2-x_{n-1}x_n & \cdots & x_{n-1}^{n-2}-x_{n-1}^{n-3}x_n & x_{n-1}^{n-1}-x_{n-1}^{n-2}x_n\\[3pt]
    1 & 0 & 0 & \cdots & 0 & 0
    \end{vmatrix}
  \end{equation*}
  \begin{equation*}
   = (-1)^{n+1}
    \begin{vmatrix}
    x_1-x_n & x_1(x_1-x_n) & \cdots & x_1^{n-3}(x_1-x_n) & x_1^{n-2}(x_1-x_n)\\[3pt]
    x_2-x_n & x_2(x_2-x_n) & \cdots & x_2^{n-3}(x_2-x_n) & x_2^{n-2}(x_2-x_n)\\[3pt]
    \vdots & \vdots &  & \vdots & \vdots\\[3pt]
    x_{n-1}-x_n & x_{n-1}(x_{n-1}-x_n) & \cdots & x_{n-1}^{n-3}(x_{n-1}-x_n) & x_{n-1}^{n-2}(x_{n-1}-x_n)\\[3pt]
    \end{vmatrix}.
  \end{equation*}
  提取每行公因子后得到$n-1$阶行列式恰好是一个
  $x_1,x_2,\cdots,x_{n-1}$的$n-1$阶$Vander Monde$行列式,
  将其标记为$V_{n-1}$.得到递推公式
  \begin{equation*}
    \begin{split}
    V_n &  =
    (-1)^{n+1}(x_1-x_n)(x_2-x_n)\cdots(x_{n-1}-x_n)\cdot
    \begin{vmatrix}
      1 & x_1 & x_1^2 & \cdots & x_1^{n-2}\\[3pt]
      1 & x_2 & x_2^2 & \cdots & x_2^{n-2}\\[3pt]
      \vdots & \vdots & \vdots & & \vdots\\[3pt]
      1 & x_{n-1} & x_{n-1}^2 & \cdots & x_{n-1}^{n-2}\\[3pt]
      1 & x_{n-2} & x_{n-2}^2 & \cdots & x_{n-2}^{n-2}
    \end{vmatrix}\\
    & = (x_n-x_1)(x_n-x_2)\cdots(x_n-x_{n-1})V_{n-1}
   \end{split}
  \end{equation*}
   即
   \begin{equation*}
     V_n= \prod_{1 \leq i <j \leq n}(x_j-x_i)
   \end{equation*}
\end{solution}

\begin{example}
  求下列行列式的值:
  \begin{equation*}
    F_n =
    \begin{vmatrix}
      \lambda & 0 & 0 & \cdots & 0 & a_n\\
      -1 & \lambda & 0 & \cdots & 0 & a_{n-1}\\
      0 & -1 & \lambda & \cdots & 0 & a_{n-2}\\
      \vdots & \vdots & \vdots &  & \vdots & \vdots\\
      0 & 0 & 0 & \cdots & \lambda & a_2\\
      0 & 0 & 0 & \cdots & -1 & \lambda + a_1
    \end{vmatrix}
  \end{equation*}
\end{example}

\begin{solution}
  按第一行展开, $a_n$的余子式是一个上三角行列式,其值等于$(-1)^{n-1}$,
  $\lambda$的余子式是与$F_n$相类似的$n-1$阶行列式,记之为$F_{n-1}$,
  于是得到递推关系式:
  \begin{equation*}
    F_n = \lambda F_{n-1}+(-1)^{1+n}(-1)^{n-1}a_n= \lambda F_{n-1}+a_n.
  \end{equation*}
  将递推公式不断代入,最后求得
  \begin{equation*}
    F_n=\lambda^n+a_1\lambda^{n-1}+a_2\lambda^{n-2}+\cdots+a_n.
  \end{equation*}
\end{solution}

% \subsection{\texorpdfstring{$Laplace$}{\textit{Laplace}}定理}
\subsection{Laplace定理}
\begin{theorem}[Laplace定理]\label{thm:Laplace}
  设$|A|$是$n$阶行列式,在$|A|$中任取$k$行(列),
  那么包含于这$k$行(列)的全部$k$阶子式与它们对应的代数余子式的成绩之和
  等于$|A|$.即若取定$k$个行:$1 \leq i_1 < i_2 < \cdots < i_k \leq n$,则
  \begin{equation}
    \label{eq:Determinant11}
    |A|=\sum\limits_{1 \leq j_1 < j_2 < \cdots < j_k \leq n}
    A\begin{pmatrix}i_1 & i_2 & \cdots & i_k & n\\
        j_1 & j_2 & \cdots & j_k & n\end{pmatrix}
    \widehat{A}\begin{pmatrix}i_1 & i_2 & \cdots & i_k & n\\
        j_1 & j_2 & \cdots & j_k & n\end{pmatrix}
  \end{equation}
  同样若取定$k$个列:$1 \leq j_1 < j_2 < \cdots < j_k \leq n$,则
  \begin{equation}
    \label{eq:Determinant12}
    |A|=\sum\limits_{1 \leq i_1 < i_2 < \cdots < i_k \leq n}
    A\begin{pmatrix}i_1 & i_2 & \cdots & i_k & n\\
        j_1 & j_2 & \cdots & j_k & n\end{pmatrix}
    \widehat{A}\begin{pmatrix}i_1 & i_2 & \cdots & i_k & n\\
        j_1 & j_2 & \cdots & j_k & n\end{pmatrix}
  \end{equation}
\end{theorem}

\begin{proof}
  如果能证明$|A|$的每个$k$阶子式与其代数余子式之积的展开式中的每一项
  都互不相同且都属于$|A|$的展开式,两者总项数相等,那么就证明
  了Laplace定理.

  首先证明$|A|$的任意$k$阶子式与其代数余子式之积的展开式中
  的每一项都属于$|A|$的展开式.

  先证明特殊情形:$i_1=1,i_2=2,\cdots,
  ,i_k=k;j_1=1,j_2=2,\cdots,j_k=k$.这时$|A|$可写为:
  \begin{equation}
    \label{eq:Determinant13}
    \begin{vmatrix}
      A_1 & *\\
      * & A_2
    \end{vmatrix},
  \end{equation}
  其中
  \begin{equation}
    \label{eq:Determinant14}
    |A_1|=A
    \left(\begin{smallmatrix}
      1 & 2 & \cdots & k\\
      1 & 2 & \cdots & k
    \end{smallmatrix}\right)=
    \begin{vmatrix}
      a_{11}& \cdots & a{1k}\\
      \vdots & & \vdots\\
      a_{k1}& \cdots & a_{kk}
    \end{vmatrix},
  \end{equation}
  \begin{equation}
    \label{eq:Determinant15}
    |A_2|=\widehat{A}
    \left(\begin{smallmatrix}
      1 & 2 & \cdots & k\\
      1 & 2 & \cdots & k
    \end{smallmatrix}\right)=
    \begin{vmatrix}
      a_{k+1,k+1}& \cdots & a{k+1,n}\\
      \vdots & & \vdots\\
      a_{n,k+1}& \cdots & a_{nn}
    \end{vmatrix}.
  \end{equation}
  \eqref{eq:Determinant14}式中的任一项具有形式:
  \begin{equation*}
    (-1)^{N(j_1,j_2,\cdots,j_k)}a_{j_11}a_{j_22}\cdots a_{j_kk},
  \end{equation*}
  其中$N(j_1,j_2,\cdots,j_k)$是排列$(j_1,j_2,\cdots,j_k)$的逆序数.
  \eqref{eq:Determinant15}式中的任一项具有形式:
  \begin{equation*}
    (-1)^{N(j_{k+1},j_{k+2},\cdots,j_n)}a_{j_{k+1},k+1}a_{j_{k+2},k+2}\cdots a_{j_nn},
  \end{equation*}
  所以
  \begin{equation*}
    A
    \left(\begin{smallmatrix}
      1 & 2 & \cdots & k\\
      1 & 2 & \cdots & k
    \end{smallmatrix}\right)
    \widehat{A}
    \left(\begin{smallmatrix}
      1 & 2 & \cdots & k\\
      1 & 2 & \cdots & k
    \end{smallmatrix}\right)
  \end{equation*}
  中的任一项具有下列形式:
  \begin{equation}\label{eq:Determinant16}
    (-1)^\sigma a_{j_11}a_{j_22}\cdots a_{j_kk}a_{j_{k+1}k+1}\cdots a_{j_nn},
  \end{equation}
  其中$\sigma = N(j_1,\cdots,j_k)+N(j_{k+1},\cdots,j_n)$. 
  注意$(j_1,\cdots,j_k)$是$(1,\cdots,k)$的一个排列, 
  ($j_{k+1}$, $\cdots$, $j_n$)是$(k+1,\cdots,n)$的一个排列,因此
  $(j_1,\cdots,j_k,j_{k+1},\cdots,j_n)$是$(1,2,\cdots,n)$的一个排列且
  \begin{equation*}    N(j_1,\cdots,j_k,j_{k+1},\cdots,j_n)=N(j_1,\cdots,j_k)+N(j_{k+1},\cdots,j_n).
  \end{equation*}
  因此,\eqref{eq:Determinant16}式是$|A|$中的某一项.

  再证明一般情况,设
  \[ 1 \leq i_1 < i_2 < \cdots < i_k \leq n;1 \leq j_1 < j_2 < \cdots < j_k \leq n.\]
  经过$i_1-1$次相邻两行的对换,可把$i_1$行调到第一行,
  经过$i_2-2$次相邻两行的对换,可把$i_2$行调到第二行, $\cdots$,
  经过$(i_1+\cdots+i_k)-\frac{1}{2}k(k+1)$次对换,可把
  第$i_1,i_2,\cdots,i_k$行调至前$k$行.同理,
  经过$(j_1+\cdots+j_k)-\frac{1}{2}k(k+1)$次对换,可把
  第$j_1,j_2,\cdots,j_k$行调至前$k$列.因此, $|A|$经过
  $(i_1+\cdots+i_k)+(j_1+\cdots+j_k)-k(k+1)$次行列的对换
  (其中$k(k+1)$必为偶数),得到了一个新的行列式:
  \[
    |C|=\begin{vmatrix}
        D & *\\
        * & B%
        \end{vmatrix}
  \]
  其中
  \[
    |D|=A
    \begin{pmatrix}
      i_1 & i_2 & \cdots & i_k\\
      j_1 & j_2 & \cdots & j_k
    \end{pmatrix}
  \]
  由以上分析可知, $|C|=(-1)^{p+q}|A|,p=i_1+\cdots+i_k,q=j_1+\cdots+j_k.$
  $|B|$是子式$|D|$在$|C|$中的代数余子式,由前面特殊情况的分析可知,
  $|D||B|$中的任一项都是$|C|$中的项.由于
  \[\widehat{A}
    \begin{pmatrix}
      i_1 & i_2 & \cdots & i_k\\
      j_1 & j_2 & \cdots & j_k
    \end{pmatrix}
    =(-1)^{p+q}|B|,\]
  因此,
  \begin{equation}\label{eq:Determinant17}
  A
    \begin{pmatrix}
      i_1 & i_2 & \cdots & i_k\\
      j_1 & j_2 & \cdots & j_k
    \end{pmatrix}
    \widehat{A}
    \begin{pmatrix}
      i_1 & i_2 & \cdots & i_k\\
      j_1 & j_2 & \cdots & j_k
    \end{pmatrix}\end{equation}
  中的任一项都是$(-1)^{p+q}|C|=|A|$中的项.

  以上分析了\eqref{eq:Determinant17}式中任一项均属于$|A|$的展开式。

  最后证明二者总项数相等:当$i_1,i_2,\cdots,i_k$固定时,对
  不同的$1 \leq j_1 < j_2 < \cdots < j_k \leq n$共有$C_n^k$个
  不同子式,每个子式完全展开后均含有$k!$项,相应的余子式也有$C_n^k$个,
  每个余子式展开后均含有$(n-k)!$项.由\eqref{eq:Determinant17}式
  展开得到的项是没有重复的,且一共有$C_n^k\cdot k!(n-1)!=n!$项. $|A|$的展开式也有$n!$项,
  因此\eqref{eq:Determinant11}式成立.\eqref{eq:Determinant12}式同理可证.
\end{proof}
%%% Local Variables:
%%% mode: latex
%%% TeX-master: "../main"
%%% End:
