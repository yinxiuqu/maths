\section{内积空间}

\subsection{内积空间的概念}

% \[
%   \mathbb{R}^3: \quad \vec{u} = (x_1, x_2, x_3), \quad
%                      \vec{v} = (y_1, y_2, y_3),
% \]

% \[
%   \text{点积(内积):} \quad \vec{u}\cdot\vec{v} = x_1y_1 + xy_2 + xy_3,
% \]

% \[
%   \text{长度:} \quad \Vert \vec{u}\Vert=(\vec{u}\cdot\vec{u})= \sqrt{x_1^2 + x_2^2 + x_3^2},
% \]

% \[
%   \text{$\vec{u},\vec{v}$的距离:} \quad \Vert\vec{u}\cdot\vec{v}\Vert
%   = \sqrt{(x_1-y_1)^2 + (x_2-y_2)^2 + (x_3-y_3)^2}.
% \]

% \subparagraph{\color{ecolor}$\mathbb{R}^3$ 中向量内积的性质}
% \begin{asparaenum}[(1)]
%     \item $\vec{u} \cdot \vec{v} = \vec{v} \cdot \vec{u}$;
%     \item $(\vec{u} + \vec{w}) \cdot \vec{v} = \vec{u} \cdot \vec{v} + \vec{w} \cdot \vec{v}$;
%     \item $(c\vec{u}) \cdot \vec{v} = c(\vec{u} \cdot \vec{v})$;
%     \item $\vec{u} \cdot \vec{u} \geq 0$, 且 $\vec{u} \cdot \vec{u} = 0$ 当且仅当$\vec{u} = 0$.
% \end{asparaenum}
% 其中 $ \vec{u}, \vec{v}, \vec{w} $ 是 $\mathbb{R}^3$ 中的任意向量, $c$ 是任一实数.

% \begin{definition}\label{dfn:concept1}
%   设$V$为实线性空间,若存在二元运算
%   \begin{align*}
%     (-,-): & V\times V \longrightarrow \mathbb{R}\\
%     & \bm{\alpha} \times \bm{\beta} \mapsto (\bm{\alpha}, \bm{\beta}) \text{且满足如下性质:}
%   \end{align*}
%   \begin{align*}
%     & (1) (\bm{\beta},\bm{\alpha})=(\bm{\alpha},\bm{\beta});\text{\color{red}对称性}\\
%     &\hspace{-0.3em}\left.\begin{aligned}
%     & (2) (\bm{\alpha}+\bm{\beta},\bm{\gamma})=(\bm{\alpha}+\bm{\gamma})+(\bm{\beta},\bm{\gamma});\\
%     & (3) (c\bm{\alpha},\bm{\beta})=c(\bm{\alpha},\bm{\beta}),c\text{为任一实数};
%     \end{aligned}\right\}\text{\color{red}第一变元的线性}\\
%     & (4) (\bm{\alpha},\bm{\alpha})\geq0\text{且等号成立当且仅当}\bm{\alpha}=0.\text{\color{red}正定性}
%   \end{align*}
%   则称$(-,-)$为$V$上的一个内积,$(\bm{\alpha},\bm{\beta})$称为向量$\bm{\alpha},\bm{\beta}$的内积.
%   给定一个内积结构的实线性空间称为实内积空间.有限维的内积空间称为Euclid空间(欧氏空间).
% \end{definition}

% \begin{definition}\label{dfn:concept2}
%   设$V$为复线性空间,若存在二元运算
%   \begin{align*}
%     (-,-): & V\times V \longrightarrow \mathbb{C}\\
%     & \bm{\alpha} \times \bm{\beta} \mapsto (\bm{\alpha}, \bm{\beta}) \text{且满足如下性质:}
%   \end{align*}
%   \begin{align*}
%     & (1) (\bm{\beta},\bm{\alpha})=\overline{(\bm{\alpha},\bm{\beta})};\text{\color{red}共轭对称性}\\
%     &\hspace{-0.3em}\left.\begin{aligned}
%     & (2) (\bm{\alpha}+\bm{\beta},\bm{\gamma})=(\bm{\alpha}+\bm{\gamma})+(\bm{\beta},\bm{\gamma});\\
%     & (3) (c\bm{\alpha},\bm{\beta})=c(\bm{\alpha},\bm{\beta}),c\text{为任一复数};
%     \end{aligned}\right\}\text{\color{red}第一变元的线性}\\
%     & (4) (\bm{\alpha},\bm{\alpha})\geq0\text{且等号成立当且仅当}\bm{\alpha}=0.\text{\color{red}正定性}
%   \end{align*}
%   则称$(-,-)$为$V$上的一个内积,$(\bm{\alpha},\bm{\beta})$称为向量$\bm{\alpha},\bm{\beta}$的内积.
%   给定一个内积结构的复线性空间称为复内积空间.有限维的复内积空间称为酉空间.
% \end{definition}

% \begin{notice}
%   (1)由共轭对称性$\Longrightarrow (\bm{\alpha},\bm{\alpha}) \in \mathbb{R}
%   \Longrightarrow$ (4)的定义有意义;

%   (2)对于实线性空间,第二变元仍线性,
%   对于复线性空间,第二变元为共轭线性,即
%   \[
%     (\bm{\gamma},c\bm{\alpha+d\bm{\beta}})=
%     \overline{c}(\bm{\gamma},\bm{\alpha})+\overline{d}(\bm{\gamma},\bm{\beta}).
%   \]
%   \[
%     \because (\bm{\gamma},c\bm{\alpha+d\bm{\beta}})=
%     \overline{(c\bm{\alpha+d\bm{\beta}}, \bm{\gamma})}=
%     \overline{c(\bm{\alpha},\bm{\gamma})+d(\bm{\beta},\bm{\gamma})}=
%     \overline{c}\overline{(\bm{\alpha},\bm{\gamma})}+\overline{d}\overline{(\bm{\beta},\bm{\gamma})}=
%     \overline{c}(\bm{\gamma},\bm{\alpha})+\overline{d}(\bm{\gamma},\bm{\beta}).
%   \]

%   (3)实内积空间的定义相容于复内积空间的定义,
%   因为实数的共轭等于它本身,实内积空间是复内积空间的特殊情况.
% \end{notice}

% \begin{example}\label{exl:concept1}
%   设 \(\mathbb{R}_n\) 是 \(n\) 维实列向量空间, \(\bm{\alpha} = (x_1, x_2, \cdots, x_n)'\),
%    \(\bm{\beta} = (y_1, y_2, \cdots, y_n)'\),定义
% \[
% (\bm{\alpha}, \bm{\beta}) \triangleq \bm{\alpha}'\cdot\bm{\beta}=x_1 y_1 + x_2 y_2 + \cdots + x_n y_n,
% \]
% 则我们定义了一个内积,这个内积称为 \(\mathbb{R}_n\) 中的标准内积.

% 设 \(\mathbb{R}^n\) 是 \(n\) 维实行向量空间, \(\bm{\alpha} = (x_1, x_2, \cdots, x_n)\),
%    \(\bm{\beta} = (y_1, y_2, \cdots, y_n)\),定义
% \[
% (\bm{\alpha}, \bm{\beta}) \triangleq \bm{\alpha}\cdot\bm{\beta}'=x_1 y_1 + x_2 y_2 + \cdots + x_n y_n,
% \]
% 则我们定义了一个内积,这个内积称为 \(\mathbb{R}^n\) 中的标准内积.
% \end{example}

% \begin{example}\label{exl:concept2}
%   设 \(\mathbb{C}_n\) 是 \(n\) 维复列向量空间, \(\bm{\alpha} = (x_1, x_2, \cdots, x_n)'\), 
%   \(\bm{\beta} = (y_1, y_2, \cdots, y_n)'\),定义
% \[
%   (\bm{\alpha}, \bm{\beta}) \triangleq \bm{\alpha}'\cdot\overline{\bm{\beta}}
%   = x_1\overline{y_1} + x_2\overline{y_2} + \cdots + x_n\overline{y_n},
% \]
% \[ \overline{(\bm{\alpha},\bm{\beta})}=\overline{(\bm{\alpha}'\cdot\overline{\bm{\beta}})}
%   =\overline{\bm{\alpha}}'\cdot\bm{\beta}
%   =(\overline{\bm{\alpha}}'\cdot\bm{\beta})'
%   =\bm{\beta}'\cdot\overline{\bm{\alpha}}
%   =(\bm{\beta},\bm{\alpha}),
% \]
% \[
%   (\bm{\alpha}, \bm{\alpha}) = x_1\overline{x_1} + x_2\overline{x_2} + \cdots + x_n\overline{x_n}
%   = |x_1|^2 + |x_2|^2 + \cdots + |x_n|^2 \geq 0.
% \]
% 则在此定义下 \(\mathbb{C}_n\) 成为一个酉空间,上述内积称为 \(\mathbb{C}_n\) 的标准内积.

% 设 \(\mathbb{C}^n\) 是 \(n\) 维复行向量空间, \(\bm{\alpha} = (x_1, x_2, \cdots, x_n)\), 
%   \(\bm{\beta} = (y_1, y_2, \cdots, y_n)\),定义
% \[
%   (\bm{\alpha}, \bm{\beta}) \triangleq \bm{\alpha}\cdot\overline{\bm{\beta}}'
%   = x_1\overline{y_1} + x_2\overline{y_2} + \cdots + x_n\overline{y_n},
% \]
% 则在此定义下 \(\mathbb{C}^n\) 成为一个酉空间,上述内积称为 \(\mathbb{C}^n\) 的标准内积.
% \end{example}

% \begin{example}\label{exl:concept3}
%   设 $V = \mathbb{R}^2$ 为二维实行向量空间,
%   若 $\bm{\alpha} = (x_1, x_2)$, $\bm{\beta} = (1, 2)$,
%   定义:
%   \[
%     (\bm{\alpha}, \bm{\beta}) = x_1y_1-x_2y_1-x_1y_2 +4x_2y_2,
%   \]
%   容易验证定义\ref{dfn:concept1}中的(1),(2),(3)都成立.
%   当 $\bm{\beta} = \bm{\alpha}$ 时, 上式为 (4)也成立.
%   因此, $\mathbb{R}^2$ 在此内积下成为二维欧氏空间,
%   此内积称为{\heiti 非标准内积}.
% \end{example}

% \begin{example}\label{exl:concept4}
%   设\(V\) 是\(n\)维实列向量空间, \(G\)是\(n\)阶正定实对称阵,
%   对\(\bm{\alpha},\bm{\beta} \in V\),定义$V$上内积:
% \[
% (\bm{\alpha}, \bm{\beta}) \triangleq \bm{\alpha}'G\bm{\beta}.
% \]
% \end{example}

% \begin{proof}
%   \begin{align*}
%     & (1) (\bm{\beta}, \bm{\alpha}) = \bm{\beta}'G\bm{\alpha} =
%     (\bm{\beta}'G\bm{\alpha})'= \bm{\alpha}'G'\bm{\beta} =
%       \bm{\alpha}'G\bm{\beta}=(\bm{\alpha},\bm{\beta}) \text{对称性成立};\\
%     & (2) (\bm{\alpha}, \bm{\beta}) = \bm{\alpha}'G\bm{\beta} \text{根据矩阵性质,第一变元的线性显然成立};\\
%     & (3) (\bm{\alpha},\bm{\alpha})= \bm{\alpha}'G\bm{\alpha} \text{根据正定阵性质,正定性显然成立}.
% \end{align*}
% \end{proof}

% \begin{notice}
%   (1) 当$G=I$时, $V$上内积就是例\ref{exl:concept1}中的标准内积,
%   是例\ref{exl:concept4}的特例;

%   (2) 当$G=\begin{pmatrix}1&-1\\-1&4\end{pmatrix}$时,
%   $V$上内积就是例\ref{exl:concept3}中的非标准内积;

%   (3) 设\(V\) 是\(n\)维实行向量空间, \(G\)是\(n\)阶正定实对称阵,
%   定义$V$上内积:
% \[
% (\bm{\alpha}, \bm{\beta}) \triangleq \bm{\alpha}G\bm{\beta}';
% \]
% \end{notice}

% \begin{example}
%   对$n$维复列向量空间$U$,若有正定Hermite矩阵$H$,
%   则可定义$U$上的内积:
%   \[
%     (\bm{\alpha}, \bm{\beta}) \triangleq \bm{\alpha}'H\overline{\bm{\beta}}.
%     \]
% \end{example}

% \begin{proof}
%   \begin{align*}
%     & (1) \overline{(\bm{\alpha},\bm{\beta})} =
%     \overline{\bm{\alpha}}'\overline{H}\bm{\beta} =
%     (\overline{\bm{\alpha}}'\overline{H}\bm{\beta})' =
%     \bm{\beta}'\overline{H}'\overline{\bm{\alpha}} =
%     \bm{\beta}'H\overline{\bm{\alpha}} =
%     (\bm{\beta}, \bm{\alpha})\text{对称性成立};\\
%     & (2) \bm{\alpha}'H\overline{\bm{\beta}}\text{由矩阵性质可知,第一变元线性成立};\\
%     & (3) (\bm{\alpha},\bm{\alpha})= \bm{\alpha}'H\overline{\bm{\alpha}} \text{由正定Hermite阵性质可知,正定性成立}.
%   \end{align*}
% \end{proof}

% \begin{example}
%   $V=C[a,b]$,是$[a,b]$上连续函数全体构成的实线性空间,
%   $f(x),g(x) \in V$,定义
%   \[ (f(t), g(t)) \triangleq \int_a^bf(t)g(t)\mathrm{d}t. \]
% \end{example}

% \begin{proof}
%   对称性由积分性质可知显然成立;第一变元的线性由积分的线性可得;下面证正定性:
%   \[
%     (f(t),f(t)) = \int_a^bf(t)^2\mathrm{d}t \geq 0
%     \text{等号成立当且仅当}f(t) \equiv 0.
%    \]
% \end{proof}

% \begin{example}
%   $V=\mathbb{R}[x],$假设$n \geq m$,
%   \begin{align*}
%     & f(x) = a_0+a_1x+\cdots+a_nx^n, \\
%     & g(x) = b_0+b_1x+\cdots+b_mx^m.
%   \end{align*}
%   \[(f(x),g(x)) = a_0b_0+a_1b_1+\cdots+a_mb_m. \]
% \end{example}

% \begin{example}
%   $V=M_n(\mathbb{R}), A,B,C \in V$, $c,d$为任一实数:
%   \[ (A,B) \triangleq \Tr(AB'). \]
% \end{example}

% \begin{proof}
%   \begin{asparaenum}[(1)]
%   \item $(B,A) = \Tr(BA')=\Tr((BA')')=\Tr(AB')=(A,B)$;
%   \item $(cA+dB,C) = \Tr((cA+dB)C')=c\Tr(AC')+d\Tr(BC')$;
%   \item $(A,A)=\Tr(AA')=\sum\limits_{i,j=1}^na_{ij}^2 \geq 0 \text{等号成立当且仅当$A=0$,此处假设$A=(a_{ij})$}$;
%   \end{asparaenum}

%   此内积称为Frobenius内积.
% \end{proof}

% \begin{definition}
%   设$V$是一个内积空间,$\bm{\alpha}\in V$,
%   $\bm{\alpha}$的长度(范数)为
%   \[ \Vert\bm{\alpha}\Vert = (\bm{\alpha}, \bm{\alpha})^{\frac{1}{2}}, \]
%   \[ \mathrm{d}(\bm{\alpha},\bm{\beta}) = \Vert\bm{\alpha}-\bm{\alpha}\Vert. \]
% \end{definition}

% \begin{example}
%   $V=\mathbb{R}_n$,标准内积, $\bm{\alpha}=(x_1,x_2,\cdots,x_n)'$,
%   \[\Vert\bm{\alpha}\Vert=\sqrt{x_1^2+x_2^2+\cdots+x_n^2}.\]

%   $V=\mathbb{C}_n$,标准内积, $\bm{\alpha}=(x_1,x_2,\cdots,x_n)'$,
%   \[\Vert\bm{\alpha}\Vert=\sqrt{|x_1|^2+|x_2|^2+\cdots+|x_n|^2}.\]
% \end{example}

% \subparagraph{\color{ecolor}范数的性质:}
% \begin{asparaenum}[(1)]
% \item 非负性: $\Vert\bm{\alpha}\Vert \geq 0\text{且等号成立当且仅当$\bm{\alpha}=0$}$;
% \item 对称性: $\mathrm{d}(\bm{\alpha},\bm{\beta})=\mathrm{d}(\bm{\beta},\bm{\alpha})$.
% \end{asparaenum}

% \begin{theorem}
%   设 \( V \) 是实或复的内积空间,\( \bm{\alpha}, \bm{\beta} \in V \),\( c \) 是任一常数(实数或复数),则

% \begin{asparaenum}[(1)]
%     \item \( \Vert c\bm{\alpha}\Vert = |c|\Vert\bm{\alpha}\Vert \);{\color{ecolor}范数的齐次性}
%     \item \( |(\bm{\alpha}, \bm{\beta})|
%       \leq \Vert\bm{\alpha}\Vert\cdot\Vert\bm{\beta}\Vert \);
%       {\color{ecolor}Cauchy-Schwarz不等式}
%     \item \( \Vert\bm{\alpha} + \bm{\beta}\Vert \leq \Vert\bm{\alpha}\Vert +
%       \Vert\bm{\beta}\Vert \).{\color{ecolor}三角不等式}
% \end{asparaenum}
% \end{theorem}

% \begin{proof}
%   (1) $\Vert c\bm{\alpha}\Vert^2=(c\bm{\alpha},c\bm{\alpha})=
%   c\cdot\overline{c}(\bm{\alpha},\bm{\alpha})=|c|^2\cdot\Vert\bm{\alpha}\Vert^2$,
%   取平方根可得结论.

%   (2)(i) $\bm{\alpha}=\bm{0}: (\bm{\alpha},\bm{\beta})=(\bm{0},\bm{\beta})=0,
%   \Vert\bm{\alpha}\Vert\cdot\Vert\bm{\beta}\Vert=0\Vert{\bm{\beta}}\Vert=0$;

%   (ii) $\bm{\alpha \neq 0:}$令
%   \[ \bm{\gamma}=\bm{\beta}-\frac{(\bm{\beta},
%       \bm{\alpha})}{\Vert\bm{\alpha}\Vert^2}\cdot\bm{\alpha}, \]
%   \[ (\bm{\gamma},\bm{\alpha}) = \left(\bm{\beta}-\frac{(\bm{\beta},
%         \bm{\alpha})}{\Vert\bm{\alpha}\Vert^2}\cdot\bm{\alpha}, \bm{\alpha}\right)
%     = (\bm{\beta},\bm{\alpha})-\frac{(\bm{\beta},
%       \bm{\alpha})}{\Vert\bm{\alpha}\Vert^2}(\bm{\alpha},\bm{\alpha}) = 0.
%   \]
%   计算$\Vert\bm{\gamma}\Vert$:
%   \begin{align*}
%     \Vert\bm{\gamma}\Vert & = (\bm{\gamma},\bm{\gamma})
%                             = (\bm{\gamma}, \bm{\beta}-\frac{(\bm{\beta},
%                             \bm{\alpha})}{\Vert\bm{\alpha}\Vert^2}\cdot\bm{\alpha})\\
%                           & = (\bm{\gamma},\bm{\beta}) -
%                             \frac{\overline{(\bm{\beta},\bm{\alpha})}}{\Vert\bm{\alpha}\Vert^2}(\bm{\gamma},\bm{\alpha})\\
%                           & = (\bm{\gamma},\bm{\beta})
%                             = (\bm{\beta}-\frac{(\bm{\beta},
%                             \bm{\alpha})}{\Vert\bm{\alpha}\Vert^2}\cdot\bm{\alpha},\bm{\beta})
%                             = (\bm{\beta},\bm{\beta})-\frac{(\bm{\beta},
%                             \bm{\alpha})}{\Vert\bm{\alpha}\Vert^2}(\bm{\alpha},\frac{(\bm{\beta},
%                             \bm{\alpha})}{\Vert\bm{\alpha}\Vert^2}(\bm{\alpha},\bm{\beta})\\
%                           & = \Vert\bm{\beta}\Vert^2- \frac{|(\bm{\alpha},\bm{\beta})|^2}{\Vert\bm{\alpha}\Vert^2}\\
%                           & \geq 0\\
%     & \Longrightarrow |(\bm{\alpha},\bm{\beta})| \leq \Vert\bm{\alpha}\Vert\cdot\Vert\bm{\beta}\Vert
%   \end{align*}
%   上式中等号成立当且仅当$\bm{\gamma}=0 \Longleftrightarrow \bm{\alpha}, \bm{\beta}$线性相关.

%   (3)
%   \begin{align*}
%     \Vert\bm{\alpha}+\bm{\beta}\Vert^2 & =
%         (\bm{\alpha}+\bm{\beta},\bm{\alpha}+\bm{\beta})\\
%                                        & = (\bm{\alpha},\bm{\alpha})+(\bm{\beta},\bm{\beta})+
%                                          (\bm{\alpha},\bm{\beta})+(\bm{\beta},\bm{\alpha})\\
%                                        & = \Vert\bm{\alpha}\Vert^2+\Vert\bm{\beta}\Vert^2
%                                          +2\mathrm{Re}(\bm{\alpha},\bm{\beta}).
%   \end{align*}
%   \begin{align*}
%     & \because |\mathrm{Re}(\bm{\alpha},\bm{\beta})| \leq |(\bm{\alpha},\bm{\beta})|
%       \leq \Vert\bm{\alpha}\Vert\cdot\Vert\bm{\beta}\Vert,\\
%     & \therefore \Vert\bm{\alpha}+\bm{\beta}\Vert^2 \leq
%       \Vert\bm{\alpha}\Vert^2+\Vert\bm{\beta}\Vert^2 +
%       2\Vert\bm{\alpha}\Vert\cdot\Vert\bm{\beta}\Vert =
%       (\Vert\bm{\alpha}\Vert+\Vert\bm{\beta}\Vert)^2\\
%     & \therefore \Vert\bm{\alpha} + \bm{\beta}\Vert \leq
%       \Vert\bm{\alpha}\Vert + \Vert\bm{\beta}\Vert.
%   \end{align*}
% \end{proof}

% \begin{example}
%   $V=\mathbb{R}_n$,标准内积,$x_i,y_i \in \mathbb{R}$, 
%   Cauchy不等式:
%   \[ (x_1y_1+x_2y_2+\cdots+x_ny_n)^2 \leq
%     (x_1^2+x_2^2+\cdots+x_n^2)(y_1^2+y_2^2+\cdots+y_n^2).\]
%   $V=C[a,b]$,积分内积, Schwarz不等式:
%   \[
%     \left( \int_{a}^{b} f(t)g(t) \, \mathrm{d}t \right)^2 \leq
%     \left( \int_{a}^{b} f(t)^2 \, \mathrm{d}t \right) \left( \int_{a}^{b} g(t)^2 \, \mathrm{d}t \right).
% \]
% \end{example}

% \begin{definition}
%   设$\bm{0} \neq \bm{\alpha}, \bm{0} \neq \bm{\beta},
%   \bm{\alpha},\bm{\beta} \in V$, $\bm{\alpha},\bm{\beta}$的夹角为:
%   \begin{align*}
%     & \text{实内积}\qquad  \cos\theta =
%       \frac{(\bm{\alpha},\bm{\beta})}{\Vert\bm{\alpha}\Vert\cdot\Vert\bm{\beta}\Vert}
%       \quad \theta\in [0,\pi];\\
%     & \text{复内积}\qquad  \cos\theta =
%       \frac{|(\bm{\alpha},\bm{\beta})|}{\Vert\bm{\alpha}\Vert\cdot\Vert\bm{\beta}\Vert}
%       \quad \theta\in [0,\frac{\pi}{2}].
%   \end{align*}
%   上式中, $|(\bm{\alpha},\bm{\beta})|$不能换成$\mathrm{Re}(\bm{\alpha},\bm{\beta})$,
%   这样将失去几何意义.
% \end{definition}

% \begin{definition}
%   若$(\bm{\alpha},\bm{\beta})=0$,则称$\bm{\alpha}\text{和}\bm{\beta}$正交(垂直),
%   记为$\bm{\alpha}\perp\bm{\beta}$,
%   \[ (\bm{\alpha},\bm{0})=(\bm{0},\bm{\beta})=0 \Longrightarrow \bm{0}\perp \bm{\alpha}, \bm{0} \perp \bm{\beta}. \]
%   \[ \text{若}\bm{\alpha}\neq \bm{0},\bm{\beta}\neq \bm{0}\text{且}
%   \bm{\alpha}\perp\bm{\beta} \Longrightarrow \text{夹角}\theta = \frac{\pi}{2}. \]
% \end{definition}

% \begin{theorem}[勾股定理]
%   若$\bm{\alpha}\perp\bm{\beta}$,则
%   \[ \Vert\bm{\alpha}+\bm{\beta}\Vert^2=\Vert\bm{\alpha}\Vert^2+\Vert\bm{\beta}\Vert^2.  \]
% \end{theorem}

% \begin{proof}
%   \begin{align*}
%     \Vert\bm{\alpha}+\bm{\beta}\Vert^2 & = (\bm{\alpha}+\bm{\beta}, \bm{\alpha}+\bm{\beta})\\
%                                        & = \Vert\bm{\alpha}\Vert^2+\Vert\bm{\beta}\Vert^2 + (\bm{\alpha},\bm{\beta})+(\bm{\beta},\bm{\alpha})\\
%     & \xlongequal{(\bm{\alpha},\bm{\beta})=(\bm{\beta},\bm{\alpha})=0} \Vert\bm{\alpha}\Vert^2+\Vert\bm{\beta}\Vert^2.
%   \end{align*}
% \end{proof}

% \subsection{内积的表示和正交基}

% 设$V$是内积空间,$\{\bm{e}_1, \bm{e}_2, \dots, \bm{e}_n\}$是$V$的一组基.
% 令$g_{ij}=(\bm{e}_i,\bm{e}_j), G=(g_{ij})_{n\times n}$.取
% \[
%   \bm{\alpha} = \sum_{i=1}^n a_i\bm{e}_i \longleftrightarrow \bm{x}= \begin{pmatrix}a_1\\a_2\\\vdots\\a_n\end{pmatrix},
%   \]
% \[
%   \bm{\beta} = \sum_{j=1}^n b_j\bm{e}_j \longleftrightarrow \bm{y}= \begin{pmatrix}b_1\\b_2\\\vdots\\b_n\end{pmatrix}.
% \]

% Case 1: $V$是欧氏空间:
% \begin{align*}
%   (\bm{\alpha},\bm{\beta}) & = 
%   (\sum_{i=1}^n a_i\bm{e}_i,\sum_{j=1}^n b_j\bm{e}_j)\\
%   & = \sum_{i=1}^n\sum_{j=1}^n a_ib_j(\bm{e}_i,\bm{e}_j) = 
%   \sum_{i=1}^n\sum_{j=1}^n a_ib_jg_{ij}\\ 
%   & = \sum_{j=1}^n(\sum_{i=1}^na_ig_{ij})b_j = 
%   (a_1,a_2,\cdots,a_n)G\begin{pmatrix}b_1\\b_2\\\vdots\\b_n\end{pmatrix}\\
%   & = \bm{x}'G\bm{y}.
% \end{align*}
% 上式中, $G$称$V$关于基$\{\bm{e}_1, \bm{e}_2, \dots, \bm{e}_n\}$的内积矩阵
% 或度量矩阵或Gram矩阵.
% \[
% g_{ij} = (\bm{e}_i,\bm{e}_j) = (\bm{e}_j,\bm{e}_i) = g_{ji}
% \Longrightarrow G\text{是实对称矩阵}.
% \]
% \[
% (\bm{\alpha},\bm{\alpha}) = \bm{x}'G\bm{x} > 0, 
% \mathbb{R}^n \ni \forall \bm{x} \neq \bm{0}(\Longleftrightarrow V \ni \forall \bm{\alpha} \neq \bm{0}) 
% \Longrightarrow G\text{是正定阵}.
% \]
% 反之,设$V$是实线性空间,$\{e_1,e_2,\cdots,e_n\}$是给定的一组基, $G$为$n$阶正定实对称阵,
% $\forall \bm{\alpha} \longleftrightarrow \bm{x}, \bm{\beta} \longleftrightarrow \bm{y}$,
% 则$V$上的内积
% \[
% (\bm{\alpha},\bm{\beta})_G \triangleq \bm{\alpha}'G\bm{\beta}
% \Longrightarrow (-,-)_G\text{是$V$上一个内积}.
% \]
% 设$V$是实线性空间,$\{e_1,e_2,\cdots,e_n\}$是给定的一组基,
% \begin{align*}
% \{V\text{上所有内积结构}\} & \xlongrightarrow {\text{一一对应}} \{n\text{阶正定实对称矩阵}\}\\
% \varphi: (-,-) & \longmapsto G=(\bm{e}_i,\bm{e}_j)_{n\times n}\\
% \psi:(\bm{\alpha},\bm{\beta}) \triangleq \bm{\alpha}'G\bm{\beta} & \longmapsfrom G
% \end{align*}

% Case 2: $V$是酉空间:
% \begin{align*}
%   (\bm{\alpha},\bm{\beta}) & = 
%   (\sum_{i=1}^n a_i\bm{e}_i,\sum_{j=1}^n b_j\bm{e}_j)\\
%   & = \sum_{i=1}^n\sum_{j=1}^n a_i\overline{b_j}(\bm{e}_i,\bm{e}_j) = 
%   \sum_{i=1}^n\sum_{j=1}^n a_i\overline{b_j}g_{ij}\\ 
%   & = \sum_{j=1}^n(\sum_{i=1}^na_ig_{ij})\overline{b_j} = 
%   (a_1,a_2,\cdots,a_n)G\begin{pmatrix}\overline{b_1}\\\overline{b_2}\\\vdots\\\overline{b_n}\end{pmatrix}\\
%   & = \bm{x}'G\overline{\bm{y}}.
% \end{align*}
% \[
% \overline{g_{ij}} = \overline{(\bm{e}_i,\bm{e}_j)} = (\bm{e}_j,\bm{e}_i) = g_{ji} \Longrightarrow G' = \overline{G}, \overline{G}'=G 
% \Longrightarrow G\text{为Hermite阵}.
% \]
% \begin{align*}
% (\bm{\alpha},\bm{\alpha}) & = \bm{x}'G\overline{\bm{x}} > 0, 0 \neq \bm{x} \in \mathbb{C}^n\\
% & = \bm{y}'G\overline{\bm{y}} > 0, 0 \neq \bm{y} \in \mathbb{C}^n\\
%  \Longrightarrow G\text{为正定Hermite阵}.
% \end{align*}
% 反之, $G$为$n$阶正定Hermite阵, 则$V$上的内积
% \[
% (\bm{\alpha},\bm{\beta})_G \triangleq \bm{\alpha}'G\overline{\bm{\beta}}.
% \]
% 设$V$是复线性空间,$\{e_1,e_2,\cdots,e_n\}$是给定的一组基,
% \begin{align*}
% \{V\text{上所有内积结构}\} & \xlongrightarrow {\text{一一对应}} \{n\text{阶正定Hermite矩阵}\}\\
% \varphi: (-,-) & \longmapsto G=(\bm{e}_i,\bm{e}_j)_{n\times n}\\
% \psi:(\bm{\alpha},\bm{\beta}) \triangleq \bm{\alpha}'G\overline{\bm{\beta}} & \longmapsfrom G
% \end{align*}

% \begin{question}
% \[
% \exists\text{一组基}\{\bm{e}_1,\bm{e}_2,\cdots,\bm{e}_n\} \st \text{Gram矩阵为}I_n 
% \Longleftrightarrow (\bm{e}_i,\bm{e}_j) = \delta_{ij}, \forall i,j=1,2,\cdots,n. 
% \]
% \end{question}

% \begin{definition}
%   设$V$是$n$维内积空间,$\{\bm{e}_1,\bm{e}_2,\cdots,\bm{e}_n\}$是$V$的一组基,
%   若$(\bm{e}_i,\bm{e}_j) = 0, \forall i\neq j$,则称$\{\bm{e}_1,\bm{e}_2,\cdots,\bm{e}_n\}$是$V$的一组正交基.进一步,若$\Vert\bm{e}_i\Vert=1, \forall i=1,2,\cdots,n$,则称$\{\bm{e}_1,\bm{e}_2,\cdots,\bm{e}_n\}$是$V$的一组标准正交基.
% \end{definition}

% \begin{example}
%   $V=\mathbb{R}_n(\text{或}\mathbb{C}_n)$,标准内积, $\bm{e}_i=(0,0,\cdots,1,\cdots,0)'$,其中$1$在第$i$个位置,其他位置全为0,则$\{\bm{e}_1,\bm{e}_2,\cdots,\bm{e}_n\}$是$V$的一组标准正交基.
% \end{example}

% \begin{example}
% $V=M_n(\mathbb{R})$,Frobenius内积, $E_{ij}$为$n$阶标准基矩阵,其$(i,j)$元为1,其他元素为0,则$\{E_{ij}\}$是$V$的一组标准正交基. 
% \end{example}

% \begin{proof}
%   \begin{align*}
%     (E_{ij},E_{kl}) & = \Tr(E_{ij}E_{kl}') = \Tr(E_{ij}E_{lk})\\
%     & = \delta_{jl}\Tr(E_{ik}) = \delta_{jl}\delta_{ik}\\ 
%     & = \begin{cases}
%       1, & (i,j)=(k,l),\\
%       0, & (i,j)\neq (k,l).
%       \end{cases}
%   \end{align*}
% \end{proof}

% \begin{theory}
%   两两正交的非零向量必线性无关.
% \end{theory}

% \begin{proof}
%   设$\bm{\alpha}_1, \bm{\alpha}_2, \cdots, \bm{\alpha}_m$是
%   两两正交的非零向量且
%   \[
%   c_1\bm{\alpha}_1+c_2\bm{\alpha}_2+\cdots+c_m\bm{\alpha}_m=0.
%   \]
%   \begin{align*}
%     & 0 = (\sum_1^m c_i\bm{\alpha}_i,\bm{\alpha}_j) 
%     = \sum_1^m c_i(\bm{\alpha}_i, \bm{\alpha}_j) 
%     = c_j(\bm{\alpha}_j, \bm{\alpha}_j)\\
%     & \Longrightarrow c_j =0 \forall 1 \leq j \leq m
%   \end{align*}
% \end{proof}

% \begin{deduction}
%   $n$维内积空间中两两正交的非零向量至多只有$n$个.
% \end{deduction}

% \begin{theory}
%   设$\bm{\beta}$与$\bm{\alpha}_1,\bm{\alpha}_2,\cdots,\bm{\alpha}_m$正交,
%   则$\bm{\beta}$与$\bm{\alpha}_1,\bm{\alpha}_2,\cdots,\bm{\alpha}_m$的线性组合
%  也正交,即
%   $\forall \bm{\gamma}\in L(\bm{\alpha}_1,\bm{\alpha}_2,\cdots,\bm{\alpha}_m)$都有
%   $(\bm{\beta},\bm{\gamma})=0$.
% \end{theory}

% \begin{proof}
%   设$\bm{\gamma} = c_1\bm{\alpha}_1+c_2\bm{\alpha}_2+\cdots+c_m\bm{\alpha}_m$,
%   \begin{align*}
%     (\bm{\beta},\bm{\gamma}) & = (\bm{\beta},c_1\bm{\alpha}_1+c_2\bm{\alpha}_2+\cdots+c_m\bm{\alpha}_m)\\
%     & = c_1(\bm{\beta},\bm{\alpha}_1)+c_2(\bm{\beta},\bm{\alpha}_2)+\cdots+c_m(\bm{\beta},\bm{\alpha}_m)\\
%     & = 0.
%   \end{align*}
% \end{proof}

% \begin{theorem}[Gram-Schmidt正交化方法]
%   设$\bm{u}_1,\bm{u}_2,\cdots,\bm{u}_m$是$V$中线性无关的向量,
%   则存在一组两两正交的非零向量$\bm{v}_1,\bm{v}_2,\cdots,\bm{v}_m$,
%   使得$L(\bm{v}_1,\bm{v}_2,\cdots,\bm{v}_m)=L(\bm{u}_1,\bm{u}_2,\cdots,\bm{u}_m)$.
% \end{theorem}

% \begin{proof}
%   对$m$进行归纳:
%   \begin{asparaenum}[(1)]
%   \item 当$m=1$时, $\bm{v}_1=\bm{u}_1$;
%   \item 假设对$m=k$成立, 即存在一组两两正交的非零向量$\bm{v}_1,\bm{v}_2,\cdots,\bm{v}_k$,
%     使得$L(\bm{v}_1,\bm{v}_2,\cdots,\\
%     \bm{v}_k)=L(\bm{u}_1,\bm{u}_2,\cdots,\bm{u}_k)$;
%   \item 当$m=k+1$时, 先从三维情况获得思路和启发.假设$V$为三维实内积空间, 如下图所示:
%     \begin{figure}[htbp]
%       \centering
%       \tdplotsetmaincoords{75}{60} % 视角参数 (theta=60, phi=270)
      
%       \begin{tikzpicture}[scale=1.2, tdplot_main_coords]
%         % 坐标轴
%         \draw[->, thick] (0,0,0) -- (3,0,0) node[right]{$\bm{v}_1$};
%         \draw[->, thick] (0,0,0) -- (0,3,0) node[above]{$\bm{v}_2$};
%         \draw[->, thick] (0,0,0) -- (0,0,3) node[below left]{$\bm{v}_3$};
    
%         % 向量 u
%         \coordinate (u) at (2,2,2);
%         \draw[->, thick, blue] (0,0,0) -- (u) node[above right]{$\bm{u}$};
    
%         % 投影线
%         \draw[dashed, gray] (u) -- (2,0,0);
%         \draw[dashed, gray] (u) -- (0,2,0);
%         \draw[dashed, gray] (u) -- (0,0,2);
    
%         % 投影向量
%         \draw[->, thick, red] (0,0,0) -- (2,0,0) node[below]{$a{\bm{v}_1}$};
%         \draw[->, thick, red] (0,0,0) -- (0,2,0) node[left]{$b{\bm{v}_2}$};
%         \draw[->, thick, red] (0,0,0) -- (0,0,2) node[below]{${\bm{v}_3}$};
%       \end{tikzpicture}
    
%       \caption{向量 $\mathbf{u}$ 在正交坐标系 $v_1,v_2,v_3$ 上的投影}
%       \label{fig:projection}
%     \end{figure}
%     设$\bm{u}_3=a\bm{v}_1+b\bm{v}_2+\bm{v}_3$,则
%     \begin{align*}
%     & (\bm{u}_3,\bm{v}_1)=(a\bm{v}_1+b\bm{v}_2+\bm{v}_3,\bm{v}_1)=a(\bm{v}_1,\bm{v}_1)
%     =a\Vert\bm{v}_1\Vert^2\\
%     & \Longrightarrow a=\frac{(\bm{u}_3,\bm{v}_1)}{\Vert\bm{v}_1\Vert^2},
%     b=\frac{(\bm{u}_3,\bm{v}_2)}{\Vert\bm{v}_2\Vert^2}\\
%     & \Longrightarrow \bm{v}_3=\bm{u}_3-\frac{(\bm{u}_3,\bm{v}_1)}{\Vert\bm{v}_1\Vert^2}\bm{v}_1
%     -\frac{(\bm{u}_3,\bm{v}_2)}{\Vert\bm{v}_2\Vert^2}\bm{v}_2.
%     \end{align*}
%     因此,令
%     \begin{equation}\label{eq:gram-schmidt}
%     \boxed{\bm{v}_{k+1} = \bm{u}_{k+1}-\sum_{i=1}^k\frac{(\bm{u}_{k+1},\bm{v}_i)}{(\bm{v}_i,\bm{v}_i)}\bm{v}_i.}
%     \end{equation}
%     先证$\bm{v}_{k+1}\neq 0$.用反正法,若$v_{k+1}=0$,则
%     \[ \bm{u}_{k+1} \in L(\bm{v}_1,\bm{v}_2,\cdots,\bm{v}_k) = L(\bm{u}_1,\bm{u}_2,\cdots,\bm{u}_k)\]
%     这与$\bm{u}_1,\bm{u}_2,\cdots,\bm{u}_{k+1}$线性无关矛盾.再证$\bm{v}_{k+1}$与$\bm{v}_1,\bm{v}_2,\cdots,\bm{v}_k$正交:
%     \begin{align*}
%       \forall 1 \leq j \leq k, & (\bm{v}_{k+1},\bm{v}_j) = (\bm{u}_{k+1}-\sum_{i=1}^k\frac{(\bm{u}_{k+1},\bm{v}_i)}{(\bm{v}_i,\bm{v}_i)}\bm{v}_i,\bm{v}_j)\\
%       & = (\bm{u}_{k+1},\bm{v}_j)-\sum_{i=1}^k\frac{(\bm{u}_{k+1},\bm{v}_i)}{(\bm{v}_i,\bm{v}_i)}(\bm{v}_i,\bm{v}_j)\\
%       & = (\bm{u}_{k+1},\bm{v}_j)-\frac{(\bm{u}_{k+1},\bm{v}_j)}{(\bm{v}_j,\bm{v}_j)}(\bm{v}_j,\bm{v}_j) = 0.
%     \end{align*}
%     从\eqref{eq:gram-schmidt}式可知, $\bm{v}_{k+1} \in L(\bm{u}_1,\bm{u}_2,\cdots,\bm{u}_{k+1})$,
%     \begin{align*}
%       L(\bm{u}_1,\bm{u}_2,\cdots,\bm{u}_{k+1}) & = 
%       L(\bm{u}_1,\bm{u}_2,\cdots,\bm{u}_k)+L(\bm{u}_{k+1})\\
%       & = L(\bm{v}_1,\bm{v}_2,\cdots,\bm{v}_k)+L(\bm{u}_{k+1}) 
%       \subseteq L(\bm{v}_1,\bm{v}_2,\cdots,\bm{v}_{k+1}).
%     \end{align*}
%     \begin{align*}
%      L(\bm{v}_1,\bm{v}_2,\cdots,\bm{v}_{k+1}) & =
%      L(\bm{v}_1,\bm{v}_2,\cdots,\bm{v}_k)+L(\bm{v}_{k+1})\\
%      & \subseteq L(\bm{v}_1,\bm{v}_2,\cdots,\bm{v}_k)+L(\bm{u}_{k+1})\\
%       & = L(\bm{u}_1,\bm{u}_2,\cdots,\bm{u}_k)+L(\bm{u}_{k+1})\\
%       & = L(\bm{u}_1,\bm{u}_2,\cdots,\bm{u}_{k+1}).
%     \end{align*}  
% \end{asparaenum}
% \end{proof}

% \begin{notice}
%   从内积空间的一组基出发,经过Gram-Schmidt正交化方法,可得到一组正交基,即
%   \[
%   \bm{u}=L(\bm{u}_1,\bm{u}_2,\cdots,\bm{u}_n) = 
%   L(\bm{v}_1,\bm{v}_2,\cdots,\bm{v}_n).
%   \]
%   设过度矩阵:
%   \[
%   (\bm{u}_1,\bm{u}_2,\cdots,\bm{u}_n) = (\bm{v}_1,\bm{v}_2,\cdots,\bm{v}_n)A.
%   \]
%   \begin{align*}
%     & \bm{u}_1 = \bm{v}_1,\\
%     & \bm{u}_2 = \bm{v}_2-\frac{(\bm{v}_2,\bm{v}_1)}{(\bm{v}_1,\bm{v}_1)}\bm{v}_1,\\
%     & \bm{u}_3 = \bm{v}_3-\frac{(\bm{v}_3,\bm{v}_1)}{(\bm{v}_1,\bm{v}_1)}\bm{v}_1
%     -\frac{(\bm{v}_3,\bm{v}_2)}{(\bm{v}_2,\bm{v}_2)}\bm{v}_2,\\
%     & \cdots\\
%     & \bm{u}_{k+1} = \bm{v}_{k+1}-\sum_{i=1}^k\frac{(\bm{u}_{k+1},\bm{v}_i)}{(\bm{v}_i,\bm{v}_i)}\bm{v}_i\\
%     \Longrightarrow A = & \begin{pmatrix}
%       1 & * & * & \cdots & *\\
%       0 & 1 & * & \cdots & *\\
%       0 & 0 & 1 & \cdots & *\\
%       \vdots & \vdots & \vdots & \ddots & \vdots\\
%       0 & 0 & 0 & \cdots & 1
%     \end{pmatrix}.
%   \end{align*}
%   单位化(标准化):
%   \[
%   \bm{v} \neq 0 \longrightarrow \bm{e} = \frac{\bm{v}}{\Vert\bm{v}\Vert} 
%   \longleftarrow \left\Vert\frac{\bm{v}}{\Vert\bm{v}\Vert}\right\Vert = 1.
%   \]
%   令
%   \[\bm{w}_i = \frac{\bm{v}_i}{\Vert\bm{v}_i\Vert}, i=1,2,\cdots,n,\]
%   则$\{\bm{w}_1,\bm{w}_2,\cdots,\bm{w}_n\}$是$V$的一组标准正交基.
%   \begin{align*}
%     & (\bm{v_1},\bm{v_2},\cdots,\bm{v_n}) = (\bm{w_1},\bm{w_2},\cdots,\bm{w_n})B\\
%     = & (\bm{w_1},\bm{w_2},\cdots,\bm{w_n})\begin{pmatrix}
%       \Vert\bm{v}_1\Vert & 0 & 0 & \cdots & 0\\
%       0 & \Vert\bm{v}_2\Vert & 0 & \cdots & 0\\
%       0 & 0 & \Vert\bm{v}_3\Vert & \cdots & 0\\
%       \vdots & \vdots & \vdots & \ddots & \vdots\\
%       0 & 0 & 0 & \cdots & \Vert\bm{v}_n\Vert
%     \end{pmatrix}\\
%     \Longrightarrow & (\bm{u}_1,\bm{u}_2,\cdots,\bm{u}_n) = (\bm{w}_1,\bm{w}_2,\cdots,\bm{w}_n)\cdot C.
%   \end{align*}
%   其中,
%   \[
%   C = B\cdot A = \begin{pmatrix}
%     \Vert\bm{v}_1\Vert & * & * & \cdots & *\\
%     0 & \Vert\bm{v}_2\Vert & * & \cdots & *\\
%     0 & 0 & \Vert\bm{v}_3\Vert & \cdots & *\\
%     \vdots & \vdots & \vdots & \ddots & \vdots\\
%     0 & 0 & 0 & \cdots & \Vert\bm{v}_n\Vert
%   \end{pmatrix}.
%   \]
% \end{notice}

% \begin{theorem}
%   任一有限维内积空间必存在标准正交基.
% \end{theorem}

% \begin{proof}
%   \begin{align*}
%     & \text{任取内积空间$V$的一组基}\{\bm{u}_1,\bm{u}_2,\cdots,\bm{u}_n\}\\ 
%     & \xlongrightarrow{\text{Gram-Schmidt正交化方法}} \text{$V$的一组正交基}\{\bm{v}_1,\bm{v}_2,\cdots,\bm{v}_n\}\\
%     & \xlongrightarrow{\text{标准化(单位化)}} \text{$V$的一组标准正交基}\{\bm{w}_1,\bm{w}_2,\cdots,\bm{w}_n\}.
%   \end{align*}
% \end{proof}

% \begin{example}
%   $V=\mathbb{R}^3$,标准内积, $\bm{u}_1=(3,0,4), \bm{u}_2=(-1,0,7), \bm{u}_3=(2,9,11)$,
%   用Gram-Schmidt正交化方法求$V$的一组标准正交基.
% \end{example}

% \begin{solution}
%   \begin{align*}
%     & \bm{v}_1 = \bm{u}_1 = (3,0,4),\\
%     & \bm{v}_2 = \bm{u}_2-\frac{(\bm{u}_2,\bm{v}_1)}{(\bm{v}_1,\bm{v}_1)}\bm{v}_1
%     = (-1,0,7)-\frac{(-1,0,7)\cdot(3,0,4)}{(3,0,4)\cdot(3,0,4)}\cdot(3,0,4)\\
%     & = (-1,0,7)-\frac{25}{25}\cdot(3,0,4) = (-1,0,7)-(3,0,4) = (-4,0,3),\\
%     & \bm{v}_3 = \bm{u}_3-\frac{(\bm{u}_3,\bm{v}_1)}{(\bm{v}_1,\bm{v}_1)}\bm{v}_1
%     -\frac{(\bm{u}_3,\bm{v}_2)}{(\bm{v}_2,\bm{v}_2)}\bm{v}_2\\
%     & = (2,9,11)-\frac{(2,9,11)\cdot(3,0,4)}{(3,0,4)\cdot(3,0,4)}\cdot(3,0,4)
%     -\frac{(2,9,11)\cdot(-4,0,3)}{(-4,0,3)\cdot(-4,0,3)}\cdot(-4,0,3)\\
%     & = (2,9,11)-\frac{50}{25}\cdot(3,0,4)-\frac{25}{25}\cdot(-4,0,3)
%     = (2,9,11)-2\cdot(3,0,4)-(-4,0,3) = (0,9,0),\\
%     & \bm{w}_1 = \frac{\bm{v}_1}{\Vert\bm{v}_1\Vert} = \frac{1}{5}(3,0,4),\\
%     & \bm{w}_2 = \frac{\bm{v}_2}{\Vert\bm{v}_2\Vert} = \frac{1}{5}(-4,0,3),\\
%     & \bm{w}_3 = \frac{\bm{v}_3}{\Vert\bm{v}_3\Vert} = (0,1,0).
%   \end{align*}
%   \end{solution}

% \begin{theory}\label{thr:gram-matrix}
%   设$V$是$n$维内积空间,$\{\bm{e}_1,\bm{e}_2,\cdots,\bm{e}_n\}, 
%   \{\bm{f}_1,\bm{f}_2,\cdots,\bm{f}_n\}$是$V$的两组基,
%   过度矩阵为$C$,即
%   \[
%   (\bm{f}_1,\bm{f}_2,\cdots,\bm{f}_n) = (\bm{e}_1,\bm{e}_2,\cdots,\bm{e}_n)C,
%   \]
%   则
%   \begin{align*}
%     \text{$V$为实内积空间时:}\quad G_{\bm{f}} & = C'G_{\bm{e}}C,\\
%     \text{$V$为复内积空间时:}\quad G_{\bm{f}} & = C'G_{\bm{e}}\overline{C}.
%   \end{align*}
% \end{theory}

% \begin{proof}
%   设
%   \[
%   C=(c_{ij})_{n\times n}, G=(g_{ij})_{n\times n},
%   \]
%   \begin{align*}
%     & \because (\bm{f}_1,\bm{f}_2,\cdots,\bm{f}_n) = (\bm{e}_1,\bm{e}_2,\cdots,\bm{e}_n)C,\\
%     & \therefore \bm{f}_k = \sum_{i=1}^n c_{ik}\bm{e}_i, 
%     \bm{f}_l = \sum_{j=1}^n c_{jl}\bm{e}_j, k,l=1,2,\cdots,n.
%   \end{align*}
%   \begin{align*}
%     (\bm{f}_k,\bm{f}_l) & = (\sum_{i=1}^n c_{ik}\bm{e}_i,\sum_{j=1}^n c_{jl}\bm{e}_j)\\
%     & = \sum_{i=1}^n\sum_{j=1}^n c_{ik}c_{jl}(\bm{e}_i,\bm{e}_j) = \sum_{i=1}^n\sum_{j=1}^n c_{ik}c_{jl}g_{ij}\\
%     & = \sum_{j=1}^n(\sum_{i=1}^n g_{ij}c_{ik})c_{jl}.
%   \end{align*}
% \end{proof}

% \begin{notice}
%   在一组基$\{\bm{e}_1,\bm{e}_2,\cdots,\bm{e}_n\}$下,
%   对于任一正定实对称矩阵$G_{\bm{e}}$,
%   $\exists \text{非异阵}C \st C'G_{\bm{e}}C = I_n$.
%   令
%   \[
%   (\bm{f}_1,\bm{f}_2,\cdots,\bm{f}_n) = (\bm{e}_1,\bm{e}_2,\cdots,\bm{e}_n)C,
%   \]
%   由引理\ref{thr:gram-matrix}可知,
%   \[
%   G_{\bm{f}} = C'G_{\bm{e}}C = I_n.
%   \]
%   即$\{\bm{f}_1,\bm{f}_2,\cdots,\bm{f}_n\}$是$V$的一组标准正交基.
% \end{notice}

% \begin{definition}
% 设$U$是内积空间$V$的子空间,
% \[
% U^{\perp} = \{\bm{v}\in V \vert (\bm{v},\bm{u})=0, \forall \bm{u}\in U\}
% \]
% 易证$U^{\perp}$是$V$的子空间,称为$U$的正交补空间.
% \end{definition}

% \begin{theorem}
%   设$V$是$n$维内积空间,$U$是$V$的子空间,
%   \begin{asparaenum}[(1)]
%   \item $ V = U \oplus U^{\perp}$;
%   \item $U$的标准正交基可扩充为$V$的标准正交基.
%   \end{asparaenum}
% \end{theorem}

% \begin{proof}
%   (1)设$\dim U=m$,将$V$上内积限制在$U$上,则$U$是$m$维内积空间,
%   由定理\ref{thr:gram-matrix}可知,$U$有一组标准正交基$\{\bm{e}_1,\bm{e}_2,\cdots,\bm{e}_m\}$,
%   令
%   \[
%   w=v-\sum_{i=1}^m(\bm{v},\bm{e}_i)\bm{e}_i, \forall \bm{v}\in V,
%   \]
%   则
%   \begin{align*}
%     (\bm{w},\bm{e}_j) & = (\bm{v}-\sum_{i=1}^m(\bm{v},\bm{e}_i)\bm{e}_i,\bm{e}_j)\\
%     & = (\bm{v},\bm{e}_j)-\sum_{i=1}^m(\bm{v},\bm{e}_i)(\bm{e}_i,\bm{e}_j)\\
%     & = 0, j=1,2,\cdots,m.
%   \end{align*}
%   从而$(\bm{w},U)=0$,即$\bm{w}\in U^{\perp}$,
%   于是
%   \[
%   U+U^{\perp}\ni \bm{v}=\sum_{i=1}^m(\bm{v},\bm{e}_i)\bm{e}_i+\bm{w}.
%   \] 
%   任取$\bm{v}\in U\cap U^{\perp}$,则$(\bm{v},\bm{v})=0$,即$\bm{v}=0$,
%   从而$U\cap U^{\perp}=\{\bm{0}\}$,即$V=U\oplus U^{\perp}$.

%   (2)取$U^{\perp}$的一组标准正交基$\{\bm{e}_{m+1},\cdots,\bm{e}_n\}$,
%   则$\{\bm{e}_1,\bm{e}_2,\cdots,\bm{e}_m,\bm{e}_{m+1},\cdots,\bm{e}_n\}$是$V$的一组标准正交基.
% \end{proof}

% \begin{definition}
%   设$V$是$n$维内积空间,$V_1,\cdots,V_m$是$V$的子空间,
%   若对$\forall \bm{\alpha}\in V_i,\forall \bm{\beta}\in V_j$,有$(\bm{\alpha},\bm{\beta})=0$, 
%   则称$V_1$与$V_2$正交.若$V=V_1+V_2+\cdots+V_m$且
%   $V_i$两两正交,则称上述和为正交和,
%   记为$V=V_1\perp V_2\perp\cdots\perp V_m$.
%   例如$V=U+U^{\perp}$即为正交和.
% \end{definition}

% \begin{theory}
%   正交和都是直和,称为正交直和.
% \end{theory}

% \begin{proof}
%   先证$V_i$与$\sum\limits_{j\neq i}V_j$正交.
%   任取$\bm{v_i} \in V_i$有
%   \[
%   (\bm{v_i},\sum_{j\neq i}\bm{v_j}) = \sum_{j\neq i}(\bm{v_i},\bm{v_j}) = 0.
%   \]
%   任取$\bm{\alpha}\in V_i\cap\sum\limits_{j\neq i}V_j$,
%   则$(\bm{\alpha},\bm{\alpha})=0$,由内积的正定性可知$\bm{\alpha}=\bm{0}$,
%   从而
%   \[
%   V=V_1\oplus V_2\oplus\cdots\oplus V_m.
%   \]
% \end{proof}

% \begin{definition}
% $V=V_1\oplus V_2\oplus\cdots\oplus V_m$,对任一$\bm{v}\in V$,
% $\bm{v}=\bm{v}_1+\bm{v}_2+\cdots+\bm{v}_m, (\bm{v}_i\in V_i)$的分解唯一.
% 构造$E_i \in \mathscr{L}(V), E_i(\bm{v})=\bm{v}_i, 1 \leq i \leq m$,
% 称$E_i$为$V$的{\heiti 投影变换}.
% \end{definition}

% \subparagraph{\color{ecolor}投影变换的性质}
% \begin{asparaenum}[(1)]

% \item $E_i^2=E_i$;

% \item $E_iE_j=0, i\neq j$;

% \item $E_1+E_2+\cdots+E_m=I_V$;
% \end{asparaenum}

% 进一步,设$V=V_1\perp V_2\perp\cdots\perp V_m$为正交直和,
% 则投影变换$E_i$称为从$V$到$V_i$的{\heiti 正交投影变换}.

% \begin{theory}
% 设$V$为$n$维内积空间, $V=U\perp U^{\perp}$, 
% 记$E$为从$V$到$U$的正交投影变换,则$\forall \bm{\alpha}, \bm{\beta} \in V$,
% \[
% (E(\bm{\alpha}),\bm{\beta}) = (\bm{\alpha},E(\bm{\beta})).
% \]
% \end{theory}

% \begin{proof}
% 设$\bm{\alpha}=\bm{u}_1+\bm{w}_1, \bm{\beta}=\bm{u}_2+\bm{w}_2, 
% \bm{u}_1,\bm{u}_2\in U, \bm{w}_1,\bm{w}_2\in U^{\perp}$,则
% $E(\bm{\alpha})=\bm{u}_1, E(\bm{\beta})=\bm{u}_2$,
% \begin{align*}
%   (E(\bm{\alpha}),\bm{\beta}) & = (\bm{u}_1,\bm{u}_2+\bm{w}_2) = (\bm{u}_1,\bm{u}_2)\\
%   (\bm{\alpha},E(\bm{\beta})) & = (\bm{u}_1+\bm{w}_1,\bm{u}_2) = (\bm{u}_1,\bm{u}_2).
% \end{align*}
% \end{proof}

% \begin{theorem}[Bessel不等式]
%   设$V$是内积空间,$\bm{\bm{v}_1},\bm{v}_2,\cdots,\bm{v}_m$是$V$中
%   两两正交的非零向量组,则$\forall \bm{y}\in V$,有
%   \[
%   \sum_{i=1}^m\frac{|(\bm{y},\bm{v}_i)|^2}{\Vert\bm{v}_i\Vert^2} \leq \Vert\bm{y}\Vert^2. 
%   \]
%   等号成立当且仅当$\bm{y}$可由$\{\bm{v}_1,\bm{v}_2,\cdots,\bm{v}_m\}$线性表示.
% \end{theorem}

% \begin{proof}
% $\because \{\bm{v}_1,\bm{v}_2,\cdots,\bm{v}_m \}$两两正交,
% $\bm{y}\in V$,设
% \[
% \bm{z}=\bm{y}-\sum_{i=1}^m\frac{(\bm{y},\bm{v}_i)}{\Vert\bm{v}_i\Vert^2}\bm{v}_i,
% \]
% 再令$\bm{x}=\sum\limits_{i=1}^m\frac{(\bm{y},\bm{v}_i)}{\Vert\bm{v}_i\Vert^2}\bm{v}_i$,则
% \begin{align*}
% & (\bm{z},\bm{v}_i) = (\bm{y},\bm{v}_i)-\sum_{j=1}^m\frac{(\bm{y},\bm{v}_j)}{\Vert\bm{v}_j\Vert^2}(\bm{v}_j,\bm{v}_i) = 0\\
% \Longrightarrow & (\bm{z},\bm{x}) = \sum_{i=1}^m\frac{(\bm{y},\bm{v}_i)}{\Vert\bm{v}_i\Vert^2}(\bm{z},\bm{v}_i) = 0\\
% \Longrightarrow & \Vert\bm{y}\Vert^2  \xlongequal{\bm{y}=\bm{x}+\bm{z}} \Vert\bm{z}+\bm{x}\Vert^2 
% \xlongequal[\text{勾股定理}]{(\bm{z},\bm{x}) = 0} \Vert\bm{z}\Vert^2+\Vert\bm{x}\Vert^2 
% \geq \Vert\bm{x}\Vert^2 \xlongequal [\text{勾股定理}]{} \sum_{i=1}^m\frac{|(\bm{y},\bm{v}_i)|^2}{\Vert\bm{v}_i\Vert^2}\\
% \Longrightarrow & \sum_{i=1}^m\frac{|(\bm{y},\bm{v}_i)|^2}{\Vert\bm{v}_i\Vert^2} \leq \Vert\bm{y}\Vert^2.
% \end{align*}
% 上式中等号成立当且仅当$\bm{z}=\bm{0}$,即$\bm{y}=\bm{x}$,
% 即$\bm{y}$可由$\{\bm{v}_1,\bm{v}_2,\cdots,\bm{v}_m\}$线性表示.
% \end{proof}

% \begin{notice}
% 上面定理对于无限维内积空间也成立:
% $\{ \bm{v}_1,\bm{v}_2,\cdots,\bm{v}_m,\cdots\}$为一列两两正交的非零向量组,则
% \[
%   \sum_{i=1}^{\infty}\frac{|(\bm{y},\bm{v}_i)|^2}{\Vert\bm{v}_i\Vert^2} \leq \Vert\bm{y}\Vert^2. 
% \]
% \end{notice}

% \subsection{伴随}

% 线性变换又称为线性算子, operator.

% 设$V$是$n$维内积空间,$\mathscr{L}(V)$是$V$上的线性变换全体,
% $\{ \bm{e}_1,\cdots,\bm{e}_n \}$是$V$的一组标准正交基,
% $\varphi\in \mathscr{L}(V)$是线性算子.任取$\bm{\alpha},\bm{\beta}\in V$,
% $\bm{x},\bm{y}\in \mathbb{R}^n$,分别是$\bm{\alpha},\bm{\beta}$在$\{ \bm{e}_1,\cdots,\bm{e}_n \}$下的坐标向量,则
% \begin{align*}
%   \bm{\alpha}  = \sum_{i=1}^n a_i\bm{e}_i & \longleftrightarrow \bm{x} = \begin{pmatrix}
%     a_1\\
%     a_2\\
%     \vdots\\
%     a_n
%   \end{pmatrix},\\
%   \bm{\beta}  = \sum_{i=1}^n b_i\bm{e}_i & \longleftrightarrow \bm{y} = \begin{pmatrix}
%     b_1\\
%     b_2\\
%     \vdots\\
%     b_n
%   \end{pmatrix},\\
%   \varphi & \xlongleftrightarrow {\text{表示矩阵}} A = (a_{ij})_{n\times n}\in \mathbb{C},\\
%   \varphi(\bm{\alpha}) & \longleftrightarrow A\bm{x},\\
%   (\varphi(\bm{\alpha}),\bm{\beta}) & = (A\bm{x})'\overline{\bm{y}} =\bm{x}'A'\overline{\bm{y}} = \bm{x}'\overline{(\overline{A}'\bm{y})}.
% \end{align*}
% \begin{align*}
% \text{定义}\psi\in \mathscr{L}(V) & \xlongleftrightarrow {\text{表示矩阵}} \overline{A}'\\
% \text{即} \psi(\bm{\beta}) & \longleftrightarrow \overline{A}'\bm{y}.
% \end{align*}
% \begin{align*}
% & (\bm{\alpha},\psi(\bm{\beta})) = \bm{x}'\overline{(\overline{A}'\bm{y})}\\
% \Longrightarrow &  (\varphi(\bm{\alpha},\bm{\beta})) = (\bm{\alpha},\psi(\bm{\beta})), \forall \bm{\alpha},\bm{\beta}\in V
% \end{align*}
% 特别的,当$V$为欧氏空间时: 
% $\psi \in \mathscr{L}(V) \xlongleftrightarrow {\text{表示矩阵}} A'$.

% \begin{definition}
% 设$\varphi$是内积空间$V$上的线性算子,
% 若存在线性算子$\varphi^*$,使得$\forall \bm{\alpha},\bm{\beta} \in V$
% \[
% (\varphi(\bm{\alpha}),\bm{\beta}) = (\bm{\alpha},\varphi^*(\bm{\beta})),
% \]
% 则称$\varphi^*$是$\varphi$的{\heiti 伴随算子},或简称{\heiti 伴随}.
% \end{definition}

% \begin{theorem}
% (1) 伴随算子若存在,必唯一;

% (2) 有限维内积空间上任一线性算子,必有伴随算子.
% \end{theorem}

% \begin{proof}
% (1) 设$\varphi^*$和$\varphi^{\#}$都是$\varphi$的伴随算子,即
% \begin{align*}
% & (\varphi(\bm{\alpha}),\bm{\beta}) = (\bm{\alpha},\varphi^*(\bm{\beta})) = (\bm{\alpha},\varphi^{\#}(\bm{\beta}))\\
% \Longrightarrow & (\bm{\alpha},\varphi^*(\bm{\beta})-\varphi^{\#}(\bm{\beta})) = 0, \forall \bm{\alpha} \in V
% \end{align*}
% 取$\bm{\alpha}=\varphi^*(\bm{\beta})-\varphi^{\#}(\bm{\beta})$,则
% \[
% (\varphi^*(\bm{\beta})-\varphi^{\#}(\bm{\beta}),\varphi^*(\bm{\beta})-\varphi^{\#}(\bm{\beta})) = 0,
% \]
% 由正定性可知$\varphi^*(\bm{\beta})=\varphi^{\#}(\bm{\beta})$,从而
% \[
% \varphi^* = \varphi^{\#}.
% \]

% (2) 由伴随的唯一性及本节开始部分的构造可知,
% \[
% \psi = \varphi^*,
% \]
% 所以,有限维内积空间上任一线性算子必有伴随算子.
% \begin{align*}
%   & \varphi \in \mathscr{L}(V) \xlongleftrightarrow {\text{表示矩阵}} A,\\
%   \Longrightarrow & \begin{cases}
%      \text{酉空间}\quad & \varphi^* \xlongleftrightarrow {\text{表示矩阵}} \overline{A}',\\
%      \text{欧氏空间}\quad & \varphi^* \xlongleftrightarrow {\text{表示矩阵}} A'.
%   \end{cases}
% \end{align*}
% \end{proof}

% \begin{theorem}
%   设$V$为$n$维内积空间,$\{\bm{e}_1,\bm{e}_2,\cdots,\bm{e}_n\}
%   是V$的一组标准正交基,$\varphi\in \mathscr{L}(V)$的表示矩阵为$A$,
%   则$\varphi$的伴随算子$\varphi^*$的表示矩阵为:
%   \[
%   \begin{cases}
%     \text{酉空间}\quad & \overline{A}',\\
%     \text{欧氏空间}\quad & A'.
%   \end{cases}
%   \]
% \end{theorem}

% \begin{theorem}\label{thm:accompany}
% 设$V$为内积空间,$\varphi,\psi \in \mathscr{L}$且
% $\varphi^*,\psi^*$存在, $c$为常数,则
% \begin{asparaenum}[(1)]
% \item $(\varphi+\psi)^* = \varphi^*+\psi^*$;
% \item $(c\varphi)^* = \overline{c}\varphi^*$;
% \item $(\varphi\psi)^* = \psi^*\varphi^*$;
% \item $(\varphi^*)^* = \varphi$;
% \item 若$\varphi$可逆,则$\varphi^*$也可逆,且$(\varphi^*)^{-1} = (\varphi^{-1})^*$.
% \end{asparaenum}
% \end{theorem}

% \begin{proof}
% 若$V$是有限维内积空间,设$\varphi$的表示矩阵为$A$, $\psi$的表示矩阵为$B$.
% $\{\bm{e}_1,\bm{e}_2,cdots,\bm{e}_n\}$是
% $V$的一组标准正交基,则
% \begin{asparaenum}[(1)]
%   \item $\overline{(A+B)}'=\overline{A}'+\overline{B}'$;
%   \item $\overline{(cA)}'=\overline{c}\overline{A}'$;
%   \item $\overline{(AB)}'=\overline{B}'\overline{A}'$;
%   \item $\overline{\overline{A}'}=A$;
%   \item 若$A$可逆,则$\overline{A}'$也可逆,且$(\overline{A}')^{-1}=\overline{A^{-1}}$.
% \end{asparaenum}

% 若$V$是无限维内积空间,(1)-(2)很容易证明,下面证(3)-(5):
% \begin{asparaenum}[(1)]
%   \setcounter{enumi}{2}%从3开始编号
%   \item $(\varphi\psi(\bm{\alpha}),\bm{\beta})=
%   (\psi(\bm{\alpha}),\varphi^*(\bm{\beta}))=
%   (\bm{\alpha},\psi^*\varphi^*(\bm{\beta})) \Longrightarrow
%     (\varphi\psi)^*=\psi^*\varphi^*.$
%   \item $(\varphi^*(\bm{\alpha}),\bm{\beta})=(\bm{\alpha},\varphi(\bm{\beta}))
%     \Longrightarrow (\varphi^*)^*=\varphi.$
%   \item $(\varphi\varphi^{-1}(\bm{\alpha}),\bm{\beta})=
%     (\varphi^{-1}(\bm{\alpha}),\varphi^*(\bm{\beta})).$
% \end{asparaenum}
% \end{proof}

% \begin{question}
%   设$V$是$n$维内积空间, $\varphi\in \mathscr{L}(V)$,

%   (1) 若$U$是$\varphi-$不变子空间,则$U^{\perp}$是$\varphi^*-$不变子空间;

%   (2) 若$\varphi$的全体特征值为$\lambda_1,\lambda_2,\cdots,\lambda_n$,
%   则$\varphi^*$的全体特征值为
%   $\overline{\lambda_1},\overline{\lambda_2},\cdots,\overline{\lambda_n}$.
% \end{question}

% \begin{proof}
%   (1) 任取$\bm{u}\in U, \bm{w}\in U^{\perp}$,则
%   \begin{align*}
%     & (\varphi(\bm{u}),\bm{w}) = (\bm{u},\varphi^*(\bm{w})) = 0,\\
%     \Longrightarrow & \varphi^*(\bm{w}) \in U^{\perp}, \forall \bm{w}\in U^{\perp}.
%   \end{align*}

%   (2) 取标准正交基$\{\bm{e}_1,\bm{e}_2,\cdots,\bm{e}_n\}$,
%   $\varphi$的表示矩阵为$A$, $\varphi^*$的表示矩阵为$\overline{A}'$,
%   由题意可知,
%   \[
%     |\lambda I_n - A| =
%     (\lambda-\lambda_1)(\lambda-\lambda_2)\cdots(\lambda-\lambda_n),
%   \]
%   则
%   \begin{align*}
%     & |\lambda I_n - \overline{A}'| = |\lambda I_n - \overline{A}|\\
%     \xlongequal {\text{令}\lambda=\overline{\mu}} &  \overline{|\mu I_n - A|}=
%     \overline{(\mu-\lambda_1)(\mu-\lambda_2)\cdots(\mu-\lambda_n)}\\
%     = & (\overline{\mu}-\overline{\lambda_1})(\overline{\mu}-\overline{\lambda_2})\cdots(\overline{\mu}-\overline{\lambda_n})\\
%     = & (\lambda_1-\mu)(\lambda_2-\mu)\cdots(\lambda_n-\mu).
%   \end{align*}
% \end{proof}

% \begin{example}
%   $V=U\perp U^{\perp}$, $E$是从$U$到$V$的正交投影变换,且
%   \[
%     (E(\bm{\alpha}),\bm{\beta}) = (\bm{\alpha},E(\bm{\beta})), \forall \bm{\alpha},\bm{\beta}\in V.
%   \]
%   则
%   \[
%     E^* = E.
%   \]
%   $E$称为{\heiti 自伴算子}.
% \end{example}

% \begin{example}
%   $V=M_n(\mathbb{R})$, Frobenius内积,且
%   $\varphi \in \mathscr{L}(V), \varphi(A) = PAQ, P,Q\in M_n(\mathbb{R}).$
% 则
% \begin{align*}
%   (\varphi(A),B) & = \Tr(\varphi(A)B')=\Tr((PAQ)B')\\
%   & = \Tr(AQB'P') = \Tr(A(P'BQ')') = (A,\varphi^*(B)).
% \end{align*}
% \end{example}

% \subsection{内积空间的同构,正交变换和酉变换}

% 设$V$为欧氏空间, $\{\bm{e}_1,\bm{e}_2,\cdots,\bm{e}_n\}$
% 为标准正交基.线性同构:
% \begin{align*}
%   \varphi: V & \longrightarrow \mathbb{R}^n(\text{标准内积})\\
%   \bm{\alpha}=\sum_{i=1}^n a_i\bm{e}_i & \longmapsto \bm{x} = \begin{pmatrix}
%     a_1\\
%     a_2\\
%     \vdots\\
%     a_n
%   \end{pmatrix}\\
%   \bm{\beta}=\sum_{i=1}^n b_i\bm{e}_i & \longmapsto \bm{y} = \begin{pmatrix}
%     b_1\\
%     b_2\\
%     \vdots\\
%     b_n
%   \end{pmatrix}
% \end{align*}
% \[
%   (\bm{\alpha},\bm{\beta}) = \bm{x}'\bm{y},
% \]
% \[
%   (\varphi(\bm{\alpha}),\varphi(\bm{\beta})) = \bm{x}'\bm{y}.
% \]
% \[
%   \therefore (\varphi(\bm{\alpha}),\varphi(\bm{\beta}))_{\mathbb{R}^n}
%   = (\bm{\alpha},\bm{\beta})_V, \forall \bm{\alpha},\bm{\beta} \in V.
% \]

% 设$V$为酉空间, $\{\bm{e}_1,\bm{e}_2,\cdots,\bm{e}_n\}$
% 为标准正交基.线性同构:
% \[
%   \varphi: V  \longrightarrow \mathbb{C}^n(\text{标准内积})
% \]
% \[
%   (\varphi(\bm{\alpha}),\varphi(\bm{\beta}))_{\mathbb{C}^n}
%   = (\bm{\alpha},\bm{\beta})_V, \forall \bm{\alpha},\bm{\beta} \in V.
% \]

% \begin{definition}
%   设$V, U$为实内积(复内积)空间, $\varphi: V \longrightarrow U$是线性映射,
%   若$\forall \bm{\alpha},\bm{\beta} \in V$有
%   \[
%     (\varphi(\bm{\alpha}),\varphi(\bm{\beta}))_U = (\bm{\alpha},\bm{\beta})_V,
%   \]
%   则称$\varphi$是保持内积的线性映射.若进一步,
%   $\varphi$为线性同构,则称$\varphi$为保积同构.
% \end{definition}

% \begin{notice}
%   (1) 线性空间的保积同构是等价关系;

%   (2) 保积同构的线性映射必为单射;
%   \begin{align*}
%     & \because \text{任取}\bm{\alpha}\in \ker\varphi,
%     \text{即}\varphi(\bm{\alpha})=\bm{0},
%     \text{则}(\bm{\alpha},\bm{\alpha}) =
%       (\varphi(\bm{\alpha}),\varphi(\bm{\alpha})) = 0\\
%     & \therefore \text{由正定性可知}\bm{\alpha} = \bm{0}.
%   \end{align*}
% \end{notice}

% \begin{example}
%   嵌入映射
%   \begin{align*}
%     \varphi: \mathbb{R}^2 & \longrightarrow \mathbb{R}^3(\text{标准内积})\\
%     (a,b)' & \longmapsto (a,b,0)'
%   \end{align*}
%   这个映射只能填满$x-y$平面,不能填满整个$\mathbb{R}^3$.

% \begin{question}
%   保积映射必是保范(保距)映射.
% \end{question}  
% \begin{proof}
%   令$\bm{\alpha}=\bm{\beta}$,
%   \begin{align*}
%     & \Vert\varphi(\bm{\alpha})\Vert^2=(\varphi(\bm{\alpha}),\varphi(\bm{\alpha}))=
%       (\bm{\alpha},\bm{\alpha})=\Vert\bm{\alpha}\Vert^2\\
%     \Longrightarrow & \Vert\varphi(\bm{\alpha})\Vert=\Vert\bm{\alpha}\Vert, \forall \bm{\alpha}.
%   \end{align*}
% \end{proof}

% \begin{question}
%   设$\varphi: V \longrightarrow U$是实(复)内积空间上的线性映射,
%   若$\varphi$保持范数,则$\varphi$保持内积.
% \end{question}

% \begin{proof}
%   \begin{align}
%     & \Vert\bm{\alpha}+\bm{\beta}\Vert^2 = \Vert\bm{\alpha}\Vert^2+
%       \Vert\bm{\beta}\Vert^2+(\bm{\alpha},\bm{\beta})+
%       (\bm{\beta},\bm{\alpha})\label{eq:isomorphism1}\\
%     & \Vert\bm{\alpha}-\bm{\beta}\Vert^2 = \Vert\bm{\alpha}\Vert^2+
%       \Vert\bm{\beta}\Vert^2-(\bm{\alpha},\bm{\beta})-
%       (\bm{\beta},\bm{\alpha})\label{eq:isomorphism2}
%   \end{align}
%   若$V$为欧式空间,则$\eqref{eq:isomorphism1}$式-$\eqref{eq:isomorphism2}$式,得
%   \[
%     (\bm{\alpha},\bm{\beta}) = \frac{1}{4}(\Vert\bm{\alpha}+\bm{\beta}\Vert^2-\Vert\bm{\alpha}-\bm{\beta}\Vert^2).
%   \]
%   则
%   \begin{align*}
%     & (\varphi(\bm{\alpha}),\varphi(\bm{\beta})) =
%       \frac{1}{4}(\Vert\varphi(\bm{\alpha})+\varphi(\bm{\beta})\Vert^2-\Vert\varphi(\bm{\alpha})-\varphi(\bm{\beta})\Vert^2)\\
%     = & \frac{1}{4}(\Vert\varphi(\bm{\alpha}+\bm{\beta})\Vert^2-\Vert\varphi(\bm{\alpha}-\bm{\beta})\Vert^2)\\
%     = & \frac{1}{4}(\Vert\bm{\alpha}+\bm{\beta}\Vert^2-\Vert\bm{\alpha}-\bm{\beta}\Vert^2)\\
%     = & (\bm{\alpha},\bm{\beta}).
%   \end{align*}
%   若$V$为酉空间,则$\eqref{eq:isomorphism1}$式$-\eqref{eq:isomorphism2}$式,得
%   \begin{align}
%     & \Vert\bm{\alpha}+\bm{\beta}\Vert^2-\Vert\bm{\alpha}-\bm{\beta}\Vert^2 =
%     2((\bm{\alpha},\bm{\beta})+2(\bm{\beta},\bm{\alpha}))\label{eq:isomorphism3}\\
%     & \Vert\bm{\alpha}+\bm{i\beta}\Vert^2-\Vert\bm{\alpha}-\bm{i\beta}\Vert^2 =
%     -2((\bm{\alpha},\bm{\beta})+2i(\bm{\beta},\bm{\alpha}))\label{eq:isomorphism4}
%   \end{align}
%   将$\eqref{eq:isomorphism3}$式$+\eqref{eq:isomorphism4}$式$\times i$,得
%   \[
%     (\bm{\alpha},\bm{\beta}) =
%     \frac{1}{4}(\Vert\bm{\alpha}+\bm{\beta}\Vert^2-
%     \Vert\bm{\alpha}-\bm{\beta}\Vert^2+
%     i\Vert\bm{\alpha}+i\bm{\beta}\Vert^2-
%     i\Vert\bm{\alpha}-i\bm{\beta}\Vert^2).
%   \]
%   则
%   \begin{align*}
%     & (\varphi(\bm{\alpha}),\varphi(\bm{\beta}))\\
%       = & \frac{1}{4}(\Vert\varphi(\bm{\alpha})+\varphi(\bm{\beta})\Vert^2-
%       \Vert\varphi(\bm{\alpha})-\varphi(\bm{\beta})\Vert^2+
%       i\Vert\varphi(\bm{\alpha})+i\varphi(\bm{\beta})\Vert^2-
%           i\Vert\varphi(\bm{\alpha})-i\varphi(\bm{\beta})\Vert^2)\\
%     = & \frac{1}{4}(\Vert\varphi(\bm{\alpha}+\bm{\beta})\Vert^2-
%         \Vert\varphi(\bm{\alpha}-\bm{\beta})\Vert^2+
%         i\Vert\varphi(\bm{\alpha}+i\bm{\beta})\Vert^2-
%         i\Vert\varphi(\bm{\alpha}-i\bm{\beta})\Vert^2)\\
%     = & \frac{1}{4}(\Vert\bm{\alpha}+\bm{\beta}\Vert^2-
%         \Vert\bm{\alpha}-\bm{\beta}\Vert^2+
%         i\Vert\bm{\alpha}+i\bm{\beta}\Vert^2-
%         i\Vert\bm{\alpha}-i\bm{\beta}\Vert^2)\\
%     = & (\bm{\alpha},\bm{\beta}).
%   \end{align*}
% \end{proof}
% \end{example}

% \begin{notice}
%   由以上两个命题可知,保积映射和保范(保距)映射等价.
% \end{notice}

% \begin{theorem}\label{thm:isomorphism}
%   $\varphi:V^n \longrightarrow U^n$是$n$维实(复)内积空间上的线性映射,
%   则下列命题等价:

%   (1) $\varphi$保持内积;

%   (2) $\varphi$是保积同构;

%   (3) $\varphi$将$V$的任一组标准正交基映为$U$的一组标准正交基;

%   (4) $\varphi$将$V$的某一组标准正交基映为$U$的一组标准正交基
% \end{theorem}  

% \begin{proof}
%     (1) $\Longrightarrow$ (2):
%     \[
%       \varphi\text{保持内积} \Longrightarrow \ker\varphi=0.
%     \]
%     由维数公式可知:
%     \[
%       \dim\IM\varphi = \dim V = \dim U
%     \]
%     从而$\IM\varphi = U$, 即$\varphi$是满射,
%     于是$\varphi: V \longrightarrow U$是线性同构.

%     (2) $\Longrightarrow$ (3):
%     任取$V$的标准正交基$\{\bm{e}_1, \bm{e}_2, \cdots, \bm{e}_n\}$,则
%     \[
%        (\varphi(\bm{e}_i),\varphi(\bm{e}_j)) = (\bm{e}_i,\bm{e}_j) = \delta_{ij}, \forall i,j
%      \]
%       即$\varphi(\bm{e}_1), \varphi(\bm{e}_2), \cdots, \varphi(\bm{e}_n)$是$U$的一组标准正交基.
  
%       (3) $\Longrightarrow$ (4): 显然.
  
%       (4) $\Longrightarrow$ (1): 设$\{\bm{e}_1, \bm{e}_2, \cdots, \bm{e}_n\}$
%       是$V$的标准正交基且
%       $\{\varphi(\bm{e}_1), \varphi(\bm{e}_2), \cdots, \varphi(\bm{e}_n)\}$
%       是$U$的标准正交基,任取
%       \begin{align*}
%          & \bm{\alpha}=\sum a_i\bm{e}_i \longleftrightarrow
%           \bm{x}= (a_1,a_2,\cdots,a_n)'\\
%         & \bm{\beta}=\sum b_i\bm{e}_i \longleftrightarrow
%           \bm{y}= (b_1,b_2,\cdots,b_n)'
%       \end{align*}
%       则
%       \[
%         \varphi(\bm{\alpha}) = \sum a_i\varphi(\bm{{e}_i}),
%         \varphi(\bm{\beta}) = \sum b_i\varphi(\bm{e}_i).
%       \]
%       \begin{align}
%         & (\varphi(\bm{\alpha}),\varphi(\bm{\beta})) =
%           (\sum a_i\varphi(\bm{e}_i),\sum b_i\varphi(\bm{e}_i)) =
%           \sum a_i\overline{b_i}\label{eq:isomorphism5}\\
%         & (\bm{\alpha},\bm{\beta}) =
%           (\sum a_i\bm{e}_i,\sum b_i\bm{e}_i) =
%           \sum a_i\overline{b_i}\label{eq:isomorphism6}
%       \end{align}
%       由$\eqref{eq:isomorphism5}$式$=\eqref{eq:isomorphism6}$式可知,
%       $\varphi$保持内积.
% \end{proof}

% \begin{deduction}
%   设$V, U$为实(复)内积空间, 则存在保积同构
%   \[
%     \varphi: V \longrightarrow U \Longleftrightarrow \dim V = \dim U. 
%     \]
% \end{deduction}

% \begin{proof}
%   必要性显然.下面证明充分性:
%   设$\dim V = \dim U = n$,
%   取$V$的标准正交基$\{\bm{e}_1, \bm{e}_2, \\
%   \cdots,\bm{e}_n\}$,
%   取$U$的标准正交基$\{\bm{f}_1, \bm{f}_2, \cdots, \bm{f}_n\}$,
%   定义一个线性映射:
%   \begin{align*}
%     \varphi: V & \longrightarrow U\\
%     \bm{e}_i & \longmapsto \bm{f}_i
%   \end{align*}
%   扩张为$\varphi\in \mathscr{L}(V, U)$.
%   由定理\ref{thm:isomorphism}(4)可知,
%   $\varphi$是保积同构.
% \end{proof}

% \begin{definition}
%   设$\varphi$是内积空间$V$上的线性算子,
%   若$\varphi$保持内积,则
%   \[
%     \text{称$\varphi$为}
%     \begin{cases}
%       \text{正交变换}\quad & \text{欧氏空间(实内积空间)},\\
%       \text{酉变换}\quad & \text{酉空间(复内积空间)}.
%     \end{cases}
%   \]
%   由定理\ref{thm:accompany}可知,
%   $\varphi$为可逆算子.
% \end{definition}

% \begin{theorem}\label{thm:orthogonal2}
%   设$\varphi$为有限维空间$V$上的线性算子,则
%   \[
%     \varphi\text{为正交算子(酉算子)} \Longleftrightarrow
%     \varphi\text{可逆且}\varphi^*= \varphi^{-1}.
%   \]
% \end{theorem}

% \begin{proof}
%   必要性($\Rightarrow$)
%   \[
%     (\varphi(\bm{\alpha}),\bm{\beta}) =
%     (\varphi(\bm{\alpha}),\varphi(\varphi^{-1}(\bm{\beta}))) =
%     (\bm{\alpha},\varphi^{-1}(\bm{\beta})) =
%     (\bm{\alpha},\varphi^*(\bm{\beta})).
%   \]

%   充分性($\Leftarrow$)
%   \[
%     (\varphi(\bm{\alpha}),\varphi(\bm{\beta})) =
%     (\bm{\alpha},\varphi^*(\varphi{\bm{\beta}})) =
%     (\bm{\alpha},\varphi^{-1}(\varphi(\bm{\beta}))) =
%     (\bm{\alpha},\bm{\beta}).
%   \]
% \end{proof}

% \begin{definition}
%   设$A\in M_n(\mathbb{R})$, 若$A'=A^{-1}$,
%   则称$A$为正交矩阵;
%   设$C\in M_n(\mathbb{C})$, 若$\overline{C}'=C^{-1}$,
%   则称$C$为酉矩阵.
% \end{definition}

% \begin{theorem}
%   设$\varphi$为$n$维内积空间$V$上的线性算子,则
%   \[
%     \varphi\text{为正交算子(酉算子)} \Longleftrightarrow
%     \varphi\text{在任(某)一组标准正交基下的表示矩阵为正交矩阵(酉矩阵)}.
%   \]
% \end{theorem}

% \begin{proof}
%   任取$V$的标准正交基$\{\bm{e}_1, \bm{e}_2, \cdots, \bm{e}_n\}$,
%   设$\varphi$的表示矩阵为$A$, 则$\varphi^*$在同一组
%   基下的表示矩阵为
%   $\begin{cases} A' \quad & V\text{为欧氏空间},\\
%     \overline{A}' \quad & V\text{为酉空间}. \end{cases}$

%   $\varphi$为正交算子(酉算子),由定理\ref{thm:orthogonal2}可知,
%   $\varphi$可逆且$\varphi^*=\varphi^{-1}$.由线性变换及其表示矩阵
%   之间的一一对应性可知,$\varphi^*$与$\varphi^{-1}$
%   在同一组标准正交基下有相同的表示矩阵,即
%   \[\begin{cases}
%     A' = A^{-1} \text{(即$A$为正交矩阵)}\quad & V\text{为欧氏空间时},\\
%     \overline{A}' = A^{-1} \text{(即$A$为酉矩阵)} \quad & V\text{为酉空间时}.
%   \end{cases}\]
% \end{proof}

% \begin{theorem}
%   设$A\in M_n(\mathbb{R})$,则下列命题等价:

%   (1) $A$为正交矩阵;

%   (2) $A$的$n$个行分块向量是$\mathbb{R}_n$(标准内积)的一组标准正交基;

%   (3) $A$的$n$个列分块向量是$\mathbb{R}^n$(标准内积)的一组标准正交基.
% \end{theorem}

% \begin{proof}
%   这里只简要证明(2)-(3).

%   (2) 设$A=\begin{pmatrix} \bm{\alpha}_1\\\bm{\alpha}_2\\\cdots\\\bm{\alpha}_n\end{pmatrix}$,
%     其中, $\bm{\alpha}_i$是$A$的行分块向量.依题意,
%     \[
%       I_n=AA'=A=\begin{pmatrix} \bm{\alpha}_1\\\bm{\alpha}_2\\\cdots\\\bm{\alpha}_n\end{pmatrix}
%       (\bm{\alpha}_1,\bm{\alpha}_2,\cdots,\bm{\alpha}_n),
%     \]
%     \[
%       a_{ij}=\bm{\alpha}_i\cdot\bm{\alpha}'_j=(\bm{\alpha}_i,\bm{\alpha}_j)_{\mathbb{R}_n}=\delta_{ij}
%     \]
%     上式中,$a_{ij}$即$A$的第$(i,j)$元素.

%     (3) 设$A=(\bm{\beta}_1,\bm{\beta}_2,\cdots,\bm{\beta}_n)$,
%     其中, $\bm{\beta}_i$是$A$的列分块向量.依题意,
%     \[
%       I_n=A'A=\begin{pmatrix}\bm{\beta}'_1\\\bm{\beta}'_2\\\cdots\\\bm{\beta}'_n\end{pmatrix},
%     \]
%     \[
%       a_{ij}= \bm{\beta}'_i\cdot\bm{\beta}_j=(\bm{\beta}_i,\bm{\beta}_j)_{\mathbb{R}^n}=\delta_{ij}.
%       \]
%     \end{proof}

% \begin{theorem}
%   设$A\in M_n(\mathbb{C})$,则下列命题等价:

%   (1) $A$为酉矩阵;

%   (2) $A$的$n$个行分块向量是$\mathbb{C}_n$(标准内积)的一组标准正交基;

%   (3) $A$的$n$个列分块向量是$\mathbb{C}^n$(标准内积)的一组标准正交基.
% \end{theorem}

% \begin{proof}
%   证明和上个定理完全一致,下面是简要证明思路:
%   \[
%   A\text{为酉矩阵} \Longleftrightarrow A\overline{A}'=I_n \Longleftrightarrow 
%   \overline{A}'A=I_n.
%   \]
% \end{proof}

% \begin{example}
%   (1) 正交矩阵是特殊的酉矩阵;

%   (2) $I_n$是正交矩阵或酉矩阵;

%   (3) $D=\{d_1,d_2,\cdots,d_n\}\in \mathbb{R}$是正交阵 
%   $ \Longleftrightarrow d_i=\pm 1, 1\leq i \leq n$;

%   (4) 二阶正交阵分类:
  
%   (I) 旋转$\theta$角度
%   \[
%   \begin{pmatrix}
%     \cos\theta & -\sin\theta\\\sin\theta & \cos\theta
%   \end{pmatrix}, |A| = 1;
%   \] 
  
%   (II)反射
%   \[
%   \begin{pmatrix}
%     \cos\theta & \sin\theta\\ \sin\theta & -\cos\theta
%   \end{pmatrix}, |A| = -1.
%   \]
% \end{example}

% \begin{question}
%   (1) 酉阵的行列式的模长为$1$,
%   正交阵的行列式值为$\pm 1$;

%   (2) 酉阵的特征值模长为$1$.
% \end{question}

% \begin{proof}
%   (1)
%   \begin{align*}
%     & A\overline{A}'=I_n\Longrightarrow 1 = |I_n| = |A\overline{A}'| = |\det(A)|^2\\
%     \Longrightarrow & |\det(A)|=1
%   \end{align*}

%   (2) 设
%   \begin{equation}\label{eq:isomorphism7}
%     A\bm{\alpha}=\lambda_0\bm{\alpha}, 0 \neq \bm{\alpha}\in \mathbb{C}^n, \lambda_n\in \mathbb{C},
%   \end{equation}
%   则
%   \begin{equation}\label{eq:isomorphism8}
%     \overline{\bm{\alpha}}'\overline{A}' = \overline{\lambda}_0\overline{\bm{\alpha}}'
%   \end{equation}
%   将\eqref{eq:isomorphism8}式左右两边分别左乘到\eqref{eq:isomorphism7}式左右两边,得:
%   \begin{equation*}
%     \overline{\bm{\alpha}}'\bm{\alpha} = |\lambda_0|^2\overline{\bm{\alpha}}'\bm{\alpha},
%   \end{equation*}
%   从而$|\lambda_0|=1$.
% \end{proof}

% \begin{theorem}[QR分解定理]
%   设$A$为$n$阶实(复)方阵,则
%   \begin{equation*}
%     A = QR
%   \end{equation*}
%   其中$Q$为正交阵(酉阵),$R$为上三角阵且
%   主对角元素全大于等于$0$.进一步,若
%   $A$非异,则上述分解必唯一.
% \end{theorem}
% \begin{proof}
%   设$A=(\bm{u}_1,\bm{u}_2,\cdots,\bm{u}_n)$,其中$\bm{u}_i$是$A$的列分块向量.
%   作一个推广的Gram-Schmidt正交化过程(因为不能确定$\bm{u}_i$是否线性无关):
%   目标是最终得到一个列向量组$\{\bm{w}_1,\bm{w}_2,\cdots,\bm{w}_n\}$
%   是正交向量组,其中$\bm{w}_i$是$\bm{0}$或者是单位向量.

%   归纳法: 设$\{\bm{w}_1,\bm{w}_2,\cdots,\bm{w}_{k-1}\}$已经构建好,
%   现求$\bm{w}_k$.令
%   \[
%     \bm{v}_k = \bm{u}_k - \sum_{j=1}^{k-1}(\bm{u}_k,\bm{w}_j)\bm{w}_j,
%   \]
% \[
% \left.\begin{cases}
%   \text{若}\bm{v}_k = \bm{0},\text{则令}\bm{w}_k = \bm{0};\\
%   \text{若}\bm{v}_k \neq \bm{0},\text{则令}\bm{w}_k = \cfrac{\bm{v}_k}{\Vert\bm{v}_k\Vert};
% \end{cases}\right\}\text{得到正交向量组}\{\bm{w}_1,\bm{w}_2,\cdots,\bm{w}_k\}.
% \]
% 因此,
% \[
% \bm{u}_k=\sum_{j=1}^{k}(\bm{u}_k,\bm{w}_j)\bm{w}_j+\Vert\bm{u}_k\Vert\bm{w}_k, (1\leq k \leq n).
% \]
% \[ 
% A=(\bm{u}_1,\bm{u}_2,\cdots,\bm{u}_n)=(\bm{w}_1,\bm{w}_2,\cdots,\bm{w}_n)\begin{pmatrix}
%   \Vert\bm{v}_1\Vert & * & * &*\\
%   & \Vert\bm{v}_2\Vert & * & *\\
%   & & \ddots &*\\
%   & & & \Vert\bm{v}_n\Vert
% \end{pmatrix}.
% \]
% 令上式中的矩阵为$R$,则
% 若某个$\bm{w}_i=\bm{0}$,则$R$的第$i$行全为$0$.
% 设$\bm{w}_1,\bm{w}_2,\cdots,\bm{w}_n$中有$r$个非零向量
% $\bm{w}_{i_1},\bm{w}_{i_2},\cdots,\bm{w}_{i_r}$,
% 将其扩张为全空间$\mathbb{R}^n$的一组标准正交基
% $\{\widetilde{\bm{w}_1},\widetilde{\bm{w}_2},\cdots,\widetilde{\bm{w}_n}\}$,
% 其中$\widetilde{\bm{w}_j} = \bm{w}_j, j=i_1,i_2,\cdots,i_r$,
% 令$(\widetilde{\bm{w}_1},\widetilde{\bm{w}_2},\cdots,\widetilde{\bm{w}_n})=Q$,
% 则$Q$为正交阵,且
% \[
% A=QR.
% \]
% \end{proof}

% \subsection{自伴随算子}


\end{document}
%%% Local Variables:
%%% mode: LaTeX
%%% TeX-master: "../main"
%%% End:
